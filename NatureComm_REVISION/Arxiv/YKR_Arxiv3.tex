\documentclass[twocolumn,preprintnumbers,amsmath,amssymb,superscriptaddress]{revtex4}
%\usepackage[pdftex]{graphicx}

\usepackage{amsmath,amsfonts,amssymb}
\usepackage[english]{babel}
\usepackage[latin1]{inputenc}
\usepackage[T1]{fontenc}
\usepackage{color}
\usepackage{float}
\usepackage{verbatim}
\usepackage{graphicx}
\usepackage{bm}
\usepackage{mathtools}
\usepackage{stmaryrd}
\usepackage{anyfontsize}


%\usepackage{epstopdf}
%\usepackage{array}
%\usepackage{tabularx}
%\usepackage{multirow}
\usepackage{color}
%\usepackage{multibox}
%\usepackage{rotating}
%\usepackage{lineno}
%\usepackage[left]{lineno}
%\usepackage[comma,sort&compress]{natbib}
%\usepackage{authblk}
%\usepackage{multicol}

%\bibliographystyle{ieeetr}

\usepackage{bibunits}

%\linenumbers
%\setlength\linenumbersep{3pt}

\begin{document}


\author{Justin D. Yeakel} \affiliation{School of Natural Sciences, University
  of California, Merced, Merced, CA 95340, USA}

\author{Christopher P. Kempes} \affiliation{The Santa Fe Institute, 1399 Hyde
  Park Road, Santa Fe, NM 87501, USA}

\author{Sidney Redner} \affiliation{The Santa Fe Institute, 1399 Hyde Park
  Road, Santa Fe, NM 87501, USA}

\title{The dynamics of starvation and recovery}%: Eco-evolutionary feedbacks}



\begin{abstract} %250 words
The eco-evolutionary dynamics of species are fundamentally linked to the energetic constraints of its constituent individuals. Of particular importance is the interplay between reproduction and the dynamics of starvation and recovery. To elucidate this interplay, we introduce a nutritional state-structured model that incorporates two classes of consumer: nutritionally replete, reproducing consumers, and undernourished, non-reproducing consumers. We obtain strong constraints on starvation and recovery rates by deriving allometric scaling relationships and find that population dynamics are typically driven to a steady state. Moreover, these rates fall within a `refuge' in parameter space, where the probability of population extinction is minimized. We also show that our model provides a natural framework to predict maximum mammalian body size by determining the relative stability of an otherwise homogeneous population to a competing population with altered percent body fat. This framework provides a principled mechanism for a selective driver of Cope's rule.
\end{abstract}

\maketitle

\begin{bibunit}[unsrt]

  The behavioral ecology of all organisms is influenced by their energetic
  states, which directly impacts how they invest reserves in uncertain
  environments.  Such behaviors are generally manifested as tradeoffs between
  investing in somatic maintenance and growth, or allocating energy towards
  reproduction~\citep{Martin:1987dl,Kirk:1997cc,Kempes:2012hy}.  The timing of
  these behaviors responds to selective pressure, as the choice of the
  investment impacts future
  fitness~\citep{Mangel:1988uaa,Mangel:2014kz,Yeakel:2013hi}.  The influence of
  resource limitation on an organism's ability to maintain its nutritional
  stores may lead to repeated delays or shifts in reproduction over the course
  of an organism's life.

  The balance between (a) somatic growth and maintenance, and (b) reproduction depends on resource availability~\citep{Morris:1987eo}.
  For example, reindeer invest less in calves born after harsh winters (when the mother's energetic state is depleted) than in calves born after moderate winters~\citep{Tveraa:2003fq}.
  Many bird species invest differently in broods during periods of resource scarcity~\citep{Daan:1988va,Jacot:2009dv}, sometimes delaying or even foregoing reproduction for a breeding season~\citep{Martin:1987dl,Stearns:1989ip,Barboza:2002in}.
  Even freshwater and marine zooplankton have been observed to avoid reproduction under nutritional stress~\citep{Threlkeld:1976ih}, and those that do reproduce have lower survival rates~\citep{Kirk:1997cc}.
  Organisms may also separate maintenance and growth from reproduction over space and time: many salmonids, birds, and some mammals return to migratory breeding grounds to reproduce after one or multiple seasons in resource-rich environments where they accumulate reserves~\citep{Weber:1998jg,Mduma:1999cp,Moore:2014hi}.

  Physiology also plays an important role in regulating reproductive expenditures during periods of resource limitation.
  Many mammals (47 species in 10 families) exhibit delayed implantation, whereby females postpone fetal development until nutritional reserves can be accumulated~\citep{Mead:1989dt,Sandell:1990kw}.
  Many other species (including humans) suffer irregular menstrual cycling and higher abortion rates during periods of nutritional stress~\citep{Bulik:1999eo,Trites:2003cc}.
  In the extreme case of unicellular organisms, nutrition directly controls growth to a reproductive state \citep{Glazier:2009hq,Kempes:2012hy}. The existence of so many independently evolved mechanisms across such a diverse suite of organisms highlights the near-universality of the fundamental tradeoff between somatic and reproductive investment.

  Including individual energetic dynamics~\citep{Kooi2000} in a
  population-level framework~\citep{Kooi2000,Sousa:2010ez} is
  challenging~\citep{Diekmann:2010da}.  A common simplifying approach is the
  classic Lotka-Volterra (LV) model, which assumes that consumer population
  growth rate depends linearly on resource density~\citep{murdoch:2003}. Here,
  we introduce an alternative approach---the Nutritional State-structured Model
  (NSM)---that accounts for resource limitation via explicit starvation. In
  contrast to the LV model, the NSM incorporates two consumer states: hungry
  and full, with only the former susceptible to mortality and only the latter
  possessing sufficient energetic reserves to reproduce.  Additionally, we
  incorporate allometrically derived constraints on the time scales for
  reproduction~\citep{Kempes:2012hy}, starvation, and recovery.  Our model
  makes several basic predictions: (i) the dynamics are typically driven to a
  refuge far from cyclic behavior and extinction risk, (ii) the steady-state
  conditions of the NSM accurately predict the measured biomass densities for
  mammals described by Damuth's law~\citep{Damuth:1987kr,allen2002,enquist1998,Pedersen:2017he},
    %  energetic equivalence principle, 
  (iii) there is an allometrically constrained upper-bound for mammalian body size, and
  (iv) the NSM provides a selective mechanism for the evolution of larger body size, known as Cope's rule~\citep{Alroy:1998p1594,Clauset:2009fh,Smith:2010p3442,Saarinen:2014br}.

  % Though straightforward conceptually, incorporating the energetic dynamics of
  % individuals~\citep{Kooi2000} into a population-level
  % framework~\citep{Kooi2000,Sousa:2010ez} presents mathematical
  % obstacles~\citep{Diekmann:2010da}.  An alternative approach involves modeling
  % the macroscale relations that guide somatic versus reproductive investment in
  % a consumer-resource system.  {\color{red}For example, macroscale
  %   Lotka-Volterra models assume that the growth rate of the consumer
  %   population depends on resource density, thus incorporating the requirement
  %   of resource availability for reproduction~\citep{murdoch:2003}.

  % In this work, we adopt an alternative approach to account for resource
  % limitation through the consequences of starvation.  Namely, in contrast to
  % the Lotka-Volterra model, we incorporate two states: hungry individuals and
  % individuals with sufficient energetic reserves to reproduce.}  Such a
  % constraint leads to reproductive time lags due to some members of the
  % population going hungry and then recovering.  Additionally, we incorporate
  % the idea that reproduction is strongly constrained allometrically
  % \citep{Kempes:2012hy}, and is not generally linearly related to resource
  % density.
  % As we shall show, these constraints influence the ensuing population dynamics in dramatic ways.\\

  \noindent \paragraph*{{\bf Nutritional state-structured model (NSM).}}
  We begin by defining the nutritional state-structured population model, where the consumer population is partitioned into two states: (a) an
  energetically replete (full) state $F$, where the consumer reproduces at a
  constant rate $\lambda$ and does not die from starvation, and (b) an
  energetically deficient (hungry) state $H$, where the consumer does not
  reproduce but dies by starvation at rate $\mu$. The dynamics of the
  underlying resource $R$ are governed by logistic growth with an intrinsic
  growth rate $\alpha$ and a carrying capacity $C$. The rate at which consumers
  transition between states and consume resources is dependent on their number,
  the abundance of resources, the efficiency of converting resources into
  metabolism, and how that metabolism is partitioned between maintenance and
  growth purposes.  We provide a physiologically and energetically mechanistic
  model for each of these dynamics and constants (see the Supplementary Information
  (SI)), and show that the system produces a simple non-dimensional form which
  we describe below.

  \begin{figure}
  \centering
  \includegraphics[width=0.45\textwidth]{fig_FixedPoint2-eps-converted-to.pdf}
  \caption{\small{ The transcritical (TC; dashed line) and Hopf bifurcation (solid line) as a
    function of the starvation rate $\sigma$ and recovery rate $\rho$ for a 100g consumer.  These
    bifurcation conditions separate parameter space into unphysical (left of the TC), cyclic,
    and steady state dynamic regimes.  The colors show the steady state densities for the energetically replete consumers $F^*$.  
    % Steady state densities for the energetically deficient consumers $H^*$ (not shown)
    % scale with those for $F^*$.  
  }\label{fig:fp}}
  \end{figure}

  Consumers transition from the full state $F$ to the hungry state $H$ at a
  rate $\sigma$---the starvation rate---and also in proportion to the absence
  of resources $(1-R)$ (the maximum resource density has been non
  dimensionalized to 1; see SI).  Conversely, consumers recover from state $H$
  to state $F$ at rate $\xi \rho$ and in proportion to $R$, where $\xi$
  represents a ratio between maximal resource consumption and the carrying
  capacity of the resource. %and $\approx 2$.
  The resources that are eaten by hungry consumers (at rate $\rho R + \delta$)
  account for their somatic growth ($\rho R$) and maintenance ($\delta$).  Full
  consumers eat resources at a constant rate $\beta$ that accounts for maximal
  maintenance and somatic growth (see the SI for mechanistic derivations of
  these rates from resource energetics).
  %This inequality accounts for hungry consumers requiring more resources to replace lost body tissue.
  The NSM represents an ecologically motivated fundamental extension of the
  idealized starving random walk model of foraging, which focuses on resource
  depletion, to include reproduction and resource
  replenishment~\citep{Benichou:2014wu,Benichou:2016wl,Chupeau:2016jf}, and is
  a more general formulation than previous models that incorporate
  starvation~\citep{Persson:1998hz}.

  In the mean-field approximation, in which the consumers and resources are
  perfectly mixed, their densities are governed by the rate equations

  \begin{align}
  \label{eq:system}
  \begin{split}
  \dot{F} &= \lambda F + \xi \rho RH - \sigma \left(1-R\right)F,  \\
  \dot{H} &= \sigma \left(1-R\right)F - \xi \rho RH - \mu H,  \\
  \dot{R} &= \alpha \left(1-R\right)R -\left(\rho R+\delta\right)H-\beta F.
  \end{split}
  \end{align}

  This system of nondimensional equations follows from a set of first-principle
  relationships for resource consumption and growth (see the SI for a full derivation and the dimensional form).
  Notice that the total consumer density $F+H$ evolves according to $\dot{F}+\dot{H}=\lambda F-\mu H$.
  This resembles the equation of motion for the predator density in the LV model~\citep{murray2011mathematical}, except that the resource density does not appear in the growth term.
  % As discussed above, the attributes of reproduction and mortality have been explicitly apportioned to the full and hungry consumers, respectively, so that the growth in the total density is decoupled from the resource density.
  The rate of reproduction is independent of resource density because the full
  consumer partitions a constant amount of energy towards reproduction, whereas
  a hungry consumer partitions no energy towards reproduction.  Similarly, the
  consumer maintenance terms ($\delta H$ and $\beta F$) are also independent of
  resource density because they represent a minimal energetic requirement for
  consumers in the $H$ and $F$ state, respectively.

  % It follows that model predictions are robust only when $R$ is of the order of 1, which holds for all cases that we explore.


  \noindent \paragraph*{ {\bf Steady states of the NSM.}} From the single
  internal fixed point (Eq.~\eqref{eq:ss}, see Methods), an obvious constraint
  on the NSM is that the reproduction rate $\lambda$ must be less than the
  starvation rate $\sigma$, so that the consumer and resource densities are
  positive.
  %In fact, when the resource density $R=0$, the rate equation for $F$ gives exponential growth of $F$ for $\lambda>\sigma$.
  The condition $\sigma = \lambda$ represents a transcritical (TC)
  bifurcation~\citep{Strogatz:2008wo} that demarcates a physical from an
  unphysical (negative steady-state densities) regime.  The biological
  implication of the constraint $\lambda<\sigma$ has a simple
  interpretation---the rate at which a macroscopic organism loses mass due to
  lack of resources is generally much faster than the rate of reproduction.  As
  we will discuss below, this inequality is also a natural consequence of
  allometric constraints~\citep{Kempes:2012hy} for organisms within empirically
  observed body size ranges. %(Fig.~\ref{fig:gvs})

  In the physical regime of $\lambda<\sigma$, the fixed point \eqref{eq:ss} may either be a stable node or a limit cycle (Fig.~\ref{fig:fp}).
  In continuous-time systems, a limit cycle arises when a pair of complex conjugate eigenvalues crosses the imaginary axis to attain positive real parts~\citep{GuckHolmes}.
  This Hopf bifurcation is defined by ${\rm Det}({\bf S}) = 0$, with $\bf S$ the Sylvester matrix, which is composed of the coefficients of the characteristic polynomial of the Jacobian matrix~\citep{Gross:2004p2428}.
  As the system parameters are tuned to be within the stable regime, but close to the Hopf bifurcation, the amplitude of the transient cycles becomes large.
  Given that ecological systems are constantly being perturbed~\citep{Hastings:2001jh}, the onset of transient cycles, even though they decay with time in the mean-field description, can increase extinction risk~\citep{Neubert:1997wk,Caswell:2005eo,Neubert:2009td}.
  %Thus the distance of a system from the Hopf bifurcation provides a measure of its persistence.

  When the starvation rate $\sigma\gg\lambda$, a substantial fraction of the
  consumers are driven to the hungry non-reproducing state.  Because
  reproduction is inhibited, there is a low steady-state consumer density and a
  high steady-state resource density.  However, if $\sigma/\lambda\to 1$ from
  above, the population is overloaded with energetically-replete (reproducing)
  individuals, thereby promoting transient oscillations between the consumer
  and resource densities (Fig.~\ref{fig:fp}).  If the starvation rate is low
  enough that the Hopf bifurcation is crossed, these oscillations become
  stable.  This threshold occurs at higher values of the starvation rate as the recovery rate $\rho$ increases, such that the range of parameter space giving rise to cyclic dynamics also increases with higher recovery rates.\\

  % Whereas the relation between consumer growth rate $\lambda$ and the starvation rate $\sigma$ defines an absolute bound of biological feasibility---the TC bifurcation---$\sigma$ also determines the sensitivity of the consumer population to changes in resource density.
  % When $\sigma\gg\lambda$, the steady-state population density is small, thereby increasing the risk of stochastic extinction.
  % On the other hand, as $\sigma$ decreases, the system will ultimately be poised either near the TC or the Hopf bifurcation (Fig.~\ref{fig:fp}).
  % If the recovery rate $\rho$ is sufficiently small, the TC bifurcation is reached and the resource eventually is eliminated.
  % If $\rho$ exceeds a threshold value, cyclic dynamics will develop as the Hopf bifurcation is approached.

  \begin{figure}
  \centering
  \includegraphics[width=0.4\textwidth]{Growth-trajectory-diagram-eps-converted-to.pdf}
  \caption{\small{ The growth trajectory over absolute time of an individual organism as a function of body mass.  
  Initial growth follows the black trajectory to an energetically replete reproductive adult mass of $m=\epsilon_\lambda M$ (see Methods). %which we assume is 95\% asymptotic mass $M$.  
  Starvation follows the red trajectory to $m = \epsilon_\sigma \epsilon_\lambda  M$. 
  Recovery follows the green curve to the replete adult mass, where this trajectory differs from the original growth because only fat is being regrown which requires a longer time to reach $\epsilon_\lambda M$. %different energetics and 
  Alternatively, death from starvation follows the blue trajectory to $m=\epsilon_\mu \epsilon_\lambda  M$.}\label{fig:growth}}
  \end{figure}

  % \noindent {\bf Role of allometry} \\
  %[Link Allometry stuff to our model - what does it provide to the story]
  %[Introduce biological and linked constraints]
  %[Allometries capture vast amounts of diversity via a single parameter --- body size]
  %[Allometries have captured intterest etc across multiple scales and biological classes of organisms]
  \noindent {\bf Results}
  \noindent \paragraph*{{\bf The allometry of extinction risk.}} While there
  are no {\it a priori} constraints on the parameters in the NSM, we expect
  that each species should be restricted to a distinct portion of the parameter
  space.  We use allometric scaling relations to constrain the covariation of
  rates in a principled and biologically meaningful manner (see Methods).
  Allometric scaling relations highlight common constraints and average trends
  across large ranges in body size and species diversity. Many of these
  relations can be derived from a small set of assumptions.  In the Methods we
  describe our framework to determine the covariation of timescales and rates
  across a range of body sizes for each of the key parameters of our model
  (cf.\ Ref.~\citep{Yodzis:1992hg}).
  % We are thereby able to define the regime of dynamics occupied by the entire class of mammals, along with the key differences between the largest and smallest mammals.

  Nearly all of the rates described in the NSM are determined by consumer
  metabolism, which can be used to describe a variety of organismal features
  \citep{Brown:2004wq}.  We derive, from first principles, the relationships
  for the rates of reproduction, starvation, recovery, and mortality as a
  function of an organism's body size and metabolic rate (see Methods).
  Because we aim to explore the starvation-recovery dynamics as a function of
  an organism's body mass $M$, we parameterize these rates in terms of the
  \emph{percent} gain and loss of the asymptotic (maximum) body mass,
  $\epsilon M$, where different values of $\epsilon$ define different states of
  the consumer (Fig.~\ref{fig:growth}; see Methods for derivations of
  allometrically constrained rate equations).  Although the rate equations
  \eqref{eq:system} are general and can in principle be used to explore the
  starvation recovery dynamics for most organisms, here we focus on allometric
  relationships for terrestrial-bound lower-trophic level endotherms (see the
  SI for values), specifically herbivorous mammals, which range from a minimum
  of $M\approx1$g (the Etruscan shrew \emph{Suncus etruscus}) to a maximum of
  $M\approx10^7$g (the early Oligocene Indricotheriinae and the Miocene
  Deinotheriinae).  Investigating other classes of organisms would simply
  involve altering the metabolic exponents and scalings associated with
  $\epsilon$. Moreover, we emphasize that our allometric equations (see
  Methods) describe mean relationships, and do not account for the (sometimes
  considerable) variance associated with individual species.  We note that
  including additional allometrically-scaled mortality terms to both $F$ and
  $H$ does not change the form of our model nor impact our quantitative findings
  (see SI for the derivation).

  % \noindent {\bf  Stabilizing effects of allometric constraints} \\
  % As the allometric derivations of the NSM rate laws reveal, starvation and recovery rates are not independent parameters, and the biologically relevant portion of the phase space shown in Fig.~\ref{fig:fp} is constrained via covarying parameters.
  % Given the parameters of terrestrial endotherms, we find that the starvation rate $\sigma$ and the recovery rate $\rho$ are constrained to lie within a small region of potential values for the known range of body sizes $M$.
  % Moreover, the dynamics for all mammalian body sizes are confined to the steady-state regime of the NSM and that limit-cycle behavior is precluded.
  %Incorporating uncertainty in allometric parameters (20\% variation around the mean; Fig.~\ref{fig:hopf}), we find that, for larger $M$, the distance to the TC and Hopf bifurcation decreases.
  % These results suggest that small mammals are marginally less prone to population oscillations---both stable limit cycles and transient cycles---than mammals with larger body size.  However, starvation and recovery rates across all values of $M$ fall squarely in the steady state region at some distance from the Hopf bifurcation. 
  % This result suggests that cyclic population dynamics should be rare, particularly in environments where resources are limiting.\\


  % Previous studies used allometric constraints to explain the periodicity of cyclic populations \citep{CalderIII:1983jd,Peterson:1984hj,Krukonis:1991fk}, suggesting a period $\propto M^{0.25}$.
  % However this relation seems to hold only for some species \citep{Hendriks:2012fc}, and potential drivers of variation and systematically different behavior range from predator and/or prey lifespans to competitive dynamics~\citep{Kendall:1999iy,Hogstedt:2005cr}.
  % Statistically significant support for the existence of population cycles among mammals is relatively rare, though predominantly based on time series for small mammals~\citep{Kendall:1998hl}.
  % However the longer gestational times and increased difficulty in collecting adequate data precludes obtaining similar-quality information for larger organisms.\\ 



  \begin{figure}
  \centering
  \includegraphics[width=0.4\textwidth]{fig_ExtinctionAllometricComb4-eps-converted-to.pdf}
  \caption{\small{ Probability of extinction for a consumer with ({\bf a}) $M=10^2$g and ({\bf b}) $M=10^6$g as a function of the starvation rate $\sigma$ and recovery rate $\rho$, where the initial density is given as $(XF^*,XH^*,R^*)$, where $X$ is a random uniform variable in $[0,2]$. Note the change in scale in panel {\bf b}.  Extinction is defined as the population trajectory falling below $0.2\times$ the allometrically constrained steady state. The white points denote the allometrically constrained starvation and recovery rate.}\label{fig:ext}}
  \end{figure}

  %Extinction to the left and the right
  % \noindent \paragraph*{{\bf Extinction risk}} 
  As the allometric derivations of the NSM rate laws reveal (see Methods),
  starvation and recovery rates are not independent parameters, and the
  biologically relevant portion of the phase space shown in Fig.~\ref{fig:fp}
  is constrained via covarying parameters.  Given the parameters of terrestrial
  endotherms, we find that the starvation rate $\sigma$ and the recovery rate
  $\rho$ are constrained to lie within a small region of potential values for
  the known range of body sizes $M$.  Indeed, starvation and recovery rates
  across all values of $M$ fall squarely in the steady-state region at some
  distance from the Hopf bifurcation.  This suggests that cyclic population
  dynamics should be rare, particularly in resource-limited environments.
  % Moreover, the dynamics for all mammalian body sizes are confined to the steady-state regime of the NSM and that limit-cycle behavior is precluded.

  Higher rates of starvation result in a larger flux of the population to the hungry state.
  In this state, reproduction is absent, thus increasing the likelihood of extinction.  From the perspective of population survival, it is the rate of starvation relative to the rate of recovery that determines the long-term dynamics of the various species (Fig.~\ref{fig:fp}).
  We therefore examine the competing effects of cyclic dynamics vs.\ changes in steady-state density on extinction risk, both as functions of $\sigma$ and $\rho$.
  To this end, we computed the probability of extinction, where we define extinction as a population trajectory falling below one fifth of the allometrically constrained steady state at any time between $t=10^8$ and $t=10^{10}$.
  This procedure was repeated for 50 replicates of the continuous-time system shown in Eq.~\ref{eq:system} for organisms with mass ranging from $10^2$ to $10^6$ grams.
  %the initial condition isdistributed around the steady state (Eq.~\ref{eq:ss}).  Specifically
  In each replicate the initial densities were chosen to be $(XF^*,XH^*,R^*)$,
  with $X$ a random variable uniformly distributed in $[0,2]$.  By allowing the
  rate of starvation to vary, we assessed extinction risk across a range of
  values for $\sigma$ and $\rho$ between ca.\ $10^{-8}$ to
  $10^{-3}$. %, thus examining a horizontal cross-section of Fig.~\ref{fig:fp}.
  Higher rates of extinction correspond to both large $\sigma$ if $\rho$ is
  small, and large $\rho$ if $\sigma$ is small.  In the former case, increased
  extinction risk arises because of the decrease in the steady-state consumer
  population density (Figs. \ref{fig:fp}b, \ref{fig:ext}).  In the latter case,
  the increased extinction risk results from higher-amplitude transient cycles
  as the system nears the Hopf bifurcation (Fig.~\ref{fig:ext}).  This
  interplay creates an `extinction refuge', such that for a constrained range
  of $\sigma$ and $\rho$, extinction probabilities are minimized.

  % \noindent \paragraph*{{\bf Extinction refuge.}} 
  We find that the allometrically constrained values of $\sigma$ and $\rho$,
  each representing different trajectories along the ontogenetic curve
  (Fig. \ref{fig:growth}), fall squarely within the extinction refuge across a
  range of $M$ (Fig. \ref{fig:ext}a,b, white points). These values are close
  enough to the Hopf bifurcation to avoid low steady-state densities, yet
  distant enough to avoid large-amplitude transient cycles.  Allometric values
  of $\sigma$ and $\rho$ fall within this relatively small window, which
  supports the possibility that a selective mechanism has constrained the
  physiological conditions driving starvation and recovery rates within
  populations.  Such a mechanism would select for organism physiology that
  generates appropriate $\sigma$ and $\rho$ values that minimize extinction
  risk.  This selection could occur via the tuning of body fat percentages,
  metabolic rates, and/or biomass maintenance efficiencies.  We also find that
  as body size increases, the size of the low extinction-risk parameter space
  shrinks (Fig.~\ref{fig:ext}b), suggesting that the population dynamics for
  larger organisms are more sensitive to variability in physiological rates.
  This finding is in accordance with, and may serve as contributing support for, observations of increased extinction risk among larger mammals \cite{Liow:2008jx}.\\
  % Moreover, larger body sizes decrease the steady state resource density, such that fluctuations for larger organisms will be more likely to drive resources to extinction.
  % OLD WORDING Such a mechanism would involve a feedback between the dynamics of the population and the fitness of individuals within the population, though to what extent the dynamics of the population influence rates of starvation and recovery would also involve potential tradeoffs in reproduction and somatic maintenance.
  % To summarize, our finding that allometrically-determined parameters fall within this low extinction probability region suggests that the NSM dynamics may both drive---and constrain---natural animal populations.\\



  %Results are insensitive to alpha but possibly sensitive to beta (eff parameter)


  %NOTES 7/8/16
  %Calder has a hypothesis that period ~ Mass^(1/4).
  %Can we get "transient period" from Jacobian?
  %Lots of food web lit on stabilizing effects of body size! Brose, Petchey for instance. Good for discussion.


  % \noindent {\bf Energetic barriers to body size}
  \noindent \paragraph*{{\bf Damuth's Law and body size limits.}} The NSM
  correctly predicts that smaller species have larger steady-state population
  densities (Fig.~\ref{fig:mass}).  Similar predictions have been made for
  carnivore populations using alternative consumer-resource models
  \citep{DeLong:2012kw}.  Moreover, we show that the NSM provides independent
  theoretical support for Damuth's Law
  \citep{Damuth:1987kr,allen2002,enquist1998,Pedersen:2017he}.  Damuth's
  law shows that species abundances, $N^{*}$, follow $N^*=0.01
  M^{-0.78}$ (g m$^{-2}$). Figure \ref{fig:mass} shows that both $F^{*}$ and $H^{*}$ scale
  as $M^{-\eta}$, with $\eta\approx 3/4$,  over a wide range of organismal sizes and that $F^{*}+H^{*}$
  closely matches the best fit to Damuth's data.  Remarkably, this result
  illustrates that the steady state values of the NSM combined with the derived
  timescales naturally give rise to Damuth's law. While the previous metabolic
  studies supporting Damuth's law provided arguments for the value of the
  exponent~\citep{allen2002}, these studies are only able to infer the
  normalization constant ($0.01$ g$^{1.78}$ m$^{-2}$ in the above equation) from the data (see SI for a discussion of the energy
  equivalence hypothesis related to these metabolic arguments). Our model
  predicts not only the exponent but also the normalization constant by
  explicitly including the resource dynamics and the parameters that determine
  growth and consumption. It should be noted that density relationships of
  individual clades follow a more shallow scaling relationship than predicted
  by Damuth's law~\cite{Pedersen:2017he}.  In the context of our model,
  this finding suggests that future work may be able to anticipate these shifts
  by accounting for differences in the physiological parameters associated with
  each clade.


  \begin{figure}
  \centering
  \includegraphics[width=0.4\textwidth]{fig_FPAllometric2-eps-converted-to.pdf}
  \caption{\small{Consumer steady states $F^*$ (green) and $H^*$ (orange) as a function of
    body mass along with the data from Damuth \citep{Damuth:1987kr}. Inset: Resource steady state $R^*$ as a function of consumer body mass.}\label{fig:mass}}
  \end{figure}

  % \paragraph*{{\bf Maximum Body Size}}
  %JUSTIN I REMOVED: Our model shows that energetic equivalence breaks down at large $M$ suggesting that this maximum is a hard limit where deviations outside of this range are energetically suboptimal.  
  %, violating the energy equivalence. % set by deviations from constant efficiency obeyed by other populations.
  With respect to predicted steady state densities, the total metabolic rate of
  $F$ and $H$ becomes infinite at a finite mass, and occurs at the same scale
  where the steady state resources vanish (Fig.~\ref{fig:mass}). This
  asymptotic behavior is governed by body sizes at which $\epsilon_{\mu}$ and
  $\epsilon_{\lambda}$ (see Fig.~\ref{fig:growth}) equal zero, causing the
  timescales (Eqn. \ref{t1}) to become infinite and the rates $\mu$ and $\lambda$ to equal
  zero.
  % A theoretical upper bound on mammalian body size is given by
  % $\epsilon_\sigma=0$, where mammals are entirely composed of metabolic
  % reserves, and this occurs at $M=8.3\times 10^8$ (g), or $120$ times the mass
  % of a male African elephant.
  The $\mu=0$ asymptote occurs first when
  $f_{0}M^{\gamma-1}+u_{0}M^{\zeta-1}=1$, and corresponds to
  $(F^*,H^*,R^*)=(0,0,0)$.  This point predicts an upper bound on mammalian
  body size at $M_{\rm max}=6.54\times10^7$ (g).  Moreover, $M_{\rm max}$,
  which is entirely determined by the population-level consequences of
  energetic constraints, is within an order of magnitude of the maximum body
  size observed in the North American mammalian fossil
  record~\citep{Alroy:1998p1594}, as well as the mass predicted from an
  evolutionary model of body size evolution~\citep{Clauset:2009fh}.  We
  emphasize that the asymptotic behavior and predicted upper bound depend only
  on the scaling of body composition and are independent of the resource
  parameters.  The prediction of an asymptotic limit on mammalian size
  parallels work on microbial life where an upper and lower bound on bacterial
  size, and an upper bound on single cell eukaryotic size, is predicted from
  similar growth and energetic scaling
  relationships~\citep{Kempes:2012hy,Kempes:2016}.  It has also been shown that
  models that incorporate the allometry of hunting and resting combined with
  foraging time predicts a maximum carnivore size between $7\times10^{5}$ and
  $1.1\times10^{6}$ (g) \cite{Carbone:1999ju,Carbone:2007dz}.  Similarly, the
  maximum body size within a particular lineage has been shown to scale with
  the metabolic normalization constant
  \cite{Okie:2013ju}. %and depend on a critical death found to be constant from data
  This complementary approach is based on the balance between growth and
  mortality, and suggests that future connections between the scaling of fat
  and muscle mass should systematically be connected with $B_{0}$ when
  comparing
  lineages. % reasons or from dissimilar scaling of cellular composition which is analogous to the body fat composition employed here

  % {\color{red}
  % 
  % It should be noted that in our model all death is related to nutritional state, however the addition of a constant death term \cite{} would only adjust the values of $\lambda$ and $\mu$ without changing the form or or qualitative results of our model. We explored the sensitivity of our results to shifts in a constant death term (see SI) and find that additional sources of death do not shift our quantitative prediction of Damuth's law except for large sizes (Figure S\ref{}). We find that additional sources of death shifts the asymptotic behavior of our model and decreases the maximum possible mammal size. Within the range of reasonable death rates $M_{max}$ varies by $VALUE\%$. These results suggest that the largest mammals are most sensitive to additional sources of death perhaps putting them at greater extinction risk.
  % 
  % }


  %Significant deviations from constant energy use occur at $M \lesssim 1$ at the small end of the mammalian range and $M\approx 6.5*10^7$ at the large end.
  %Compellingly, this dynamic bound, which is determined by the rate of energetic recovery, is close to the minimum observed mammalian body size of ca.\ 1.3-2.5g, a range that occurs as the recovery rate begins its decline.
  %In addition to known transport limitations~\citep{West:2002ud}, we suggest that an additional constraint of lower body size stems from the dynamics of starvation.
  %This result mirrors other efforts \citep{Kempes:2012hy,Kempes:2016} where at a given scale multiple limitations constrain the smallest possibilities for life within a class of organisms.

  %The combined steady-state dynamics and allometric timescales of the NSM predict that larger mammals should outcompete smaller ones, and this suggests that the model may provide a framework with which to explore the energetic mechanisms behind Cope's rule. %, which we explore further below.


  \noindent \paragraph*{{\bf A mechanism for Cope's rule}} Metabolite transport
  constraints are widely thought to place strict boundaries on biological
  scaling~\citep{Brown:1993p708,West:1997cg,Brown:2004wq} and thereby lead to
  specific predictions on the minimum possible body size for
  organisms~\citep{West:2002ud}.  Above this bound, a number of energetic and
  evolutionary mechanisms have been explored to assess the costs and benefits
  associated with larger body masses, particularly for mammals.  One important
  such example is the \emph{fasting endurance hypothesis}, which contends that
  larger body size, with consequent lower metabolic rates and increased ability
  to maintain more endogenous energetic reserves, may buffer organisms against
  environmental fluctuations in resource availability~\citep{Millar:1990p923}.
  Over evolutionary time, terrestrial mammalian lineages show a significant
  trend towards larger body size---Cope's
  rule~\citep{Alroy:1998p1594,Clauset:2009fh,Smith:2010p3442,Saarinen:2014br}.
  It is thought that within-lineage drivers generate selection towards an
  optimal upper bound of roughly $10^7$ (g)~\citep{Alroy:1998p1594}, a value
  that is likely limited by higher extinction risk for large taxa over longer
  timescales~\citep{Clauset:2009fh}.  These trends are thought to be driven by
  a combination of climate change and niche
  availability~\citep{Saarinen:2014br}; however the underpinning energetic
  costs and benefits of larger body sizes, and how they influence dynamics over
  ecological timescales, have not been explored.
  % We argue that the NSM provides a suitable framework to explore these issues.



  % \noindent{\bf A within-lineage driver of Cope's Rule}
  % \paragraph*{{\bf Body Size Competition}} 
  The NSM predicts that the steady state resource density $R^{*}$ decreases
  with increasing body size of the consumer population (Fig.~\ref{fig:mass},
  inset), and classic resource competition theory predicts that the species
  surviving on the lowest resource abundance will outcompete others
  \citep{tilman1981,dutkiewicz2009,barton2010}. Thus, the combined NSM
  steady-state dynamics and allometric timescales (see Eq.~\eqref{t1}) predict
  that larger mammals have an intrinsic competitive advantage given a common
  resource.  
  % We will now show that the NSM indeed provides a mechanistic understanding of the energetic dynamics that give rise to both observed limitations on mammalian body size, as well as the observed trend towards larger body size over evolutionary time known as Cope's Rule.

  \begin{figure}
  \centering
  \includegraphics[width=0.4\textwidth]{fig_Invasion-eps-converted-to.pdf}
  \caption{\small{Competitive outcomes for a resident species with body mass
      $M$ vs. a closely related competing species with modified body mass
      $M^\prime=M(1+\chi)$.  The blue region denotes proportions of modified
      mass $\chi$ resulting in exclusion of the resident species.  The red
      region denotes values of $\chi$ that result in a mass that is below the
      starvation threshold and are thus infeasible.  Arrows point to the
      predicted optimal mass from our model $M_{\rm opt}=1.748\times 10^7$,
      which may serve as an evolutionary attractor for body mass.  The black
      wedge points to the largest body mass known for terrestrial mammals
      (\emph{Deinotherium} spp.) at $1.74\times10^7$
      (g)~\citep{Smith:2010p3442}.}\label{fig:invasion}}
  \end{figure}

   

  % Next we examine to what extent a more realistic upper bound to body mass may serve as an evolutionary attractor, thus providing a suitable within-lineage mechanism for Cope's rule.

  However, the above resource relationships do not offer a mechanism for how
  body size is selected.  We directly assess competitive outcome between two
  closely related species: a resident species of mass $M$, and a competing
  species (denoted by $^\prime$) where individuals have a different proportion
  of body fat such that $M^\prime=M(1+\chi)$.  For $\chi < 0$, the competing
  individuals have fewer metabolic reserves than the resident species and vice
  versa for $\chi>0$.  For the allowable values of $\chi$ (see SI), the mass of
  the competitor $M'$ should exceed the minimal amount of body fat,
  $1+\chi>\epsilon_{\sigma}$, and the adjusted time to reproduce must be
  positive, which, given Eq.~\ref{t1}, implies that
  $1-\epsilon_{\lambda}^{1-\eta}\left(1+\chi\right)^{1-\eta}>0$.  These
  conditions imply that $\chi\in(-f_0M^{\gamma-1},1/\epsilon_{\lambda}-1)$
  where the upper bound approximately equals $0.05$ and the lower bound is
  mass-dependent.  The modified mass of the competitor leads to altered rates
  of starvation $\sigma(M^\prime)$, recovery $\rho(M^\prime)$, and the
  maintenance of both starving $\delta(M^\prime)$ and full consumers
  $\beta(M^\prime)$ (see the SI for derivations of competitor rates).
  Importantly, $\epsilon_\sigma$, which determines the point along the growth
  curve that defines the body composition of starved foragers, is assumed to
  remain unchanged for the competing population (see SI).
  %There is no internal fixed point corresponding to a state where both original residents and invaders coexist (except for the trivial state $\chi=0$).

  To assess the susceptibility of the resident species to competitive
  exclusion, we determine which consumer pushes the steady-state resource
  density $R^*$ to lower values for a given value of $\chi$, with the
  expectation that a population capable of surviving on lower resource
  densities has a competitive advantage \citep{tilman1981}.  We find that for
  $M\leq 1.748\times10^7$ (g), having additional body fat ($\chi > 0$) results
  in a lower steady state resource density ($R^{\prime *}<R^*$), such that the
  competitor has an intrinsic advantage over the resident species
  (Fig. \ref{fig:invasion}).  However, for $M> 1.748\times10^7$ (g), leaner
  individuals ($\chi < 0$) have lower resource steady state densities.

  %, and this is due to the changing
  %covariance between energetic rates as a function of modified energetic
  %reserves \sid{I don't understand the phrase after the comma AND STILL DON'T}.

  The observed switch in susceptibility as a function of $\chi$ at
  $M_{\rm opt}= 1.748\times10^7$ (g) thus serves as an attractor, such that the
  NSM predicts organismal mass to increase if $M<M_{\rm opt}$ and decrease if
  $M>M_{\rm opt}$.  This value is close to but smaller than the asymptotic
  upper bound for terrestrial mammal body size predicted by the NSM, and is remarkably close to independent estimates of the largest land
  mammals, the early Oligocene \emph{Indricotherium} at $\approx$
  $1.5\times10^7$ (g) and the late Miocene \emph{Deinotherium} at $\approx$
  $1.74\times10^7$ (g) ~\citep{Smith:2010p3442}.  Additionally, our calculation
  of $M_{\rm opt}$ as a function of mass-dependent physiological rates is
  similar to theoretical estimates of maximum body size \citep{Clauset:2009fh},
  and provides independent theoretical support for the observation of a
  `maximum body size attractor' explored by Alroy~\citep{Alroy:1998p1594}.
  % {\color{red} We note that the model of \cite{Brown:1993p708} predicts 

  An optimal size for mammals at intermediate body mass was predicted by Brown et al.\ based on reproductive maximization and the transition between hungry and full individuals \cite{Brown:1993p708}. 
  By coupling the NSM to resource dynamics as well as introducing an explicit treatment of storage, we show that species with larger body masses have an inherent competitive advantage for size classes up to $M_{\rm opt}= 1.748\times10^7$ based on resource competition. Moreover, the mass distributions in Ref. \cite{Brown:1993p708} show that intermediate mammal sizes have the greatest species diversity, in contrast to our efforts, which consider total biomass and predict a much larger $M_{\rm opt}$.
  Compellingly, recent work shows that many communities can be dominated by the biomass of the large \cite{Hempson:2015hka}.
  % shows that the largest mammals have an advantage in terms resource competition. 
  %as well as  \citep{Alroy:1998p1594,Clauset:2009fh}.
  %This value is only XX\% the estimated size of Indricotheriinae, a group that includes the largest known terrestrial mammal.
  While the state of the environment as well as the competitive landscape will determine whether specific body sizes are selected for or against~\citep{Saarinen:2014br}, we propose that the dynamics of starvation and recovery described in the NSM provide a general selective mechanism for the evolution of larger body size among terrestrial mammals.\\

  \noindent {\bf Discussion} \\

  %and we suggest that the starvation dynamics described here act in concert with these other factors is the only driver of body size evolution.


  %One might be concerned a greater number of large mammals are currently not observed in the modern world given that larger mammals are less susceptible to extinction.
  %However, recent research suggests that the pleistocene may have been much more populated with a significant diversity of very large mammals \citep{Doughty:2013kd,Doughty:2015hy,Doughty:2015je} which were also much more geographically widespread than today.
  %These results, combined with our findings, suggest that the modern diversity of mammals may not represent a true steady state the current distribution of nutrients and large seeds may be very different from the past \citep{Doughty:2013kd,Doughty:2015hy,Doughty:2015je}.

  %Closing
  The energetics associated with somatic maintenance, growth, and reproduction
  are important elements that influence the dynamics of all
  populations~\citep{Stearns:1989ip}.  The NSM incorporates the dynamics of
  starvation and recovery that are expected to occur in resource-limited
  environments.  We found that incorporating allometrically-determined rates
  into the NSM predicts that: (i) extinction risk is minimized, (ii) the
  derived steady-states quantitatively reproduce Damuth's law, and (iii) the
  selective mechanism for the evolution of larger body sizes agrees with Cope's
  rule.  The NSM offers a means by which the dynamic consequences of energetic
  constraints can be assessed using macroscale interactions between and among
  species.

  %Future efforts will involve exploring the consequences of these dynamics in a spatially explicit framework, thus incorporating elements such as movement costs and spatial heterogeneity, which may elucidate additional tradeoffs associated with the dynamics of starvation and recovery.



  \section*{Methods}
  \small{
  {\bf Analytical solution to the NSM}
  Equation~\eqref{eq:system} has three fixed points: two trivial fixed points at $(F^*,H^*,R^*)=(0,0,0)$ and $(0,0,1)$, and one non-trivial, internal fixed point at
  \begin{align}
  \label{eq:ss}
  \begin{split}
  F^* &= (\sigma-\lambda)\frac{ \alpha  \lambda  \mu ^2  (\mu +\xi  \rho )}{A (\lambda  \rho  B+\mu  \sigma  (\beta  \mu +\lambda  (\delta +\rho )))}, \\
  H^* &= (\sigma-\lambda)\frac{ \alpha  \lambda ^2 \mu  (\mu +\xi  \rho )}{A (\lambda  \rho  B+\mu  \sigma  (\beta  \mu +\lambda  (\delta +\rho )))}, \\
  R^* &= (\sigma - \lambda)\frac{\mu  }{A}.
  \end{split}
  \end{align}
  where $A=(\lambda \xi \rho +\mu \sigma )$ and
  $B=(\beta \mu \xi +\delta \lambda \xi -\lambda \mu )$. The stability of this
  fixed point is determined by the Jacobian matrix $\bf J$, with
  $J_{ij}=\partial{\dot X_i}/\partial{X_j}$, when evaluated at the internal
  fixed point, and $\mathbf{X}$ is the vector $(F,H,R)$.  The parameters in
  Eq.~\eqref{eq:system} are such that the real part of the largest eigenvalue
  of $\mathbf{J}$ is negative, so that the system is stable with respect to
  small perturbations from the fixed point.  Because this fixed point is
  unique, it is the global attractor for all population trajectories for any
  initial condition where the resource and consumer densities are both nonzero.

  {\bf Metabolic scaling relationships} The scaling relation between an
  organism's metabolic rate $B$ and its body mass $M$ at reproductive maturity
  is known to scale as $B = B_0 M^\eta$, where the scaling exponent $\eta$ is
  typically close to $2/3$ or $3/4$ for metazoans (e.g.,
  Ref.~\citep{West:2002it,Brown:2004wq}), and has taxonomic shifts for
  unicellular species between $\eta\approx 1$ in eukaryotes and
  $\eta\approx 1.76$ in bacteria \citep{DeLong:2010dy,Kempes:2012hy}.
  %Justin adding an explanation for beta

  Several efforts have shown how a partitioning of $B$ between growth and
  maintenance purposes can be used to derive a general equation for both the
  growth trajectories and growth rates of organisms ranging from bacteria to
  metazoans
  \citep{West:2001bv,moses2008rmo,gillooly2002esa,hou,Savage:2004ed,Kempes:2012hy}. This relation is derived from the simple balance condition 
  $B_{0}m^{\eta}=E_{m}\dot{m}+B_{m}m\,,$
  % \begin{eqnarray}
  % \label{balance}
  % B_{0}m^{\eta}=E_{m}\frac{dm}{dt}+B_{m}m\,,
  % \end{eqnarray}
  \citep{West:2001bv,moses2008rmo,gillooly2002esa,hou,Savage:2004ed,Kempes:2012hy} where $E_{m}$ is the energy needed to synthesize a unit of mass, $B_{m}$ is
  the metabolic rate to support an existing unit of mass, and $m$ is the mass
  of the organism at any point in its development.  This balance has the
  general solution \citep{bettencourt,Kempes:2012hy}
  \begin{eqnarray}
  \label{m1}
  \left(\frac{m\left(t\right)}{M}\right)^{1-\eta}\!=1\!-\!\left[1\!-\!\left(\frac{m_{0}}{M}\right)^{1\!-\!\eta}\right]e^{-a\left(1\!-\!\eta\right)t/M^{1-\eta}},
  \end{eqnarray}
  where, for $\eta<1$, $M=(B_{0}/B_{m})^{1/(1-\eta)}$ is the asymptotic mass,
  $a=B_{0}/E_{m}$, and $m_0$ is mass at birth, itself varying allometrically
  (see the SI).  We now use this solution to define the timescale for
  reproduction and recovery from starvation (Fig.~\ref{fig:growth}; see
  \citep{moses2008rmo} for a detailed presentation of these timescales). The
  time that an organism takes to reach a particular mass $\epsilon M$ is given
  by the timescale
  \begin{equation}
  \label{t1}
  \tau\left(\epsilon\right) = \ln\left[\frac{1-\left(m_{0}/M\right)^{1-\eta}}{1-\epsilon^{1-\eta}}\right]\frac{M^{1-\eta}}{a\left(1-\eta\right)},
  \end{equation}
  where we define values of $\epsilon$ below to describe a variety of
  timescales, along with the rates related to $\tau$.  For example, the rate of
  reproduction is given by the timescale to go from the birth mass to the adult
  mass. The time to reproduce is given by Equation \ref{t1} as
  $t_{\lambda}=\tau\left(\epsilon_{\lambda}\right)$, where $\epsilon_{\lambda}$
  is the fraction of the asymptotic mass where an organism is reproductively
  mature and should be close to one (typically
  $\epsilon_{\lambda}\approx0.95\;$ \citep{West:2001bv}). Our reproductive
  rate, $\lambda$, is a specific rate, or the number of offspring produced per
  time per individual, defined as $\dot{F} = \lambda F$. In isolation this
  functional form gives the population growth
  $F\left(t\right) = F_{0}e^{\lambda t}$ which can be related to the
  reproductive timescale by assuming that when $t=t_{\lambda}$ it is also the
  case that $F=\nu F_{0}$, where $\nu-1$ is the number of offspring produced
  per reproductive cycle. Following this relationship the growth rate is given
  by $\lambda=\ln\left(\nu\right)/t_{\lambda}$, which is the standard
  relationship (e.g.,~\cite{Savage:2004ed}) and will scales as
  $\lambda\propto M^{\eta-1}$ for $M\gg m_{0}$ for any constant value of
  $\epsilon_{\lambda}$
  \citep{West:2001bv,moses2008rmo,gillooly2002esa,hou,Kempes:2012hy}.


  The rate of recovery $\rho = 1/t_\rho$ requires that an organism accrues
  sufficient tissue to transition from the hungry to the full state.  Since
  only certain tissues can be digested for energy (for example the brain cannot
  be degraded to fuel metabolism), we define the rates for starvation, death,
  and recovery by the timescales required to reach, or return from, specific
  fractions of the replete-state mass (see the SI, Table I, for
  parameterizations).  We define $m_{\sigma}=\epsilon_{\sigma} M$, where
  $\epsilon_{\sigma}<1$ is the fraction of replete-state mass where
  reproduction ceases. This fraction will deviate from a constant if tissue
  composition systematically scales with adult mass.  For example, making use
  of the observation that body fat in mammals scales with overall body size
  according to $M_{\rm fat}=f_{0}M^{\gamma}$ and assuming that once this mass
  is fully digested the organism starves, this would imply that
  $\epsilon_{\sigma}=1-f_{0}M^{\gamma}/M$. It follows that the recovery
  timescale, $t_{\rho}$, is the time to go from mass
  $m=\epsilon_{\sigma} \epsilon_{\lambda} M$ to $m=\epsilon_{\lambda}M$
  (Fig. \ref{fig:growth}). Using Eqs.~\eqref{m1} and \eqref{t1} this timescale
  is given by simply considering the growth curve starting from a mass of
  $m_{0}^{\prime}=\epsilon_{\sigma}\epsilon_{\lambda}M$, in which case
  \begin{eqnarray}
  \label{rhotimescale}
  t_{\rho}=\ln\left[\frac{1-\left(\epsilon_{\sigma}\epsilon_{\lambda}\right)^{1-\eta}}{1-\epsilon_\lambda^{1-\eta}}\right]\frac{M^{1-\eta}}{a^{\prime}\left(1-\eta\right)}
  \end{eqnarray}
  where $a^{\prime}=B_{0}/E_{m}^{\prime}$ accounts for possible deviations in
  the biosynthetic energetics during recovery (see the SI). It should be noted that more complicated ontogenetic models explicitly handle
  storage \citep{hou}, whereas this feature is implicitly covered by the body
  fat scaling in our framework.



  To determine the starvation rate, $\sigma$, we are interested in the time
  required for an organism to go from a mature adult that reproduces at rate
  $\lambda$, to a reduced-mass hungry state where reproduction is impossible.
  For starving individuals we assume that an organism must meet its maintenance
  requirements by using the digestion of existing mass as the sole energy
  source.  This assumption implies the metabolic balance
  $\dot{m}E_{m}^{\prime}=-B_{m}m$ or $\dot{m}=-a^{\prime}m/M^{1-\eta}$
  % \begin{eqnarray}
  % \frac{dm}{dt}E_{m}^{\prime}=-B_{m}m
  % \end{eqnarray}
  % or
  % \begin{eqnarray}
  % \frac{dm}{dt}=-\frac{a^{\prime}}{M^{1-\eta}}m
  % \end{eqnarray}
  where $E_{m}^{\prime}$ is the amount of energy stored in a unit of existing
  body mass, which differs from $E_{m}$, the energy required to
  synthesis a unit of biomass \citep{hou}. Given the replete mass, $M$, of an organism, the
  above energy balance prescribes the mass trajectory of a non-consuming
  organism: $m\left(t\right)=Me^{-a^{\prime}t/M^{1-\eta}}$.
  % \begin{eqnarray}
  % \label{mt}
  % m\left(t\right)=Me^{-a^{\prime}t/M^{1-\eta}}.
  % \end{eqnarray}
  The timescale for starvation is
  given by the time it takes $m(t)$ to reach $\epsilon_{\sigma} M$, which gives
  \begin{equation}
  \label{eq:sigma}
  t_{\sigma}=-\frac{M^{1-\eta}}{a^{\prime}}\ln\left(\epsilon_{\sigma}\right).
  \end{equation}
  The starvation rate is then $\sigma=1/t_{\sigma}$, which scales with
  replete-state mass as $1/M^{1-\eta}\ln\left(1-f_{0}M^{\gamma}/M\right)$.  An important
  feature is that $\sigma$ does not have a simple scaling dependence on
  $\lambda$, which is important for the dynamics that we
  later discuss.

  The time to death should follow a similar relation, but defined by a lower
  fraction of replete-state mass, $m_{\mu}=\epsilon_{\mu} M$ where $\epsilon_\mu < \epsilon_\sigma$.
  Suppose, for example, that an organism dies once it has digested all fat and
  muscle tissues, and that muscle tissue scales with body mass according to
  $M_{\rm musc}=u_{0}M^{\zeta}$.  This gives
  $\epsilon_{\mu}=1-\left(f_{0}M^{\gamma}+u_{0}M^{\zeta}\right)/M$. Muscle
  mass has been shown to be roughly proportional to body mass~\citep{Folland:2008ij} in
  mammals and thus $\epsilon_{\mu}$ is merely $\epsilon_{\sigma}$ minus a constant. The time to go from starvation to death is the total time to reach $\epsilon_{\mu}M$ minus the time to starve, or $t_{\mu}=-M^{1-\eta}\ln\left(\epsilon_{\mu}\right)/a^{\prime}-t_{\sigma}$,
  % \begin{eqnarray}
  % \label{mutimescale}
  % t_{\mu}=-\frac{M^{1-\eta}}{a^{\prime}}\ln\left(\epsilon_{\mu}\right)-t_{\sigma},
  % \end{eqnarray}
  and $\mu=1/t_{\mu}$.
  }

\def\bibfont{\footnotesize}
% \putbib[aa_starving3]

\begin{thebibliography}{10}
\expandafter\ifx\csname url\endcsname\relax
  \def\url#1{\texttt{#1}}\fi
\expandafter\ifx\csname urlprefix\endcsname\relax\def\urlprefix{URL }\fi
\providecommand{\bibinfo}[2]{#2}
\providecommand{\eprint}[2][]{\url{#2}}

\bibitem{Martin:1987dl}
\bibinfo{author}{Martin, T.~E.}
\newblock \bibinfo{title}{{Food as a limit on breeding birds: A life-history
  perspective}}.
\newblock \emph{\bibinfo{journal}{Annu. Rev. Ecol. Syst.}}
  \textbf{\bibinfo{volume}{18}}, \bibinfo{pages}{453--487}
  (\bibinfo{year}{1987}).

\bibitem{Kirk:1997cc}
\bibinfo{author}{Kirk, K.~L.}
\newblock \bibinfo{title}{{Life-history responses to variable environments:
  Starvation and reproduction in planktonic rotifers}}.
\newblock \emph{\bibinfo{journal}{Ecology}} \textbf{\bibinfo{volume}{78}},
  \bibinfo{pages}{434--441} (\bibinfo{year}{1997}).

\bibitem{Kempes:2012hy}
\bibinfo{author}{Kempes, C.~P.}, \bibinfo{author}{Dutkiewicz, S.} \&
  \bibinfo{author}{Follows, M.~J.}
\newblock \bibinfo{title}{{Growth, metabolic partitioning, and the size of
  microorganisms}}.
\newblock \emph{\bibinfo{journal}{Proc. Natl. Acad. Sci. USA}}
  \textbf{\bibinfo{volume}{109}}, \bibinfo{pages}{495--500}
  (\bibinfo{year}{2012}).

\bibitem{Mangel:1988uaa}
\bibinfo{author}{Mangel, M.} \& \bibinfo{author}{Clark, C.~W.}
\newblock \emph{\bibinfo{title}{{Dynamic Modeling in Behavioral Ecology}}}
  (\bibinfo{publisher}{Princeton University Press},
  \bibinfo{address}{Princeton}, \bibinfo{year}{1988}).

\bibitem{Mangel:2014kz}
\bibinfo{author}{Mangel, M.}
\newblock \bibinfo{title}{{Stochastic dynamic programming illuminates the link
  between environment, physiology, and evolution}}.
\newblock \emph{\bibinfo{journal}{B. Math. Biol.}}
  \textbf{\bibinfo{volume}{77}}, \bibinfo{pages}{857--877}
  (\bibinfo{year}{2014}).

\bibitem{Yeakel:2013hi}
\bibinfo{author}{Yeakel, J.~D.}, \bibinfo{author}{Dominy, N.~J.},
  \bibinfo{author}{Koch, P.~L.} \& \bibinfo{author}{Mangel, M.}
\newblock \bibinfo{title}{{Functional morphology, stable isotopes, and human
  evolution: a model of consilience}}.
\newblock \emph{\bibinfo{journal}{Evolution}} \textbf{\bibinfo{volume}{68}},
  \bibinfo{pages}{190--203} (\bibinfo{year}{2014}).

\bibitem{Morris:1987eo}
\bibinfo{author}{Morris, D.~W.}
\newblock \bibinfo{title}{{Optimal allocation of parental investment}}.
\newblock \emph{\bibinfo{journal}{Oikos}} \textbf{\bibinfo{volume}{49}},
  \bibinfo{pages}{332--339} (\bibinfo{year}{1987}).

\bibitem{Tveraa:2003fq}
\bibinfo{author}{Tveraa, T.}, \bibinfo{author}{Fauchald, P.},
  \bibinfo{author}{Henaug, C.} \& \bibinfo{author}{Yoccoz, N.~G.}
\newblock \bibinfo{title}{{An examination of a compensatory relationship
  between food limitation and predation in semi-domestic reindeer}}.
\newblock \emph{\bibinfo{journal}{Oecologia}} \textbf{\bibinfo{volume}{137}},
  \bibinfo{pages}{370--376} (\bibinfo{year}{2003}).

\bibitem{Daan:1988va}
\bibinfo{author}{Daan, S.}, \bibinfo{author}{Dijkstra, C.},
  \bibinfo{author}{Drent, R.} \& \bibinfo{author}{Meijer, T.}
\newblock \bibinfo{title}{{Food supply and the annual timing of avian
  reproduction}}.
\newblock In \bibinfo{editor}{Ouellet, H.} (ed.) \emph{\bibinfo{booktitle}{Acta
  XIX Congressus Internationalis Ornithologici, Volume I: Proceedings XIX
  International Ornithological Congress, 1986, Ottawa}},
  \bibinfo{pages}{392--407} (\bibinfo{publisher}{Proceedings XIX International
  Ornithological Congress}, \bibinfo{address}{Ottawa}, \bibinfo{year}{1989}).

\bibitem{Jacot:2009dv}
\bibinfo{author}{Jacot, A.}, \bibinfo{author}{Valcu, M.}, \bibinfo{author}{van
  Oers, K.} \& \bibinfo{author}{Kempenaers, B.}
\newblock \bibinfo{title}{{Experimental nest site limitation affects
  reproductive strategies and parental investment in a hole-nesting
  passerine}}.
\newblock \emph{\bibinfo{journal}{Animal Behaviour}}
  \textbf{\bibinfo{volume}{77}}, \bibinfo{pages}{1075--1083}
  (\bibinfo{year}{2009}).

\bibitem{Stearns:1989ip}
\bibinfo{author}{Stearns, S.~C.}
\newblock \bibinfo{title}{{Trade-offs in life-history evolution}}.
\newblock \emph{\bibinfo{journal}{Funct. Ecol.}} \textbf{\bibinfo{volume}{3}},
  \bibinfo{pages}{259} (\bibinfo{year}{1989}).

\bibitem{Barboza:2002in}
\bibinfo{author}{Barboza, P.} \& \bibinfo{author}{Jorde, D.}
\newblock \bibinfo{title}{{Intermittent fasting during winter and spring
  affects body composition and reproduction of a migratory duck}}.
\newblock \emph{\bibinfo{journal}{J Comp Physiol B}}
  \textbf{\bibinfo{volume}{172}}, \bibinfo{pages}{419--434}
  (\bibinfo{year}{2002}).

\bibitem{Threlkeld:1976ih}
\bibinfo{author}{Threlkeld, S.~T.}
\newblock \bibinfo{title}{{Starvation and the size structure of zooplankton
  communities}}.
\newblock \emph{\bibinfo{journal}{Freshwater Biol.}}
  \textbf{\bibinfo{volume}{6}}, \bibinfo{pages}{489--496}
  (\bibinfo{year}{1976}).

\bibitem{Weber:1998jg}
\bibinfo{author}{Weber, T.~P.}, \bibinfo{author}{Ens, B.~J.} \&
  \bibinfo{author}{Houston, A.~I.}
\newblock \bibinfo{title}{{Optimal avian migration: A dynamic model of fuel
  stores and site use}}.
\newblock \emph{\bibinfo{journal}{Evolutionary Ecology}}
  \textbf{\bibinfo{volume}{12}}, \bibinfo{pages}{377--401}
  (\bibinfo{year}{1998}).

\bibitem{Mduma:1999cp}
\bibinfo{author}{Mduma, S. A.~R.}, \bibinfo{author}{Sinclair, A. R.~E.} \&
  \bibinfo{author}{Hilborn, R.}
\newblock \bibinfo{title}{{Food regulates the Serengeti wildebeest: a 40-year
  record}}.
\newblock \emph{\bibinfo{journal}{J. Anim. Ecol.}}
  \textbf{\bibinfo{volume}{68}}, \bibinfo{pages}{1101--1122}
  (\bibinfo{year}{1999}).

\bibitem{Moore:2014hi}
\bibinfo{author}{Moore, J.~W.}, \bibinfo{author}{Yeakel, J.~D.},
  \bibinfo{author}{Peard, D.}, \bibinfo{author}{Lough, J.} \&
  \bibinfo{author}{Beere, M.}
\newblock \bibinfo{title}{{Life-history diversity and its importance to
  population stability and persistence of a migratory fish: steelhead in two
  large North American watersheds}}.
\newblock \emph{\bibinfo{journal}{J. Anim. Ecol.}}
  \textbf{\bibinfo{volume}{83}}, \bibinfo{pages}{1035--1046}
  (\bibinfo{year}{2014}).

\bibitem{Mead:1989dt}
\bibinfo{author}{Mead, R.~A.}
\newblock \bibinfo{title}{{The Physiology and Evolution of Delayed Implantation
  in Carnivores}}.
\newblock In \bibinfo{editor}{Gittleman, J.~L.} (ed.)
  \emph{\bibinfo{booktitle}{Carnivore Behavior, Ecology, and Evolution}},
  \bibinfo{pages}{437--464} (\bibinfo{publisher}{Springer US},
  \bibinfo{address}{Ithaca}, \bibinfo{year}{1989}).

\bibitem{Sandell:1990kw}
\bibinfo{author}{Sandell, M.}
\newblock \bibinfo{title}{{The evolution of seasonal delayed implantation}}.
\newblock \emph{\bibinfo{journal}{Q Rev Biol}} \textbf{\bibinfo{volume}{65}},
  \bibinfo{pages}{23--42} (\bibinfo{year}{1990}).

\bibitem{Bulik:1999eo}
\bibinfo{author}{Bulik, C.~M.} \emph{et~al.}
\newblock \bibinfo{title}{{Fertility and reproduction in women with anorexia
  nervosa}}.
\newblock \emph{\bibinfo{journal}{J. Clin. Psychiat.}}
  \textbf{\bibinfo{volume}{60}}, \bibinfo{pages}{130--135}
  (\bibinfo{year}{1999}).

\bibitem{Trites:2003cc}
\bibinfo{author}{Trites, A.~W.} \& \bibinfo{author}{Donnelly, C.~P.}
\newblock \bibinfo{title}{{The decline of Steller sea lions Eumetopias jubatus
  in Alaska: a review of the nutritional stress hypothesis}}.
\newblock \emph{\bibinfo{journal}{Mammal Rev.}} \textbf{\bibinfo{volume}{33}},
  \bibinfo{pages}{3--28} (\bibinfo{year}{2003}).

\bibitem{Glazier:2009hq}
\bibinfo{author}{Glazier, D.~S.}
\newblock \bibinfo{title}{{Metabolic level and size scaling of rates of
  respiration and growth in unicellular organisms}}.
\newblock \emph{\bibinfo{journal}{Funct. Ecol.}} \textbf{\bibinfo{volume}{23}},
  \bibinfo{pages}{963--968} (\bibinfo{year}{2009}).

\bibitem{Kooi2000}
\bibinfo{author}{Kooijman, S. A. L.~M.}
\newblock \emph{\bibinfo{title}{{Dynamic Energy and Mass Budgets in Biological
  Systems}}} (\bibinfo{address}{Cambridge}, \bibinfo{year}{2000}).

\bibitem{Sousa:2010ez}
\bibinfo{author}{Sousa, T.}, \bibinfo{author}{Domingos, T.},
  \bibinfo{author}{Poggiale, J.~C.} \& \bibinfo{author}{Kooijman, S. A. L.~M.}
\newblock \bibinfo{title}{{Dynamic energy budget theory restores coherence in
  biology}}.
\newblock \emph{\bibinfo{journal}{Philos. T. Roy. Soc. B}}
  \textbf{\bibinfo{volume}{365}}, \bibinfo{pages}{3413--3428}
  (\bibinfo{year}{2010}).

\bibitem{Diekmann:2010da}
\bibinfo{author}{Diekmann, O.} \& \bibinfo{author}{Metz, J. A.~J.}
\newblock \bibinfo{title}{{How to lift a model for individual behaviour to the
  population level?}}
\newblock \emph{\bibinfo{journal}{Philos. T. Roy. Soc. B}}
  \textbf{\bibinfo{volume}{365}}, \bibinfo{pages}{3523--3530}
  (\bibinfo{year}{2010}).

\bibitem{murdoch:2003}
\bibinfo{author}{Murdoch, W.~W.}, \bibinfo{author}{Briggs, C.~J.} \&
  \bibinfo{author}{Nisbet, R.~M.}
\newblock \emph{\bibinfo{title}{{Consumer-resource Dynamics}}},
  vol.~\bibinfo{volume}{36} of \emph{\bibinfo{series}{Monographs in population
  biology}} (\bibinfo{publisher}{Princeton University Press},
  \bibinfo{address}{Princeton}, \bibinfo{year}{2003}).

\bibitem{Damuth:1987kr}
\bibinfo{author}{Damuth, J.}
\newblock \bibinfo{title}{{Interspecific allometry of population density in
  mammals and other animals: the independence of body mass and population
  energy-use}}.
\newblock \emph{\bibinfo{journal}{Biol. J. Linn. Soc.}}
  \textbf{\bibinfo{volume}{31}}, \bibinfo{pages}{193--246}
  (\bibinfo{year}{1987}).

\bibitem{allen2002}
\bibinfo{author}{Allen, A.~P.}, \bibinfo{author}{Brown, J.~H.} \&
  \bibinfo{author}{Gillooly, J.~F.}
\newblock \bibinfo{title}{{Global biodiversity, biochemical kinetics, and the
  energetic-equivalence rule}}.
\newblock \emph{\bibinfo{journal}{Science}} \textbf{\bibinfo{volume}{297}},
  \bibinfo{pages}{1545--1548} (\bibinfo{year}{2002}).

\bibitem{enquist1998}
\bibinfo{author}{Enquist, B.~J.}, \bibinfo{author}{Brown, J.~H.} \&
  \bibinfo{author}{West, G.~B.}
\newblock \bibinfo{title}{{Allometric scaling of plant energetics and
  population density}}.
\newblock \emph{\bibinfo{journal}{Nature}} \textbf{\bibinfo{volume}{395}},
  \bibinfo{pages}{163--165} (\bibinfo{year}{1998}).

\bibitem{Pedersen:2017he}
\bibinfo{author}{Pedersen, R.~{\O}.}, \bibinfo{author}{Faurby, S.} \&
  \bibinfo{author}{Svenning, J.-C.}
\newblock \bibinfo{title}{{Shallow size{\textendash}density relations within
  mammal clades suggest greater intra-guild ecological impact of large-bodied
  species}}.
\newblock \emph{\bibinfo{journal}{J. Anim. Ecol.}}
  \textbf{\bibinfo{volume}{86}}, \bibinfo{pages}{1205--1213}
  (\bibinfo{year}{2017}).

\bibitem{Alroy:1998p1594}
\bibinfo{author}{Alroy, J.}
\newblock \bibinfo{title}{{Cope's rule and the dynamics of body mass evolution
  in North American fossil mammals}}.
\newblock \emph{\bibinfo{journal}{Science}} \textbf{\bibinfo{volume}{280}},
  \bibinfo{pages}{731--734} (\bibinfo{year}{1998}).

\bibitem{Clauset:2009fh}
\bibinfo{author}{Clauset, A.} \& \bibinfo{author}{Redner, S.}
\newblock \bibinfo{title}{{Evolutionary model of species body mass
  diversification}}.
\newblock \emph{\bibinfo{journal}{Phys. Rev. Lett.}}
  \textbf{\bibinfo{volume}{102}}, \bibinfo{pages}{038103}
  (\bibinfo{year}{2009}).

\bibitem{Smith:2010p3442}
\bibinfo{author}{Smith, F.~A.} \emph{et~al.}
\newblock \bibinfo{title}{{The evolution of maximum body size of terrestrial
  mammals}}.
\newblock \emph{\bibinfo{journal}{Science}} \textbf{\bibinfo{volume}{330}},
  \bibinfo{pages}{1216--1219} (\bibinfo{year}{2010}).

\bibitem{Saarinen:2014br}
\bibinfo{author}{Saarinen, J.~J.} \emph{et~al.}
\newblock \bibinfo{title}{{Patterns of maximum body size evolution in Cenozoic
  land mammals: Eco-evolutionary processes and abiotic forcing}}.
\newblock \emph{\bibinfo{journal}{Proc Biol Sci}}
  \textbf{\bibinfo{volume}{281}}, \bibinfo{pages}{20132049}
  (\bibinfo{year}{2014}).

\bibitem{Benichou:2014wu}
\bibinfo{author}{B{\'e}nichou, O.} \& \bibinfo{author}{Redner, S.}
\newblock \bibinfo{title}{{Depletion-controlled starvation of a diffusing
  forager.}}
\newblock \emph{\bibinfo{journal}{Phys. Rev. Lett.}}
  \textbf{\bibinfo{volume}{113}}, \bibinfo{pages}{238101}
  (\bibinfo{year}{2014}).

\bibitem{Benichou:2016wl}
\bibinfo{author}{B{\'e}nichou, O.}, \bibinfo{author}{Chupeau, M.} \&
  \bibinfo{author}{Redner, S.}
\newblock \bibinfo{title}{{Role of depletion on the dynamics of a diffusing
  forager}}.
\newblock \emph{\bibinfo{journal}{J Phys A-Math Theor}}
  \textbf{\bibinfo{volume}{49}}, \bibinfo{pages}{394003}
  (\bibinfo{year}{2016}).

\bibitem{Chupeau:2016jf}
\bibinfo{author}{Chupeau, M.}, \bibinfo{author}{B{\'e}nichou, O.} \&
  \bibinfo{author}{Redner, S.}
\newblock \bibinfo{title}{{Universality classes of foraging with resource
  renewal}}.
\newblock \emph{\bibinfo{journal}{Phys. Rev. E}} \textbf{\bibinfo{volume}{93}},
  \bibinfo{pages}{032403} (\bibinfo{year}{2016}).

\bibitem{Persson:1998hz}
\bibinfo{author}{Persson, L.}, \bibinfo{author}{Leonardsson, K.},
  \bibinfo{author}{De~Roos, A.~M.}, \bibinfo{author}{Gyllenberg, M.} \&
  \bibinfo{author}{Christensen, B.}
\newblock \bibinfo{title}{{Ontogenetic scaling of foraging rates and the
  dynamics of a size-structured consumer-resource model}}.
\newblock \emph{\bibinfo{journal}{Theor Popul Biol}}
  \textbf{\bibinfo{volume}{54}}, \bibinfo{pages}{270--293}
  (\bibinfo{year}{1998}).

\bibitem{murray2011mathematical}
\bibinfo{author}{Murray, J.~D.}
\newblock \emph{\bibinfo{title}{{Mathematical Biology: I. An Introduction}}},
  vol. \bibinfo{volume}{110} of \emph{\bibinfo{series}{Interdisciplinary
  Applied Mathematics}} (\bibinfo{publisher}{Springer New York},
  \bibinfo{year}{2011}).

\bibitem{Strogatz:2008wo}
\bibinfo{author}{Strogatz, S.~H.}
\newblock \emph{\bibinfo{title}{{Nonlinear Dynamics and Chaos: With
  Applications to Physics, Biology, Chemistry, and Engineering}}}.
\newblock Studies in nonlinearity (\bibinfo{publisher}{Westview Press},
  \bibinfo{address}{Boulder}, \bibinfo{year}{2008}).

\bibitem{GuckHolmes}
\bibinfo{author}{Guckenheimer, J.} \& \bibinfo{author}{Holmes, P.}
\newblock \emph{\bibinfo{title}{{Nonlinear Oscillations, Dynamical Systems, and
  Bifurcations of Vector Fields}}} (\bibinfo{publisher}{Springer},
  \bibinfo{address}{New York}, \bibinfo{year}{1983}).

\bibitem{Gross:2004p2428}
\bibinfo{author}{Gross, T.} \& \bibinfo{author}{Feudel, U.}
\newblock \bibinfo{title}{{Analytical search for bifurcation surfaces in
  parameter space}}.
\newblock \emph{\bibinfo{journal}{Physica D}} \textbf{\bibinfo{volume}{195}},
  \bibinfo{pages}{292--302} (\bibinfo{year}{2004}).

\bibitem{Hastings:2001jh}
\bibinfo{author}{Hastings, A.}
\newblock \bibinfo{title}{{Transient dynamics and persistence of ecological
  systems}}.
\newblock \emph{\bibinfo{journal}{Ecol. Lett.}} \textbf{\bibinfo{volume}{4}},
  \bibinfo{pages}{215--220} (\bibinfo{year}{2001}).

\bibitem{Neubert:1997wk}
\bibinfo{author}{Neubert, M.} \& \bibinfo{author}{Caswell, H.}
\newblock \bibinfo{title}{{Alternatives to resilience for measuring the
  responses of ecological systems to perturbations}}.
\newblock \emph{\bibinfo{journal}{Ecology}} \textbf{\bibinfo{volume}{78}},
  \bibinfo{pages}{653--665} (\bibinfo{year}{1997}).

\bibitem{Caswell:2005eo}
\bibinfo{author}{Caswell, H.} \& \bibinfo{author}{Neubert, M.~G.}
\newblock \bibinfo{title}{{Reactivity and transient dynamics of discrete-time
  ecological systems}}.
\newblock \emph{\bibinfo{journal}{J Differ Equ Appl}}
  \textbf{\bibinfo{volume}{11}}, \bibinfo{pages}{295--310}
  (\bibinfo{year}{2005}).

\bibitem{Neubert:2009td}
\bibinfo{author}{Neubert, M.} \& \bibinfo{author}{Caswell, H.}
\newblock \bibinfo{title}{{Detecting reactivity}}.
\newblock \emph{\bibinfo{journal}{Ecology}} \textbf{\bibinfo{volume}{90}},
  \bibinfo{pages}{2683--2688} (\bibinfo{year}{2009}).

\bibitem{Yodzis:1992hg}
\bibinfo{author}{Yodzis, P.} \& \bibinfo{author}{Innes, S.}
\newblock \bibinfo{title}{{Body size and consumer-resource dynamics}}.
\newblock \emph{\bibinfo{journal}{Am. Nat.}} \textbf{\bibinfo{volume}{139}},
  \bibinfo{pages}{1151--1175} (\bibinfo{year}{1992}).

\bibitem{Brown:2004wq}
\bibinfo{author}{Brown, J.}, \bibinfo{author}{Gillooly, J.},
  \bibinfo{author}{Allen, A.}, \bibinfo{author}{Savage, V.} \&
  \bibinfo{author}{West, G.}
\newblock \bibinfo{title}{{Toward a metabolic theory of ecology}}.
\newblock \emph{\bibinfo{journal}{Ecology}} \textbf{\bibinfo{volume}{85}},
  \bibinfo{pages}{1771--1789} (\bibinfo{year}{2004}).

\bibitem{Liow:2008jx}
\bibinfo{author}{Liow, L.~H.} \emph{et~al.}
\newblock \bibinfo{title}{{Higher origination and extinction rates in larger
  mammals}}.
\newblock \emph{\bibinfo{journal}{Proc. Natl. Acad. Sci. USA}}
  \textbf{\bibinfo{volume}{105}}, \bibinfo{pages}{6097--6102}
  (\bibinfo{year}{2008}).

\bibitem{DeLong:2012kw}
\bibinfo{author}{DeLong, J.~P.} \& \bibinfo{author}{Vasseur, D.~A.}
\newblock \bibinfo{title}{{A dynamic explanation of size{\textendash}density
  scaling in carnivores}}.
\newblock \emph{\bibinfo{journal}{Ecology}} \textbf{\bibinfo{volume}{93}},
  \bibinfo{pages}{470--476} (\bibinfo{year}{2012}).

\bibitem{Kempes:2016}
\bibinfo{author}{Kempes, C.~P.}, \bibinfo{author}{Wang, L.},
  \bibinfo{author}{Amend, J.~P.}, \bibinfo{author}{Doyle, J.} \&
  \bibinfo{author}{Hoehler, T.}
\newblock \bibinfo{title}{{Evolutionary tradeoffs in cellular composition
  across diverse bacteria}}.
\newblock \emph{\bibinfo{journal}{ISME J}} \textbf{\bibinfo{volume}{10}},
  \bibinfo{pages}{2145--2157} (\bibinfo{year}{2016}).

\bibitem{Carbone:1999ju}
\bibinfo{author}{Carbone, C.}, \bibinfo{author}{Mace, G.~M.},
  \bibinfo{author}{Roberts, S.~C.} \& \bibinfo{author}{Macdonald, D.~W.}
\newblock \bibinfo{title}{{Energetic constraints on the diet of terrestrial
  carnivores}}.
\newblock \emph{\bibinfo{journal}{Nature}} \textbf{\bibinfo{volume}{402}},
  \bibinfo{pages}{286--288} (\bibinfo{year}{1999}).

\bibitem{Carbone:2007dz}
\bibinfo{author}{Carbone, C.}, \bibinfo{author}{Teacher, A.} \&
  \bibinfo{author}{Rowcliffe, J.~M.}
\newblock \bibinfo{title}{{The Costs of Carnivory}}.
\newblock \emph{\bibinfo{journal}{PLoS Biol}} \textbf{\bibinfo{volume}{5}},
  \bibinfo{pages}{e22} (\bibinfo{year}{2007}).

\bibitem{Okie:2013ju}
\bibinfo{author}{Okie, J.~G.} \emph{et~al.}
\newblock \bibinfo{title}{{Effects of allometry, productivity and lifestyle on
  rates and limits of body size evolution}}.
\newblock \emph{\bibinfo{journal}{Proc Biol Sci}}
  \textbf{\bibinfo{volume}{280}}, \bibinfo{pages}{20131007--20131007}
  (\bibinfo{year}{2013}).

\bibitem{Brown:1993p708}
\bibinfo{author}{Brown, J.}, \bibinfo{author}{Marquet, P.} \&
  \bibinfo{author}{Taper, M.}
\newblock \bibinfo{title}{{Evolution of body size: consequences of an energetic
  definition of fitness}}.
\newblock \emph{\bibinfo{journal}{Am. Nat.}} \textbf{\bibinfo{volume}{142}},
  \bibinfo{pages}{573--584} (\bibinfo{year}{1993}).

\bibitem{West:1997cg}
\bibinfo{author}{West, G.~B.}, \bibinfo{author}{Brown, J.~H.} \&
  \bibinfo{author}{Enquist, B.~J.}
\newblock \bibinfo{title}{{A general model for the origin of allometric scaling
  laws in biology}}.
\newblock \emph{\bibinfo{journal}{Science}} \textbf{\bibinfo{volume}{276}},
  \bibinfo{pages}{122--126} (\bibinfo{year}{1997}).

\bibitem{West:2002ud}
\bibinfo{author}{West, G.~B.}, \bibinfo{author}{Woodruff, W.~H.} \&
  \bibinfo{author}{Brown, J.~H.}
\newblock \bibinfo{title}{{Allometric scaling of metabolic rate from molecules
  and mitochondria to cells and mammals.}}
\newblock \emph{\bibinfo{journal}{Proc. Natl. Acad. Sci. USA}}
  \textbf{\bibinfo{volume}{99 Suppl 1}}, \bibinfo{pages}{2473--2478}
  (\bibinfo{year}{2002}).

\bibitem{Millar:1990p923}
\bibinfo{author}{Millar, J.} \& \bibinfo{author}{Hickling, G.}
\newblock \bibinfo{title}{{Fasting endurance and the evolution of mammalian
  body size}}.
\newblock \emph{\bibinfo{journal}{Funct. Ecol.}} \textbf{\bibinfo{volume}{4}},
  \bibinfo{pages}{5--12} (\bibinfo{year}{1990}).

\bibitem{tilman1981}
\bibinfo{author}{Tilman, D.}
\newblock \bibinfo{title}{{Tests of resource competition theory using four
  species of lake michigan algae}}.
\newblock \emph{\bibinfo{journal}{Ecology}} \textbf{\bibinfo{volume}{62}},
  \bibinfo{pages}{802--815} (\bibinfo{year}{1981}).

\bibitem{dutkiewicz2009}
\bibinfo{author}{Dutkiewicz, S.}, \bibinfo{author}{Follows, M.~J.} \&
  \bibinfo{author}{Bragg, J.~G.}
\newblock \bibinfo{title}{{Modeling the coupling of ocean ecology and
  biogeochemistry}}.
\newblock \emph{\bibinfo{journal}{Global Biogeochem. Cycles}}
  \textbf{\bibinfo{volume}{23}}, \bibinfo{pages}{1--15} (\bibinfo{year}{2009}).

\bibitem{barton2010}
\bibinfo{author}{Barton, A.~D.}, \bibinfo{author}{Dutkiewicz, S.},
  \bibinfo{author}{Flierl, G.}, \bibinfo{author}{Bragg, J.} \&
  \bibinfo{author}{Follows, M.~J.}
\newblock \bibinfo{title}{{Patterns of diversity in marine phytoplankton}}.
\newblock \emph{\bibinfo{journal}{Science}} \textbf{\bibinfo{volume}{327}},
  \bibinfo{pages}{1509--1511} (\bibinfo{year}{2010}).

\bibitem{Hempson:2015hka}
\bibinfo{author}{Hempson, G.~P.}, \bibinfo{author}{Archibald, S.} \&
  \bibinfo{author}{Bond, W.~J.}
\newblock \bibinfo{title}{{A continent-wide assessment of the form and
  intensity of large mammal herbivory in Africa}}.
\newblock \emph{\bibinfo{journal}{Science}} \textbf{\bibinfo{volume}{350}},
  \bibinfo{pages}{1056--1061} (\bibinfo{year}{2015}).

\bibitem{West:2002it}
\bibinfo{author}{West, G.~B.}, \bibinfo{author}{Woodruff, W.~H.} \&
  \bibinfo{author}{Brown, J.~H.}
\newblock \bibinfo{title}{{Allometric scaling of metabolic rate from molecules
  and mitochondria to cells and mammals.}}
\newblock \emph{\bibinfo{journal}{Proc. Natl. Acad. Sci. USA}}
  \textbf{\bibinfo{volume}{99 Suppl 1}}, \bibinfo{pages}{2473--2478}
  (\bibinfo{year}{2002}).

\bibitem{DeLong:2010dy}
\bibinfo{author}{DeLong, J.~P.}, \bibinfo{author}{Okie, J.~G.},
  \bibinfo{author}{Moses, M.~E.}, \bibinfo{author}{Sibly, R.~M.} \&
  \bibinfo{author}{Brown, J.~H.}
\newblock \bibinfo{title}{{Shifts in metabolic scaling, production, and
  efficiency across major evolutionary transitions of life}}.
\newblock \emph{\bibinfo{journal}{Proc. Natl. Acad. Sci. USA}}
  \textbf{\bibinfo{volume}{107}}, \bibinfo{pages}{12941--12945}
  (\bibinfo{year}{2010}).

\bibitem{West:2001bv}
\bibinfo{author}{West, G.~B.}, \bibinfo{author}{Brown, J.~H.} \&
  \bibinfo{author}{Enquist, B.~J.}
\newblock \bibinfo{title}{{A general model for ontogenetic growth}}.
\newblock \emph{\bibinfo{journal}{Nature}} \textbf{\bibinfo{volume}{413}},
  \bibinfo{pages}{628--631} (\bibinfo{year}{2001}).

\bibitem{moses2008rmo}
\bibinfo{author}{Moses, M.~E.} \emph{et~al.}
\newblock \bibinfo{title}{{Revisiting a Model of Ontogenetic Growth: Estimating
  Model Parameters from Theory and Data.}}
\newblock \emph{\bibinfo{journal}{Am. Nat.}} \textbf{\bibinfo{volume}{171}},
  \bibinfo{pages}{632--645} (\bibinfo{year}{2008}).

\bibitem{gillooly2002esa}
\bibinfo{author}{Gillooly, J.~F.}, \bibinfo{author}{Charnov, E.~L.},
  \bibinfo{author}{West, G.~B.}, \bibinfo{author}{Savage, V.~M.} \&
  \bibinfo{author}{Brown, J.~H.}
\newblock \bibinfo{title}{{Effects of size and temperature on developmental
  time}}.
\newblock \emph{\bibinfo{journal}{Nature}} \textbf{\bibinfo{volume}{417}},
  \bibinfo{pages}{70--73} (\bibinfo{year}{2002}).

\bibitem{hou}
\bibinfo{author}{Hou, C.} \emph{et~al.}
\newblock \bibinfo{title}{{Energy uptake and allocation during ontogeny}}.
\newblock \emph{\bibinfo{journal}{Science}} \textbf{\bibinfo{volume}{322}},
  \bibinfo{pages}{736--739} (\bibinfo{year}{2008}).

\bibitem{Savage:2004ed}
\bibinfo{author}{Savage, V.~M.}, \bibinfo{author}{Gillooly, J.~F.},
  \bibinfo{author}{Brown, J.~H.}, \bibinfo{author}{West, G.~B.} \&
  \bibinfo{author}{Charnov, E.~L.}
\newblock \bibinfo{title}{{Effects of Body Size and Temperature on Population
  Growth}}.
\newblock
  \emph{\bibinfo{journal}{http://dx.doi.org.proxy.lib.sfu.ca/10.1086/679735}}
  \textbf{\bibinfo{volume}{163}}, \bibinfo{pages}{429--441}
  (\bibinfo{year}{2004}).

\bibitem{bettencourt}
\bibinfo{author}{Bettencourt, L. M.~A.}, \bibinfo{author}{Lobo, J.},
  \bibinfo{author}{Helbing, D.}, \bibinfo{author}{Kuhnert, C.} \&
  \bibinfo{author}{West, G.~B.}
\newblock \bibinfo{title}{{Growth, innovation, scaling, and the pace of life in
  cities}}.
\newblock \emph{\bibinfo{journal}{Proc. Natl. Acad. Sci. USA}}
  \textbf{\bibinfo{volume}{104}}, \bibinfo{pages}{7301--7306}
  (\bibinfo{year}{2007}).

\bibitem{Folland:2008ij}
\bibinfo{author}{Folland, J.~P.}, \bibinfo{author}{Mc~Cauley, T.~M.} \&
  \bibinfo{author}{Williams, A.~G.}
\newblock \bibinfo{title}{{Allometric scaling of strength measurements to body
  size}}.
\newblock \emph{\bibinfo{journal}{Eur J Appl Physiol}}
  \textbf{\bibinfo{volume}{102}}, \bibinfo{pages}{739--745}
  (\bibinfo{year}{2008}).

\end{thebibliography}


\end{bibunit}

% \bibliography{aa_starving3}



%
% \begin{acknowledgments}
%   We thank Luis Bettencourt, Jean Philippe Gibert, Eric Libby, and Seth Newsome for helpful
%   discussions and comments on the manuscript.  J.D.Y. was supported by
%   startup funds at the University of California, Merced, and an Omidyar
%   Postdoctoral Fellowship at the Santa Fe Institute.  C.P.K. was supported by
%   an Omidyar Postdoctoral Fellowship at the Santa Fe Institute.  S.R. was
%   supported by grants DMR-1608211 and 1623243 from the National Science
%   Foundation, and by the John Templeton Foundation, all at the Santa Fe
%   Institute.
% \end{acknowledgments}
%
\clearpage

\begin{bibunit}[unsrt]


% \addto\captionsenglish{\renewcommand{\figurename}{Figure S\!}}
\setcounter{table}{0}
\renewcommand{\thetable}{S\arabic{table}}%
\setcounter{figure}{0}
\renewcommand{\thefigure}{S\arabic{figure}}%


\section*{Supporting Information for ``The dynamics of starvation and recovery''}

{\bf Mechanisms of Starvation and Recovery}
To understand the dynamics of starvation, recovery, reproduction, and resource competition, our framework partitions consumers into two classes: (a) a full class that is able to reproduce and, (b) a hungry class that experiences mortality at a given rate and is unable to reproduce. For the dynamics of growth, reproduction, and resource consumption, past efforts have combined the overall metabolic rate, as dictated by body size, with a growth rate that is dependent on resource abundance and, in turn, dictates resource consumption (see Refs. \citep{Kempes:2012hy,kempes2014morphological} for a brief review of this perspective). This approach has been used to understand a range of phenomena including a derivation of ontogenetic growth curves from a partitioning of metabolism into maintenance and biosynthesis (e.g. \citep{West:2001bv,moses2008rmo,hou,Kempes:2012hy}) and predictions for the steady-state resource abundance in communities of cells \citep{kempes2014morphological}. Here we leverage these mechanisms, combined with several additional concepts, to define our Nutritional State Model (NSM).

We consider the following generalized set of explicit dynamics for starvation, recovery, reproduction, and resource growth and consumption
\begin{align}
\begin{split}
\dot{F_{d}} &= \lambda_{\text{max}} F_{d} + \rho_{\text{max}}R_{d}H_{d}/k - \sigma \left(1-\frac{R_{d}}{C}\right)F_{d},  \\
\dot{H_{d}} &= \sigma \left(1-\frac{R_{d}}{C}\right)F_{d} - \rho_{\text{max}}R_{d} H_{d}/k - \mu H_{d},  \\
\dot{R_{d}} &= \alpha R_{d}\left(1-\frac{R_{d}}{C}\right) -\\
& \left[\left(\frac{\rho_{\text{max}}R_{d}}{Y_{H}k}+P_{H}\right)H_{d}+\left(\frac{\lambda_{\text{max}}}{Y_{F}}+P_{F}\right)F_{d}\right].
\label{bigdynamics}
\end{split}
\end{align}
where each term has a mechanistic meaning that we detail below (we will denote the dimensional equations with the subscript $_{d}$ before introducing the non-dimensional form that is presented in the main text). In the above equations $Y$ represents the yield coefficient (e.g., Refs. \citep{pirt,Heijnen}) which is the quantity of resources required to build a unit of organism (gram of mammal produced per gram of resource consumed) and $P$ is the specific maintenance rate of resource consumption (g resource $\cdot$ s$^{-1}$ $\cdot$ g organism$^{-1}$). If we pick $F_{d}$ and $H_{d}$ to have units of (g organisms $\cdot$ m$^{-2}$), then all of the terms of $\dot{R_{d}}$, such as $\frac{\rho\left(R_{d}\right)}{Y}H_{d}$, have units of (g resource $\cdot$ m$^{-2}$ $\cdot$ s$^{-1}$) which are the units of net primary productivity (NPP), a natural choice for $\dot{R_{d}}$. This choice also gives $R_{d}$ as (g $\cdot$ m$^{-2}$) which is also a natural unit and is simply the biomass density. In these units $\alpha$ (s$^{-1}$) is the specific growth rate of $R_{d}$, $C$ is the carrying capacity, or maximum density, of $R_{d}$ in a particular environment, and $k$ is the half-saturation constant (half the density of resources that would lead to maximum growth).
%(note this is not fully explicit because I don't know how to deal with the response of $\sigma$ to resources, although I have an idea for a derivation which may be necessary given the following approximations)




%\begin{align}
%\begin{split}
%\dot{F_{d}} &= \lambda\left(R_{d}\right) F_{d} + \rho\left(R_{d}\right)H_{d} - \sigma \left(1-\frac{R_{d}}{C}\right)F_{d},  \\
%\dot{H_{d}} &= \sigma \left(1-\frac{R_{d}}{C}\right)F_{d} - \rho\left(R_{d}\right)H_{d} - \mu H_{d},  \\
%\dot{R_{d}} &= \alpha R_{d}\left(1-\frac{R_{d}}{C}\right) - \\
%&\left[\left(\frac{\rho\left(R_{d}\right)}{Y}+P_{H}\right)H_{d}+\left(\frac{\lambda\left(R_{d}\right)}{Y}+P_{F}\right)F_{d}\right],
%\end{split}
%\end{align}
%
%In this set of equations $\lambda\left(R_{d}\right)$ and $\rho\left(R_{d}\right)$ are the growth and recovery rates as functions of the current resource availability. Typically these can be written as $\lambda\left(R_{d}\right)=\lambda_{max}S\left(R_{d}\right)$ or $\lambda\left(R_{d}\right)=\lambda_{max}S\left(R_{d}\right)$ where $\lambda_{max}$ and $\rho_{max}$ are the maximum growth and recovery rates respectively, which scale with body size as discussed later, and $S\left(R_{d}\right)$ is a saturating function of resources. The saturating function could, for example, be a Michaelis-Menten or Monod function of the form $\frac{R_{d}}{k+R_{d}}$, where $k$ is the half-saturation constant.
%A simplified version of the Michaelis-Menten or Monod functional form, which captures the essential features, is a linear function that saturates to a constant value above a certain abundance of $R_{d}$.
%
%Before describing the values of each of these constants, and a general non-dimensionalization of the system of equations, it is important to consider the resource regimes associated with the above equations which lead to a simplification. As discussed above, the resource saturation function should be defined by a linear regime proportional to $R_{d}$ when $R_{d}<<k$, and a constant value for $R_{d}>>k$. Thus for hungry individuals, $H_{d}$, where $R_{d}<<k$, we have that $\rho\left(R_{d}\right)\approx\rho_{max}R_{d}/k$, and for the full class, $F_{d}$, of organisms $\lambda\left(R_{d}\right)\approx\lambda_{max}$, such that the above relationships reduce to





%where $\beta=\frac{\lambda_{max}}{Y_{F}}+P$ which is just a constant that depends on the size of an organisms via the allometries for $\lambda_{max}$ and $P$ discussed later.

We can formally non-dimensionalize this system by the rescaling of $F=fF_{d}$, $H=fH_{d}$, $R=qR_{d}$, $t=st_{d}$, in which case our system of equations becomes
%(ignoring the $\sigma (1-R)F$ terms which I don't have a dimensional form for yet),:
\begin{align}
\begin{split}
&\dot{F} = \frac{1}{s}\left[\lambda_{\text{max}} F + \rho_{\text{max}}\frac{R}{qk}H - \sigma \left(1-\frac{R}{qC}\right)F\right],  \\
&\dot{H} = \frac{1}{s}\left[\sigma \left(1-\frac{R}{qC}\right)F - \rho_{\text{max}}\frac{R}{qk} H - \mu H\right],  \\
& \dot{R} = \\
&\frac{1}{s}\left[\alpha R\left(1-\frac{R}{qC}\right) -\frac{q}{f}\left[\left(\frac{\rho_{\text{max}}R}{Y_{H}kq}+P_{H}\right)H+\left(\frac{\lambda_{\text{max}}}{Y_{F}}+P_{F}\right)F\right]\right].
\end{split}
\end{align}
If we make the natural choice of $s=1$, $q=1/C$, and $f=1/Y_{H}k$, then we are left with
\begin{align}
\begin{split}
\dot{F} &= \lambda F + \xi \rho RH - \sigma \left(1-R\right)F,  \\
\dot{H} &= \sigma \left(1-R\right)F - \xi \rho RH - \mu H,  \\
\dot{R} &= \alpha R\left(1-R\right) -\left(\rho R+\delta\right)H-\beta F
\label{reduceddynamics}
\end{split}
\end{align}
where we have dropped the subscripts on $\lambda_{\text{max}}$ and $\rho_{\text{max}}$ for simplicity, and $\xi\equiv C/k$, $\delta\equiv Y_{H}kP_{H}/C$, and $\beta\equiv Y_{H}k\left(\frac{\lambda_{\text{max}}}{Y_{F}}+P_{F}\right)/C$. The above equations represent the system of equations presented in the main text.
\\

{\bf Parameter Values and Estimates}
All of the parameter values employed in our model have either been directly measured in previous studies or can be estimated from combining several previous studies. Below we outline previous measurements and simple estimates of the parameters.

Metabolic rate has been generally reported to follow an exponent close to $\eta=0.75$ (e.g., Refs. \citep{West:2001bv,moses2008rmo} and the supplement for Ref. \citep{hou}). We make this assumption in the current paper, although alternate exponents, which are known to vary between roughly $0.25$ and $1.5$ for single species \citep{moses2008rmo}, could be easily incorporated into our framework, and this variation is effectively handled by the $20\%$ variations that we consider around mean trends. The exponent not only defines several scalings in our framework, but also the value of the metabolic normalization constant, $B_{0}$, given a set of data.  For mammals the metabolic normalization constant has been reported to vary between $0.018$ (W g$^{-0.75}$) and $0.047$ (W g$^{-0.75}$; Refs. \citep{hou,West:2001bv}, where the former value represents basal metabolic rate and the latter represents the field metabolic rate. We employ the field metabolic rate for our NSM model which is appropriate for active mammals (Table 1).

An important feature of our framework is the starting size, $m_{0}$, of a mammal which adjusts the overall timescales for reproduction. This starting size is known to follow an allometric relationship with adult mass of the form $m_{0}=n_{0}M^{\upsilon}$ where estimates for the exponent range between $0.71$ and $0.94$ (see Ref. \citep{peters1986ecological} for a review). We use $m_{0}=0.097M^{0.92}$ \citep{blueweiss1978relationships} which encompasses the widest range of body sizes \citep{peters1986ecological}.

The energy to synthesize a unit of biomass, $E_{m}$, has been reported to vary between $1800$ to $9500$ (J g$^{-1}$) (e.g. Refs. \citep{West:2001bv,moses2008rmo,hou}) in mammals with a mean value across many taxonomic groups of $5,774$ (J g$^{-1}$) \citep{moses2008rmo}. The unit energy available during starvation, $E^{\prime}$, could range between $7000$ (J g$^{-1}$), the return of the total energy stored during ontogeny \citep{hou} to a biochemical upper bound of $E^{\prime}=36,000$ (J g$^{-1}$) for the energetics of palmitate \citep{stryer,hou}. For our calculations we use the measured value for bulk tissues of $7000$ which assumes that the energy stored during ontogeny is returned during starvation \citep{hou}.

For the scaling of body composition it has been shown that fat mass follows $M_{\rm fat}=f_{0}M^{\gamma}$, with measured  relationships following  $0.018M^{1.25}$ ~\citep{Dunbrack:1993ec}, $0.02M^{1.19}$ ~\citep{Lindstedt:1985hm}, and $0.026M^{1.14}$ ~\citep{Lindstedt:2002td}. We use the values from \citep{Lindstedt:1985hm} which falls in the middle of this range. Similarly, the muscle mass follows $M_{\rm musc}=u_{0}M^{\zeta}$ with $u_{0}=0.383$ and $\zeta=1.00$ ~\citep{Lindstedt:2002td}.

%We also connect the resource growth rate to the total metabolic rate of an organism. That is, we are interested in the relative rates of resource recovery and consumption by the total population. From \citep{allen2002global} the total resource use of a population with an individual body size of $M$ is given by $B_{pop}=0.00061x^{-0.03}$ (W m$^{-2}$). Considering an energy density of $18200$ (J g$^{-1}$) of grass \citep{estermann} and an NPP between and $1.59\times10^{-6}$ and $7.92\times10^{-5}$ (g s$^{-1}$ m$^{-2}$) would give a range of resource rates between  $0.029$ and $1.44$ (W m$^{-2}$). This gives a ratio of total resource consumption to supply rates between $0.00042$ and $0.021$, and we used a value of $0.002$ in our calculations and simulations.

Typically the value of $\xi=C/k$ should roughly be $2$. The value of $\rho$, $\lambda$, $\sigma$, and $\mu$ are all simple rates (note that we have not rescaled time in our non-dimensionalization) as defined in the maintext. Given that our model considers transitions over entire stages of ontogeny or nutritional states, the value of $Y$ must represent yields integrated over entire life stages. Given an energy density of $E_{d}=18200$ (J g$^{-1}$) for grass \citep{estermann} the maintenance value is given by $P_{F}=B_{0}M^{3/4}/ME_{d}$, and the yield for a full organism will be given by $Y_{F}=ME_{d}/B_{\lambda}$ (g individual $\cdot$ g grass $^{-1}$), where $B_{\lambda}$ is the lifetime energy use for reaching maturity given by
\begin{equation}
B_{\lambda}=\int_{0}^{t_{\lambda}}B_{0}m\left(t\right)^{\eta}dt.
\end{equation}
Similarly, the maintenance resource consumption rate for hungry individuals is $P_{H}=B_{0}(\epsilon_{\sigma}M)^{3/4}/(\epsilon_{\sigma}M)E_{d}$, and the yield for hungry individuals (representing the cost on resources to return to the full state) is given by $Y_{H}=ME_{d}/B_{\rho}$ where
\begin{equation}
B_{\rho}=\int_{\tau\left(\epsilon_{\sigma}\epsilon_{\lambda}\right)}^{t_{\lambda}}B_{0}m\left(t\right)^{\eta}dt.
\end{equation}
Taken together, these relationships allow us to calculate $\rho$, $\delta$, and $\beta$.

Finally, the value of $\alpha$ can be roughly estimated by the NPP divided by the corresponding biomass densities. From the data in Ref. \citep{michaletz2014convergence} we estimate the value of $\alpha$ to range between $2.81\times10^{-10}$ (s$^{-1}$) and $2.19\times10^{-8}$ (s$^{-1}$) globally. It should be noted that the value of $\alpha$ sets the overall scale of the $F^{*}$ and $H^{*}$ steady states along with $B_{tot}$ for each type. As such, we use $\alpha$ as our fit parameter to match these steady states with the data from Damuth \citep{damuth1987interspecific}. We find that the best fit is $\alpha=9.45\times10^{-9}$ (s$^{-1}$) which compares well with the calculated range above. However, two points are important to note here: first, our framework predicts the overall scaling of $F^{*}$ and $H^{*}$ independently of $\alpha$ and this correctly matches data, and second, both the asymptotic behavior and slope of $F^{*}$ and $H^{*}$ are independent of $\alpha$, such that our prediction of the maximum mammal size does not depend on $\alpha$.
\\
%For the growth rate $\lambda$ we consider the standard model of $\ln\left(\upsilon\right)/t_{\lambda}$

%More complicated models of fecundity (which, for example, account for the average length of adulthood and the number of individuals produced over this span) could be employed. However, the scaling of population growth rate has been studied in detail before and follows a relationship of $$ which matches the theory well for $\phi=.95$ and $=$.

%In our calculations we include $20\%$ variation around this value which could account for differences in efficiency during



 \begin{table}[h]
\caption{Parameter values for mammals}
\label{param}
    \begin{center}
    \footnotesize
     \begin{tabular}{p{3.8cm} c p{2.2cm} p{1.4cm}}
     \hline
    
     Definition & Parameter & Value & References  \\
     \hline
   Asymptotic adult mass & $M$ & (g) &  \\
   Initial mass of an organism & $m_{0}$ & (g) &  \\
   Metabolic rate scaling exponent & $\eta$ & $3/4$  &  (e.g. \citep{West:2001bv,moses2008rmo,hou}) \\
   Metabolic Normalization Constant & $B_{0}$ & $0.047$ (W g$^{-0.75}$)    & \citep{hou}  \\
   Initial mass scaling exponent & $\upsilon$ & $0.92$ &  \citep{blueweiss1978relationships,peters1986ecological} \\
   Initial mass scaling normalization constant & $n_{0}$ & $0.097$ (g$^{1-\upsilon}$) & \citep{blueweiss1978relationships,peters1986ecological}  \\   
   Fat mass scaling exponent & $\gamma$ & $1.19$ & \citep{Lindstedt:1985hm} \\
   Fat scaling normalization constant & $f_{0}$ & $0.02$ (g$^{1-\eta}$) & \citep{Lindstedt:1985hm}\\
   Muscle mass scaling exponent & $\zeta$ & $1.00$  & \citep{Lindstedt:2002td} \\
   Muscle scaling normalization constantv& $u_{0}$ & $0.38$ (g$^{1-\zeta}$)  & \citep{Lindstedt:2002td} \\
   Energy to synthesis a unit of mass & $E_{m}$ & $5774$ (J gram$^{-1}$)  &  \citep{moses2008rmo,West:2001bv,hou} \\
   Energy to synthesis a unit of mass during recovery & $E_{m}^{\prime}$ & $7000$ (J gram$^{-1}$) & \citep{stryer,hou} \\
   Specific resource growth rate & $\alpha$ & $9.45\times10^{-9}$ (s$^{-1}$) & see text  \\
   Fraction of asymptotic mass representing full state & $\epsilon_{\lambda}$ & $0.95$ & \citep{West:2001bv}  \\
   Fraction of asymptotic mass representing starving state & $\epsilon_{\sigma}$ & $1-f_{0}M^{\gamma-1}$ & see text  \\
   Fraction of asymptotic mass representing death & $\epsilon_{\mu}$ & $1-\frac{f_{0}M^{\gamma}+u_{0}M^{\zeta}}{M}$ & see text \\
   Carrying capacity (maximum density) of resources & $C$ & (g m$^{-2}$) & \\
   Half Saturation Constant & $k$ & (g m$^{-2}$) &   \\
   Normalized carrying capacity & $\xi$ & $C/k\approx2$ &   \\
   Reproductive fecundity & $\nu$ & $2$ & \citep{}  \\ 
   
   
%   $a$ & $1.78\times10^{-6}$  \quad \quad \\
%   $\lambda_{0}$ & $3.39\times10^{-7}$ (s$^{-1}$ gram$^{1-\eta}$) \quad \quad \\
   

   \hline
    \end{tabular}
    \end{center}
   \end{table}



{\bf Rate equations for invaders with modified body mass}
We allow an invading subset of the resident population with mass $M$ to have an altered mass $M^\prime = M(1+\chi)$ where $\chi$ varies between $\chi_{\rm min} <0$ and $\chi_{\rm max}>0$, where $\chi<0$ denotes a leaner invader and $\chi > 0$ denotes an invader with additional reserves of body fat.
Importantly, we assume that the invading and resident individuals have the same proportion of non-fat tissues.
For the allowable values of $\chi$ the adjusted mass should exceed the amount of body fat, $1+\chi>\epsilon_{\sigma}$, and the adjusted time to reproduce must be positive, which given our solution for $\tau(\epsilon)$ (see main text), implies that $1-\epsilon_{\lambda}^{1-\eta}\left(1+\chi\right)^{1-\eta}>0$.
Together these conditions imply that  $\chi\in(-f_0M^{\gamma-1},1/\epsilon_{\lambda}-1)$ where the upper bound approximately equals $0.05$.

Although the starved state of invading organisms remains unchanged, the rate of starvation from the modified full state to the starved state, the rate of recovery from the starved state to the modified full state, and the maintenance rates of both, will be different, such that $\sigma^\prime = \sigma(M^\prime)$, $\rho^\prime = \rho(M^\prime)$, $\beta^\prime = \beta(M^\prime)$, $\delta^\prime = \delta(M^\prime)$.
Rates of starvation and recovery for the invading population are easily derived by adjusting the starting or ending state before and after starvation and recovery, leading to the following timescales:

\begin{align}
t_{\sigma^\prime} &= -\frac{M^{1-\eta}}{a^{\prime}}\ln \left(\frac{\epsilon_\sigma}{\chi +1}\right), \\ \nonumber
t_{\rho^\prime} &= \ln \left(\frac{1-(\epsilon_\lambda \epsilon_\sigma)^{1/4}}{1-( \epsilon_\lambda(\chi +1))^{1/4}}\right)\frac{M^{1-\eta}}{a^{\prime}\left(1-\eta\right)}.
\end{align}


The maintenance rates for the invading population require more careful consideration.
First, we must recalculate the yields $Y$, as they must now be integrated over life stages that have also been slightly modified by the addition or subtraction of body fat reserves.
Given an energy density of $E_{d}=18200$ (J g$^{-1}$) for grass \citep{estermann} the maintenance value of the invading population is given by $P_{F}=B_{0}(1+\chi)M^{3/4}/(1+\chi)ME_{d}$, and the yield for a full organism will be given by $Y_{F}=(1+\chi)ME_{d}/B^{\prime}_{\lambda}$ (g individual $\cdot$ g grass $^{-1}$) where $B^{\prime}_{\lambda}$ is the lifetime energy use for the invading population reaching maturity given by
\begin{equation}
B^{\prime}_{\lambda}=\int_{0}^{t_{\lambda^\prime}}B_{0}m\left(t\right)^{\eta}dt.
\end{equation}
where
\begin{equation}
t_{\lambda^\prime} = \frac{M^{1-\eta} }{a(1-\eta)}\ln \left(\frac{1-(m_0/M)^{1-\eta}}{1-(\epsilon_\lambda (1+\chi))^{1-\eta}} \right).
\end{equation}
Note that we do not use this timescale to determine the reproductive rate of the invading consumer---which is assumed to remain the same as the resident population---but only to calulate the lifetime energy use.
Similarly, the maintenance for hungry individuals $P^\prime_{H}=B_{0}(\epsilon_{\sigma}(1+\chi)M)^{3/4}/(\epsilon_{\sigma}(1+\chi)M)E_{d}$ and the yield for hungry individuals (representing the cost on resources to return to the full state) is given by $Y^\prime_{H}=(1+\chi)ME_{d}/B^{\prime}_{\rho}$ where
\begin{equation}
B^{\prime}_{\rho}=\int_{\tau\left(\epsilon_{\sigma}\epsilon_{\lambda}\right)}^{t_{\lambda^\prime}}B_{0}m\left(t\right)^{\eta}dt.
\end{equation}
Finally, we can calculate the maintenance of the invaders as

\begin{align}
  \delta^\prime &= P^\prime_{H}Y^\prime_{H}/\xi \\ \nonumber
  \beta^\prime &= \left(\frac{\lambda_{\rm max}}{Y^\prime_{F}}+P^\prime_{F} \right)Y^\prime_{H}/\xi.
\end{align}

To determine whether or not the invader or resident population has an advantage, we compute $R^*(M)$ and $R^*(M^\prime=M(1+\chi))$ for values of $\chi \in (-f_0M^{\gamma-1},1/\epsilon_{\lambda}-1)$, and the invading population is assumed to have an advantage over the resident population if $R^*(M^\prime)<R^*(M)$.


{\bf Sensitivity to additional death terms}

It should be noted that our set of dynamics (Equations \ref{bigdynamics} and \ref{reduceddynamics}) could include a constant death term of the form $-d_{F}F$ and $-d_{H}H$ to represent death not directly linked to starvation. Adding terms of this form to our model would simply adjust the effective value of $\lambda$ and $\mu$, and we could rewrite Equation \ref{reduceddynamics} with $\lambda^{\prime}=\lambda-d$ and $\mu^{\prime}=\mu-d$. These substitutions would not alter the functional form of our model nor the steady-states and qualitative results, however the quantitative values could shift based on the size of $d$ relative to $\lambda$ and $\mu$. 

Survivorship has a well-known functional form which changes systematically with size (e.g. \cite{calder1984}). Typically survivorship is defined using the Gompertz curve 
\begin{equation}
F=F_{0}e^{\left(c_{0}/c_{1}\right)\left(1-e^{c_{1}t}\right)}
\label{gompertz}
\end{equation}
where the parameters have the following allometric dependencies on adult mass $c_{0}=a_{0}M^{b_{0}}$ and $c_{1}=a_{1}M^{b_{1}}$, with $a_{0}=1.88\times10^{-8}$ (s g$^{-b_{0}}$), $b_{0}=-0.56$, $a_{1}=1.45\times10^{-7}$ (s g$^{-b_{1}}$), and $b_{1}=-0.27$ (see \cite{calder1984} for a review).

\begin{figure}
\centering
\includegraphics[width=0.4\textwidth]{mortality-rate-comparison-eps-converted-to.pdf}
\caption{\small{The rates of reproduction $\lambda$ (blue), starvation-based mortality $\mu$ (red), and survivorship-based death $\bar{d}$ (black) as a function of adult mass.}\label{fig:ratescomp}}
\end{figure}

We are interested in the specific death rate of the form $\dot{F}=-dF$, and using the derivative of Equation \ref{gompertz} we find that $d=c_{0}e^{c_{1}t}$. Our model considers the average rates over a population and lifecycle and the average death rate is given by 
\begin{eqnarray}
\bar{d}&=&\frac{1}{t_{\text{exp}}}\int_{0}^{t_{\text{exp}}}c_{0}e^{c_{1}t} dt \\
&=&\frac{c_{0}\left(e^{c_{1} t_{\text{exp}}}-1\right)}{c_{1}t_{\text{exp}}}
\end{eqnarray}
where $t_{\text{exp}}$ is the expected lifespan following the allometry of $t_{\text{exp}}=a_{2}M^{b_{2}}$ with $a_{2}=4.04\times10^{6}$ (s g$^{-b_{2}}$) and $b_{2}=0.30$ ~\cite{damuth1982analysis,calder1984}. Given the allometries above we have that
\begin{equation}
\bar{d}=\frac{a_{0} \left(e^{a_{1}a_{2}M^{b_{1}+b_{2}}}-1\right) M^{b_{0}-b_{1}-b_{2}}}{a_{1} a_{2}}
\end{equation}
which scales roughly like $M^{b_{0}}$ because $b_{1}$ and $b_{2}$ are close in value but opposite in sign. In Figure S\ref{fig:ratescomp} we compare the value of $\bar{d}$ to the reproductive, $\lambda$, and starvation-based mortality, $\mu$, rates. The values of $\bar{d}$ are orders of magnitude smaller than these rates for all mammalian masses, and thus, adding this non-starvation based death rate to our model does not shift our results within numerical confidence. 

\begin{figure}[h!]
\centering
\includegraphics[width=0.4\textwidth]{fig_FPenergyequiv-eps-converted-to.pdf}
\caption{\small{ Total energetic use $B_{\rm tot}$ of consumer populations at the steady state as a function of body mass ($F^*$ is shown in green and $H^*$ in orange).  The data are from Damuth \citep{Damuth:1987kr} and have been converted to
  total population metabolism using the allometric relationships for
  metabolic rate (e.g. Refs.~\citep{West:2001bv,hou,moses2008rmo}).}\label{fig:equivalence}}
\end{figure}

{\bf NSM and the energy equivalence hypothesis}

The energy equivalence hypothesis is based on the observation that if one assumes that the total metabolism of an ecosystem $B_{\rm tot}$ is equally partitioned between all species ($B_{i}$, the total metabolism of one species, is a constant), then the abundances should follow $N\left(M\right)B\left(M\right)=B_{i}$ implying that $N\left(M\right)\propto M^{-\eta}$, where $\eta$ is the metabolic scaling exponent \citep{allen2002,enquist1998}. As $\eta \approx 3/4$ this hypothesis is consistent with Damuth's law \citep{allen2002}. However, the actual equivalence of energy usage of diverse species has not been measured at the population level for a variety of whole populations. Figure S\ref{fig:equivalence} recasts the results of the NSM in terms of this hypothesis and shows that $F^{*}B$ is nearly constant over the same range of mammalian sizes up to the asymptotic behavior for the largest terrestrial mammals. 



{\bf Application of NSM limits to aquatic mammals}
A theoretical upper bound on mammalian body size is given by $\epsilon_\sigma=0$, where mammals are entirely composed of metabolic reserves, and this occurs at $M=8.3\times 10^8$ (g), or $120$ times the mass of a male African elephant. We note this particular limit as it may have future relevance to considerations of the ultimate constraints on aquatic mammals.

% \putbib[aa_starving_supplement]

\def\bibfont{\footnotesize}


% \bibliography{aa_starving_supplement}
\begin{thebibliography}{10}
\expandafter\ifx\csname url\endcsname\relax
  \def\url#1{\texttt{#1}}\fi
\expandafter\ifx\csname urlprefix\endcsname\relax\def\urlprefix{URL }\fi
\providecommand{\bibinfo}[2]{#2}
\providecommand{\eprint}[2][]{\url{#2}}

\bibitem{Kempes:2012hy}
\bibinfo{author}{Kempes, C.~P.}, \bibinfo{author}{Dutkiewicz, S.} \&
  \bibinfo{author}{Follows, M.~J.}
\newblock \bibinfo{title}{{Growth, metabolic partitioning, and the size of
  microorganisms.}}
\newblock \emph{\bibinfo{journal}{PNAS}} \textbf{\bibinfo{volume}{109}},
  \bibinfo{pages}{495--500} (\bibinfo{year}{2012}).

\bibitem{kempes2014morphological}
\bibinfo{author}{Kempes, C.~P.}, \bibinfo{author}{Okegbe, C.},
  \bibinfo{author}{Mears-Clarke, Z.}, \bibinfo{author}{Follows, M.~J.} \&
  \bibinfo{author}{Dietrich, L.~E.}
\newblock \bibinfo{title}{Morphological optimization for access to dual
  oxidants in biofilms}.
\newblock \emph{\bibinfo{journal}{Proceedings of the National Academy of
  Sciences}} \textbf{\bibinfo{volume}{111}}, \bibinfo{pages}{208--213}
  (\bibinfo{year}{2014}).

\bibitem{West:2001bv}
\bibinfo{author}{West, G.~B.}, \bibinfo{author}{Brown, J.~H.} \&
  \bibinfo{author}{Enquist, B.~J.}
\newblock \bibinfo{title}{{A general model for ontogenetic growth}}.
\newblock \emph{\bibinfo{journal}{Nature}} \textbf{\bibinfo{volume}{413}},
  \bibinfo{pages}{628--631} (\bibinfo{year}{2001}).

\bibitem{moses2008rmo}
\bibinfo{author}{Moses, M.~E.} \emph{et~al.}
\newblock \bibinfo{title}{{Revisiting a model of ontogenetic growth: Estimating
  model parameters from theory and data}}.
\newblock
  \emph{\bibinfo{journal}{http://dx.doi.org.proxy.lib.sfu.ca/10.1086/679735}}
  \textbf{\bibinfo{volume}{171}}, \bibinfo{pages}{632--645}
  (\bibinfo{year}{2008}).

\bibitem{hou}
\bibinfo{author}{Hou, C.} \emph{et~al.}
\newblock \bibinfo{title}{{Energy uptake and allocation during ontogeny}}.
\newblock \emph{\bibinfo{journal}{Science}} \textbf{\bibinfo{volume}{322}},
  \bibinfo{pages}{736--739} (\bibinfo{year}{2008}).

\bibitem{pirt}
\bibinfo{author}{Pirt, S.}
\newblock \bibinfo{title}{The maintenance energy of bacteria in growing
  cultures}.
\newblock \emph{\bibinfo{journal}{Proceedings of the Royal Society of London B:
  Biological Sciences}} \textbf{\bibinfo{volume}{163}},
  \bibinfo{pages}{224--231} (\bibinfo{year}{1965}).

\bibitem{Heijnen}
\bibinfo{author}{Heijnen, J.} \& \bibinfo{author}{Roels, J.}
\newblock \bibinfo{title}{A macroscopic model describing yield and maintenance
  relationships in aerobic fermentation processes}.
\newblock \emph{\bibinfo{journal}{Biotechnology and Bioengineering}}
  \textbf{\bibinfo{volume}{23}}, \bibinfo{pages}{739--763}
  (\bibinfo{year}{1981}).

\bibitem{peters1986ecological}
\bibinfo{author}{Peters, R.~H.}
\newblock \emph{\bibinfo{title}{The Ecological Implications of Body Size}},
  vol.~\bibinfo{volume}{2} (\bibinfo{publisher}{Cambridge University Press},
  \bibinfo{address}{Cambridge}, \bibinfo{year}{1986}).

\bibitem{blueweiss1978relationships}
\bibinfo{author}{Blueweiss, L.} \emph{et~al.}
\newblock \bibinfo{title}{Relationships between body size and some life history
  parameters}.
\newblock \emph{\bibinfo{journal}{Oecologia}} \textbf{\bibinfo{volume}{37}},
  \bibinfo{pages}{257--272} (\bibinfo{year}{1978}).

\bibitem{stryer}
\bibinfo{author}{Stryer, L.}
\newblock \emph{\bibinfo{title}{{Biochemistry, Fourth Edition}}}
  (\bibinfo{publisher}{W.H. Freeman and Company}, \bibinfo{address}{New York},
  \bibinfo{year}{1995}).

\bibitem{Dunbrack:1993ec}
\bibinfo{author}{Dunbrack, R.~L.} \& \bibinfo{author}{Ramsay, M.~A.}
\newblock \bibinfo{title}{{The Allometry of Mammalian Adaptations to Seasonal
  Environments: A Critique of the Fasting Endurance Hypothesis}}.
\newblock \emph{\bibinfo{journal}{Oikos}} \textbf{\bibinfo{volume}{66}},
  \bibinfo{pages}{336--342} (\bibinfo{year}{1993}).

\bibitem{Lindstedt:1985hm}
\bibinfo{author}{Lindstedt, S.~L.} \& \bibinfo{author}{Boyce, M.~S.}
\newblock \bibinfo{title}{{Seasonality, Fasting Endurance, and Body Size in
  Mammals}}.
\newblock \emph{\bibinfo{journal}{Am. Nat.}} \textbf{\bibinfo{volume}{125}},
  \bibinfo{pages}{873--878} (\bibinfo{year}{1985}).

\bibitem{Lindstedt:2002td}
\bibinfo{author}{Lindstedt, S.~L.} \& \bibinfo{author}{Schaeffer, P.~J.}
\newblock \bibinfo{title}{{Use of allometry in predicting anatomical and
  physiological parameters of mammals.}}
\newblock \emph{\bibinfo{journal}{Lab. Anim.}} \textbf{\bibinfo{volume}{36}},
  \bibinfo{pages}{1--19} (\bibinfo{year}{2002}).

\bibitem{estermann}
\bibinfo{author}{Estermann, B.~L.}, \bibinfo{author}{Wettstein, H.-R.},
  \bibinfo{author}{Sutter, F.} \& \bibinfo{author}{Kreuzer, M.}
\newblock \bibinfo{title}{Nutrient and energy conversion of grass-fed dairy and
  suckler beef cattle kept indoors and on high altitude pasture}.
\newblock \emph{\bibinfo{journal}{Animal Research}}
  \textbf{\bibinfo{volume}{50}}, \bibinfo{pages}{477--493}
  (\bibinfo{year}{2001}).

\bibitem{michaletz2014convergence}
\bibinfo{author}{Michaletz, S.~T.}, \bibinfo{author}{Cheng, D.},
  \bibinfo{author}{Kerkhoff, A.~J.} \& \bibinfo{author}{Enquist, B.~J.}
\newblock \bibinfo{title}{Convergence of terrestrial plant production across
  global climate gradients}.
\newblock \emph{\bibinfo{journal}{Nature}} \textbf{\bibinfo{volume}{512}},
  \bibinfo{pages}{39--43} (\bibinfo{year}{2014}).

\bibitem{damuth1987interspecific}
\bibinfo{author}{Damuth, J.}
\newblock \bibinfo{title}{Interspecific allometry of population density in
  mammals and other animals: the independence of body mass and population
  energy-use}.
\newblock \emph{\bibinfo{journal}{Biological Journal of the Linnean Society}}
  \textbf{\bibinfo{volume}{31}}, \bibinfo{pages}{193--246}
  (\bibinfo{year}{1987}).

\bibitem{calder1984}
\bibinfo{author}{Calder, W.~A.}
\newblock \emph{\bibinfo{title}{Size, function, and life history}}
  (\bibinfo{publisher}{Harvard University Press}, \bibinfo{year}{1984}).

\bibitem{damuth1982analysis}
\bibinfo{author}{Damuth, J.}
\newblock \bibinfo{title}{Analysis of the preservation of community structure
  in assemblages of fossil mammals}.
\newblock \emph{\bibinfo{journal}{Paleobiology}} \textbf{\bibinfo{volume}{8}},
  \bibinfo{pages}{434--446} (\bibinfo{year}{1982}).

\bibitem{allen2002}
\bibinfo{author}{Allen, A.~P.}, \bibinfo{author}{Brown, J.~H.} \&
  \bibinfo{author}{Gillooly, J.~F.}
\newblock \bibinfo{title}{{Global biodiversity, biochemical kinetics, and the
  energetic-equivalence rule}}.
\newblock \emph{\bibinfo{journal}{Science}} \textbf{\bibinfo{volume}{297}},
  \bibinfo{pages}{1545--1548} (\bibinfo{year}{2002}).

\bibitem{enquist1998}
\bibinfo{author}{Enquist, B.~J.}, \bibinfo{author}{Brown, J.~H.} \&
  \bibinfo{author}{West, G.~B.}
\newblock \bibinfo{title}{{Allometric scaling of plant energetics and
  population density}}.
\newblock \emph{\bibinfo{journal}{Nature}} \textbf{\bibinfo{volume}{395}},
  \bibinfo{pages}{163--165} (\bibinfo{year}{1998}).

\bibitem{Damuth:1987kr}
\bibinfo{author}{Damuth, J.}
\newblock \bibinfo{title}{{Interspecific allometry of population density in
  mammals and other animals: the independence of body mass and population
  energy-use}}.
\newblock \emph{\bibinfo{journal}{Biol. J. Linn. Soc.}}
  \textbf{\bibinfo{volume}{31}}, \bibinfo{pages}{193--246}
  (\bibinfo{year}{1987}).

\end{thebibliography}


\end{bibunit}




\end{document}
