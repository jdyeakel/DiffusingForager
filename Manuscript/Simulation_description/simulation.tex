\documentclass[11pt]{iopart}
\usepackage{amssymb}
\usepackage{epsfig}
\usepackage{subfigure}
\usepackage{graphicx}
\usepackage{textcomp}
\usepackage{cite}
\usepackage{float}
\makeatletter
\makeatother

\expandafter\let\csname equation*\endcsname\relax
\expandafter\let\csname endequation*\endcsname\relax
\usepackage{amsmath}
\newcommand{\agt}{\mathrel{\raise.3ex\hbox{$>$\kern-.75em\lower1ex\hbox{$\sim$}}}}
\begin{document}

\title{Event-Driven Simulation of Foraging Dynamics}


\begin{abstract}

  We outline an event-driven simulation for the foraging model, which
  involves a resource that renews by logistic growth, as well as two classes
  of foragers---full and hungry.  Full foragers reproduce at a fixed rate and
  are not vulnerable to mortality.  However, a full forager can become hungry
  when resources are scarce; conversely, a hungry forager can become full
  when the resource is abundant.  Hungry foragers do not reproduce and die at
  a fixed rate.
\end{abstract}

%\maketitle

\section{The Model}

We assume that foragers can exist in two discrete states---full and hungry.
Full foragers $F$ are those that have just encountered and consumed a unit of
resource $R$.  On the other hand, a full forager that does not encounter a
resource as it wanders converts to a hungry forager $H$ with rate equal to
the product of a parameter $\sigma$ and the density of non-resources.
Whenever a forager, either full or hungry, encounters a resource, one unit of
the resource is consumed.  If the forager was hungry, it turns into a full
forager with rate $\rho$.  During the time that a forager is hungry, it dies
with mortality rate $\mu$, while full foragers do not experience mortality
risk.  Furthermore, full foragers reproduce with rate $\lambda$.  Finally, we
assume that, in the absence of foragers, the underlying resource undergoes
logistic growth, with growth rate $\alpha$ and carrying capacity equal to
one.

According to these processes and also under the assumption that the densities
of full foragers, hungry foragers, and resources (also denoted by $F$,
$H$, and $R$, respectively) are perfectly mixed, they evolve according to the
rate equations:
\begin{align}
  \label{RE}
\begin{split}
\dot F &= \lambda F + \rho  RH - \sigma (1-R)F\,,\\
\dot H &= \sigma (1-R)F - \rho RH - \mu H\,, \\
\dot R &= \alpha R(1-R) -  R(F+H),\\
\end{split}
\end{align}
where the overdot denotes time derivative.

We now outline an event-driven algorithm that mimic these rate equations.  We
will also generalize to the situation where the full and hungry foragers
undergo diffusion with possibly different diffusion coefficients on a
finite-dimensional lattice.  Suppose that the system at some time consists of
$N_F$ full foragers, $N_H$ hungry foragers, and $N_R$ individual resources.
The total number of particles $N=N_F+N_H+N_R$.

In each elemental event an individual is picked: a full forager with
probability $N_F/N$, a hungry forager with probability $N_H/N$, and a
resource with probability $N_R/N$.  If an $F$ is picked, it reproduces at
rate $\lambda$ and becomes hungry with rate $\sigma(1-R)$.  If an $H$ is
picked, it becomes full with rate $\rho R$ and dies with rate $\mu$.
Finally, if an $R$ is picked, it grows with rate $\alpha(1-R)$ and is eaten
with rate $(F+H)$.  Thus the total rate for each elemental event is
\begin{equation}
\mathcal{R}=F\big[\lambda +\sigma(1-R)\big]+ H\big[\rho R+\mu\big]+
R\big[\alpha(1-R)+(F+H)\big]\,.
\end{equation}

Now let's look at how the system evolves according to its constituent
processes.  If an $F$ is picked, then it may either reproduce or go hungry
according to the steps outlined above.  That is:
\begin{align*}
&\mathrm{growth,~ prob.\ } \lambda/\mathcal{R}\hskip 0.82 in N_F\to N_F\!+\!1\,,\\
&\mathrm{starve,~ prob.\ } \sigma(1-R)/\mathcal{R} \hskip 0.35 in N_F\to N_F\!-\!1,\,
  N_H\to N_H\!+\!1\,.
\end{align*}
Similarly, if an $H$ is picked, it may either become full or die following the
processes given above.  That is:
\begin{align*}
&\mathrm{become\ full,~  prob.\ } \rho R/\mathcal{R}\hskip 0.4in N_H\to N_H\!-\!1,\,
  N_F\to N_F\!+\!1\,,\\
&\mathrm{die,~ prob.\ } \mu/\mathcal{R}\hskip 1.12 in N_H\to N_H\!-\!1\,.
\end{align*}
Finally, if an $R$ is picked, it may either grow or be eaten following the
processes given above.  That is:
\begin{align*}
&\mathrm{grow,~ prob.\ } \alpha(1\!-\!R)/\mathcal{R}\hskip 0.8in N_R\to N_R\!+\!1\,,\phantom{~~~~~~~~~~}\\
&\mathrm{eaten,~ prob.\ } (F\!+\!H)/\mathcal{R}\hskip 0.82in  N_R\to N_R\!-\!1\,.\phantom{~~~~~~~~~~}
\end{align*}

We now determine the change in the expected number of individuals of each
type in a single event that are consistent with the above processes.  For
full foragers, this change is
\begin{subequations}
\begin{equation}
  \Delta N_F =\left[\frac{N_F}{N}\big(\lambda-\sigma(1-R)\big)+\frac{N_H}{N}\rho
      R\right]/\mathcal{R}\,.
\end{equation}
The term proportional to $N_F$ comes from processes in which a full forager
is picked while the term proportional to $N_H$ comes from processes in which
a hungry forager is picked and it converted to a full forager.  Consequently,
the change in the density of full foragers simply is
\begin{equation}
  \Delta F =\left[F\big(\lambda-\sigma(1-R)\big)+H\rho
    R\right]/N\mathcal{R}\,.
\end{equation}
\end{subequations}
Thus if we take the time step for each event to be
$\Delta t = (N\mathcal{R})^{-1}$, the above reduces to the rate equation
\eqref{RE} for $F$.  Thus in each microscopic event of the model, the time
should be advanced by $\Delta t = (N\mathcal{R})^{-1}$.

In a similar fashion, the change in the expected number of hungry foragers in
a single event is given by
\begin{subequations}
\begin{equation}
  \Delta N_H =\left[-\frac{N_H}{N}(\rho R+\mu)+\frac{N_F}{N}\sigma
    (1-R)\right]/\mathcal{R}\,,
\end{equation}
so that the change in the density of hungry foragers simply is
\begin{equation}
  \Delta H =\left[-H(\rho R+\mu)+F\sigma (1-R)\right]/N\mathcal{R}\,.
\end{equation}
\end{subequations}
Finally, the change in the expected number of individual resources in a
single event is given by
\begin{subequations}
\begin{equation}
  \Delta N_R =\left[\frac{N_R}{N}\big(\alpha(1-R)\big)-\frac{(N_F+N_H)}{N}\right]/\mathcal{R}\,,
\end{equation}
so that the change in the density of resources is
\begin{equation}
  \Delta R =\left[R\big(\alpha(1-R)\big)-(F+H)\right]/N\mathcal{R}\,.
\end{equation}
\end{subequations}
The equations for $\Delta F$, $\Delta H$, and $\Delta R$ then reproduce the
original rate equations \eqref{RE} when the time step for an elemental event
is taken to be $\Delta t = (N\mathcal{R})^{-1}$.

We can straightforwardly generalize the above approach to the situation where
the foragers are diffusing on a lattice.  Suppose that the full and hungry
foragers diffuse with respective diffusion coefficients $D_F$ and $D_H$.  In
this case, the rate equations \eqref{RE} generalize to the set of partial
differential equations
\begin{align}
  \label{pde}
\begin{split}
\frac{\partial F}{\partial t} &= \lambda F + \rho  RH - \sigma (1-R)F+D_F\nabla^2F\,,\\[0.125in]
\frac{\partial H}{\partial t}  &= \sigma (1-R)F - \rho RH - \mu H+D_H\nabla^2H\,, \\[0.125in]
\frac{\partial R}{\partial t}  &= \alpha R(1-R) -  R(F+H).\\
\end{split}
\end{align}
Here the densities $F,H,R$ are now functions of space and time,
$F=F(\mathbf{r},t)$ and similarly for $H$ and $R$.

We now determine the total rate for an elemental event.  If an $F$ is picked,
it reproduces at rate $\lambda$, becomes hungry with rate $\sigma(1-R)$, and
moves with rate $D_F$.  If an $H$ is picked, it becomes full with rate
$\rho R$, dies with rate $\mu$, and moves with rate $D_H$.  As before, if an
$R$ is picked, it grows with rate $\alpha(1-R)$ and is eaten with rate
$(F+H)$.  Thus the total rate for each elemental event is
\begin{equation}
\mathcal{R}_d=F\big[\lambda +\sigma(1-R)+D_F\big]+ H\big[\rho R+\mu+D_H\big]+
R\big[\alpha(1-r)+(F+H)\big]\,.
\end{equation}
Here the subscript $d$ indicates that this rate pertains to the foraging
model with diffusing foragers on a $d$-dimensional lattice.

As before, we determine at how the system evolves according to the various
constituent processes.  If an $F$ is picked, it may either reproduce, go
hungry, or move.  That is:
\begin{align*}
&\mathrm{growth,~ prob.\ } \lambda/\mathcal{R}_d\hskip 0.82 in N_F\to N_F\!+\!1\,,\\
&\mathrm{starve,~ prob.\ } \sigma(1-R)/\mathcal{R}_d \hskip 0.35 in N_F\to N_F\!-\!1,\,\\
&\mathrm{move,~ prob.\ } D_F/\mathcal{R}_d \hskip 0.8 in N_F\to N_F\,.
\end{align*}
Similarly, if an $H$ is picked, it may become full, die, or move following
the processes given above.  That is:
\begin{align*}
&\mathrm{become\ full,~  prob.\ } \rho R/\mathcal{R}_d\hskip 0.4in N_H\to N_H\!-\!1,\,
  N_F\to N_F\!+\!1\,,\\
&\mathrm{die,~ prob.\ } \mu/\mathcal{R}_d\hskip 1.12 in N_H\to N_H\!-\!1\,,\\
&\mathrm{move,~ prob.\ } D_H/\mathcal{R}_d\hskip 0.82 in N_H\to N_H\,.
\end{align*}
The evolution of the resource is the same as the mean-field case.  Following
the same logic as in the mean-field case, the evolution of the densities will
be described by the evolution equations \eqref{pde} when the time step of
each elemental event is taken to be $(N\mathcal{R}_d)^{-1}$.

There are several additional detailed subtleties of the simulation that are
worth mentioning.  First, it seems that there does not need to be any
constraint on the number of foragers on any site, so I think it's best to let
this number be arbitrary.  However, the carrying capacity of the resource has
been set to 1, so that the number of individual resources at any site should
be either 0 or 1.

Now let's turn to the mechanics of the various update events.  In the
mean-field limit, when a full forager reproduces, its offspring should be
placed anywhere.  For a finite-dimensional lattice, it seems reasonable to
place the offspring at the same site as the parent.  Diffusion of a full
forager on a lattice can be modeled by allowing it to hop to a nearest
neighbor; similarly for the motion of hungry foragers.

The mechanics of the resource evolution is a bit subtle.  In the mean-field
limit, if an individual resource is picked a new resource is created with
probability proportional to $1-R$, where $R$ is the global density of
resource, which is necessarily less than 1.  If a new resource is created, it
should be place randomly on any one of the sites that does not contain a
resource, but this site could be occupied by a forager.  On a lattice, the
creation of a new resource should be proportional to a local version of
$1-R$.  I think it's reasonable to consider the $z$ nearest-neighbor sites of
the current resource and put a new individual resource at one of the vacant
nearest neighbors with probability $1-{n}/{z}$, where $n$ is the number
of individual resources in the local neighborhood of $z$ sites.  The
disappearance of an individual resource by consumption is straightforward.


\end{document}

