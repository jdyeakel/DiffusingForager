\documentclass[11pt]{iopart}
\usepackage{amssymb}
\usepackage{epsfig}
\usepackage{subfigure}
\usepackage{graphicx}
\usepackage{textcomp}
\usepackage{cite}
\usepackage{float}
\makeatletter
\makeatother

\expandafter\let\csname equation*\endcsname\relax
\expandafter\let\csname endequation*\endcsname\relax
\usepackage{amsmath}
\newcommand{\agt}{\mathrel{\raise.3ex\hbox{$>$\kern-.75em\lower1ex\hbox{$\sim$}}}}
\begin{document}

\title{Event-Driven Simulation of Foraging Dynamics}


\begin{abstract}

  We outline an event-driven simulation for the foraging model, which
  involves a resource that renews by logistic growth, as well as two classes
  of foragers---full and hungry.  Full foragers reproduce at a fixed rate and
  are not vulnerable to mortality.  However, a full forager can become hungry
  when resources are scarce; conversely, a hungry forager can become full
  when the resource is abundant.  Hungry foragers do not reproduce and die at
  a fixed rate.
\end{abstract}

%\maketitle

\section{The Model}

We assume that foragers can exist in two discrete states---full and hungry.
Full foragers $F$ are those that have just encountered and consumed a unit of
resource $R$.  On the other hand, a full forager that does not encounter a
resource as it wanders converts to a hungry forager $H$ with rate equal to
the product of a parameter $\sigma$ and the density of non-resources.
Whenever a forager, either full or hungry, encounters a resource, one unit of
the resource is consumed.  If the forager was hungry, it turns into a full
forager with rate $\rho$.  During the time that a forager is hungry, it dies
with mortality rate $\mu$, while full foragers do not experience mortality
risk.  Furthermore, full foragers reproduce with rate $\lambda$.  Finally, we
assume that, in the absence of foragers, the underlying resource undergoes
logistic growth, with growth rate $\alpha$ and carrying capacity equal to
one.

According to these processes and also under the assumption that the densities
of full foragers, hungry foragers, and resources (also denoted by $F$,
$H$, and $R$, respectively) are perfectly mixed, they evolve according to the
rate equations:
\begin{align}
  \label{RE}
\begin{split}
\dot F &= \lambda F + \rho  RH - \sigma (1-R)F\,,\\
\dot H &= \sigma (1-R)F - \rho RH - \mu H\,, \\
\dot R &= \alpha R(1-R) -  R(F+H),\\
\end{split}
\end{align}
where the overdot denotes time derivative.  An important feature is the
assumption that the carrying capacity is set equal to 1.  This means that
when the system is completely occupied by resources there can be no further
generation of the resource.  The approach described below can be generalized
to carrying capacity less than 1, but one can take the carrying capacity
equal to 1 without loss of generality.  Here is another subtle point worth
mentioning: $\dot R$ does not involve the recruitment rate $\rho$.  From the
perspective of the resource, either type of forager can consume it.  However,
from the perspective of the hungry forager, once it has eaten a resource, it
converts to a full forager with rate $\rho$.  One has to be careful to not
overcount the events that involve interactions between a hungry forager and a
resource.

We now outline an event-driven algorithm that mimic these rate equations. 
Suppose that the system at a given time consists of $N_F$ full foragers,
$N_H$ hungry foragers, and $N_R$ individual resources.  The total number of
particles $N=N_F+N_H+N_R$.  There are six basic processes embodied by the
rate equations \eqref{RE}:
\begin{enumerate}
\item Reproduction:  $F\to 2F$
\item Starvation:  $F\to H$
\item Recruitment: $H\to F$ and $R\to 0$
\item Death: $H\to 0$
\item Growth: $R\to 2R$
\item Consumption: $R\to 0$
\end{enumerate}
Note that recruitment also involves consumption of a resource, so we must be
careful to not double count.  So I define ``Recruitment'' as the consumption
of the resource by a hungry forager and ``Consumption'' as the consumption of
the resource by a full forager.

In an event-driven simulation, one of the above events is picked with the
appropriate probability (to be defined below), the selected event is
implemented, and then the time is updated accordingly.  The total rate for
any event is proportional to
\begin{equation}
\mathcal{R}=F\big[\lambda +\sigma(1-R)\big]+ H\big[\rho R+\mu\big]+
R\big[\alpha(1-R)+(F+H)\big]\,.
\end{equation}
Here $F=N_F/V$, $H=N_H/V$, and $R=N_R/V$, where $N_i$ is the total number of
particles of type $i$, and $V$ is the total number of lattice sites.  Thus $F$,
$H$, and $R$ are the densities of each type of entity.  The
total rate is defined only up to an overall constant, but this constant is
immaterial because the time step is also proportional to this same constant.

At the simplest level one merely keeps three variables: $N_F$, $N_H$, and
$N_R$ and updates these numbers according to the steps outlined below.
However, we want a simulation that can be directly adapted to treat foragers
that diffuse on a lattice.  Thus the algorithm outlined below is based on
populating a lattice with the three types of entities and then updating the
occupancy on the lattice sites according to the following event-driven
simulation:
\begin{enumerate}
\item Reproduction.  With probability $\lambda F/\mathcal{R}$, pick one of
  the full foragers and allow it to reproduce.  The offspring should be
  placed anywhere in the system; this is consistent with there being no
  spatial structure.
\item Starvation.  With probability $\sigma(1-R)F/\mathcal{R}$, pick a full
  forager and change it to a hungry forager.  The factor $1-R$ is the
  fraction of resource-free space.
\item Recruitment.  With probability $\rho HR/\mathcal{R}$, pick a hungry
  forager, turn it into a full forager, and remove the resource.
\item Death.  With probability $\mu H/\mathcal{R}$, pick a hungry forager and
  remove it.
\item Growth.  With probability $\alpha R(1-R)/\mathcal{R}$, pick a resource
  unit and create another unit of resource anywhere in the system.
\item Consumption.  With probability $(F+H)R/\mathcal{R}$, pick a resource
  unit and remove it.
\end{enumerate}
By explicit construction, the probabilities for the above events sum to 1.
Note also that a good test of the correctness of the above lattice-based
algorithm is to compare it with the three-variable update rule that is
mentioned at the start of this paragraph.

We now determine the change in the expected number of individuals of each
type in a single event.  For full foragers, the change in its number $N_F$ is
\begin{subequations}
\begin{equation}
  \Delta N_F =\left[F\big(\lambda-\sigma(1-R)\big)+\rho H
    R\right]/\mathcal{R}\,.
\end{equation}
The terms proportional to $F$ describes processes in which a full forager is
picked, while the term proportional to $H$ describes processes in which a
hungry forager is picked and converted to a full forager.  Thus the change in
the density of full foragers is
\begin{equation}
  \Delta F =\frac{\Delta N_F}{V} =\left[F\big(\lambda-\sigma(1-R)\big)+H\rho
    R\right]/V\mathcal{R}\,.
\end{equation}
\end{subequations}
Thus if we take the time step for each event to be
$\Delta t = (V\mathcal{R})^{-1}$, the above reduces to the rate equation
\eqref{RE} for $F$.  Thus in each microscopic event of the model, the time
should be advanced by $\Delta t = (V\mathcal{R})^{-1}$.

In a similar fashion, the change in the expected number of hungry foragers in
a single event is given by
\begin{subequations}
\begin{equation}
  \Delta N_H =\left[-H(\rho R+\mu)+F\sigma
    (1-R)\right]/\mathcal{R}\,,
\end{equation}
so that the change in the density of hungry foragers simply is
\begin{equation}
  \Delta H =\left[-H(\rho R+\mu)+F\sigma (1-R)\right]/V\mathcal{R}\,.
\end{equation}
\end{subequations}
Finally, the change in the expected number of individual resources in a
single event is given by
\begin{subequations}
\begin{equation}
  \Delta N_R =\left[R\big(\alpha(1-R)\big)-(F+H)\right]/\mathcal{R}\,,
\end{equation}
so that the change in the resource density is
\begin{equation}
  \Delta R =\left[R\big(\alpha(1-R)\big)-(F+H)\right]/V\mathcal{R}\,.
\end{equation}
\end{subequations}
The expressions for $\Delta F/\Delta t$, $\Delta H/\Delta t$, and
$\Delta R/\Delta t$ reproduce the original rate equations \eqref{RE} when the
time step for an elemental event is $\Delta t = (V\mathcal{R})^{-1}$.  To
summarize, pick an event according to the probabilities enumerated above and
update the densities according (3b), (4b), and (5b), and then increment the
time by $(V\mathcal{R})^{-1}$ after each event.  In the three-variable
simulation, we keep only $N_F$, $N_H$, and $N_R$.  In an update, pick an
event according to the above probabilities, update $N_F$, $N_H$, and $N_R$
appropriately,and then increment the time.

There are several aspects of this algorithm that need to be specified for a
lattice simulation.  First, there is no constraint on the number of foragers
on any site, but given that we are describing a harsh environment, it seems
that the parameters should be chosen so that this number is not large.  Since
the carrying capacity of the resource has been set to 1, this means that
$N_R$ cannot exceed $V$.  However, the number of resource units on any given
site can exceed 1.  With these details, a simulation that is based on the
above steps should reproduce the predictions of the rate equations
\eqref{RE}.

We now generalize the above approach to the situation where the foragers are
diffusing on a lattice.  Suppose that the full and hungry foragers diffuse
with respective diffusion coefficients $D_F$ and $D_H$.  In this case, the
rate equations \eqref{RE} generalize to the reaction-diffusion equations
\begin{align}
  \label{pde}
\begin{split}
\frac{\partial F}{\partial t} &= \lambda F + \rho  RH - \sigma (1-R)F+D_F\nabla^2F\,,\\[0.125in]
\frac{\partial H}{\partial t}  &= \sigma (1-R)F - \rho RH - \mu H+D_H\nabla^2H\,, \\[0.125in]
\frac{\partial R}{\partial t}  &= \alpha R(1-R) -  R(F+H)\,.\\
\end{split}
\end{align}
Here the densities $F,H,R$ are now functions of space and time,
$F=F(\mathbf{r},t)$ and similarly for $H$ and $R$.

To construct an event-driven simulation that mimic these rate equations, we
have to include two additional processes:
\begin{enumerate}
\item[(vii)] Diffusion of full foragers
\item[(viii)] Diffusion of hungry foragers
\end{enumerate}
In an event-driven simulation, one of the eight possible events is picked
with the appropriate probability (see below), the selected event is
implemented, and the time is updated accordingly.

The total rate for any event is
\begin{align}
\label{R}
\mathcal{R}=&\frac{1}{V}\sum_{\mathbf{r}}\Big\{F(\mathbf{r})\big[\lambda
  +\sigma(1-R(\mathbf{r}))+D_F\big]
+ H(\mathbf{r})\big[\rho R(\mathbf{r})+\mu+D_H\big]\Big.\nonumber \\[-0.05in]
&\Big.\hskip 0.5in
  +R(\mathbf{r})\big[\alpha(1-R(\mathbf{r'}))+(F(\mathbf{r})+H(\mathbf{r}))\big]\Big\}\,\nonumber
  \\[0.1in]
&=F\big[\lambda +\sigma(1-R)+D_F\big]
+ H\big[\rho R+\mu+D_H\big]+R\big[\alpha(1-R)+(F+H)\big]\,,
\end{align}
in which the role of diffusion is now included.  Notice also the presence of
a two-particle correlation in the term $R(\mathbf{r})(1-R(\mathbf{r}'))$,
where $\mathbf{r'}$ is a neighbor of $\mathbf{r}$.  Thus going to the second
line of the above equation is not quite kosher, but below I give a
prescription to deal with this issue in a way that avoids using a correlation
function.

The steps of an event-driven simulation are:
\begin{enumerate}
\item Reproduction.  With probability $\lambda F/\mathcal{R}$, pick one of
  the full foragers at random and let it reproduce.  Since the process occurs
  on a finite-dimensional lattice, the offspring should be placed close to
  the parent.  This could be either on the same site or a neighboring site.
\item Starvation.  With probability $\sigma(1-R)F/\mathcal{R}$, pick a random
  full forager that is on a resource-free site and change the forager from
  full to hungry.  Note the two-particle nature of this rule: we need both
  the forager type and the resource state on a given site.  There is a way to
  get around tracking sites that are simultaneously resource free and
  occupied by a full forager.  This leads to the alternative rule:
\item[(ii')] Starvation'.  With probability $\sigma F/\mathcal{R}$, pick a
  random full forager.  If the site is resource free, change the forager from
  full to hungry.  If not, nothing happens.
\item Recruitment.  With probability $\rho HR/\mathcal{R}$, pick a random
  hungry forager on a site that contains a resource unit and change the
  forager from hungry to full and remove the resource.  This process also
  involves a two-body reaction and perhaps a simpler way to implement this
  two-body feature is the following:
\item[(iii')] Recruitment'.  With probability $\rho H/\mathcal{R}$, pick a
  random hungry forager.  If the site contains a resource unit, change the
  forager from hungry to full and remove the resource.
\item Death.  With probability $\mu H/\mathcal{R}$, pick and remove a random
  hungry forager.
\item Growth.  With probability $\alpha R(1-R)/\mathcal{R}$, pick a random
  neighboring pair of sites in which one contains a resource unit and the
  neighbor does not.  Create a resource on the site without a resource.  A
  computationally simpler alternative is the following:
\item[(v)'] Growth.  With probability $\alpha R/\mathcal{R}$, pick a random
  resource unit.  A new resource unit is created on one of the neighboring
  sites with probability $1-m/z$, where $z$ is the number of
  nearest-neighbors and $m$ is the number of these neighbors that are
  occupied by resource.  Thus if the local neighborhood is devoid of
  resources, growth necessarily occurs at one of these neighbors, while if
  the local neighborhood is full of resources, there is no growth.  This rule
  ensures that the growth rate is proportional to $(1-R)_{\rm local}$.
\item Consumption.  With probability $(F+H)\,R/\mathcal{R}$, pick a random
  resource unit on a site that is also occupied by a forager of either type
  and remove the resource  Alternatively:
\item[(vi')] With probability $R/\mathcal{R}$, pick a random resource unit.
  If the site also contains a forager, remove the resource.  Note the
  implicit assumption that the density of foragers is not large because the
  rate of resource removal does not increase if there are multiple foragers
  on the site.  If this forager density is large then the following will
  work:
\item[(vi'')] With probability $(F+H)/\mathcal{R}$, pick a random forager
  (either full or hungry).  If the site also contains a resource, remove the
  resource.  This prescription works if the forager density is small or large
  (as long as the resource density is less than 1) so perhaps it's best to
  use this last alternative.
\item Diffusion.  With probability $D_F\, F/\mathcal{R}$, pick a random full
  forager and move it to a neighboring site.
\item Diffusion.  With probability $D_H\, H/\mathcal{R}$, pick a random
  hungry forager and move it to a neighboring site.
\end{enumerate}
By explicit construction, the probabilities for the above events sum to 1.
Note also that the steps (ii'), (iii'), (v'), and (vi'') are computationally
simpler than the unprimed steps, but allow for the possibility of a null
step.  A true event-driven simulation avoids all null steps, but perhaps the
computational tradeoff in bookkeeping simplicity is worth the loss of
efficiency in allowing for null steps.  This is somewhat a matter of user
taste.

We now determine the change in the expected number of individuals of each
type at a given site $\mathbf{r}$ in a single event.  For full foragers, the
change in its number $N_F(\mathbf{r})$ at site $\mathbf{r}$ is
\begin{subequations}
\begin{equation}
  \Delta N_F(\mathbf{r})  =\Big\{F(\mathbf{r})\big(\lambda-\sigma(1-R(\mathbf{r}))\big)+
\rho H(\mathbf{r})  R(\mathbf{r})
+D_F\big[\sum_{\mathbf{r}'}F(\mathbf{r}')-zF(\mathbf{r})\big]\Big\}/\mathcal{R}\,.
\end{equation}
The terms proportional to $F$ accounts for processes in which a full forager
is picked, while the term proportional to $H$ accounts for processes in which
a hungry forager is picked and converts to a full forager.  The terms
proportional to $D_F$ account for nearest-neighbor hopping: $F(\mathbf{r})$
increases because full foragers at neighboring sites $\mathbf{r}'$ hop to
$\mathbf{r}$, while $F(\mathbf{r})$ decreases because full foragers at
$\mathbf{r}$ hop to one of the $z$ neighboring sites.  Thus the change in the
density of full foragers at site $\mathbf{r}$ is (after taking the continuum
limit for the nearest-neighbor hopping)
\begin{equation}
  \Delta F(\mathbf{r}) =\frac{\Delta N_F(\mathbf{r})}{V} 
  =\left[F(\mathbf{r})\big(\lambda-\sigma(1-R(\mathbf{r}))\big)+\rho H(\mathbf{r})
    R(\mathbf{r})+\nabla^2F(\mathbf{r})\right]/V\mathcal{R}\,.
\end{equation}
\end{subequations}
Thus if we take the time step for each event to be
$\Delta t = (V\mathcal{R})^{-1}$, the above reduces to the partial
differential evolution equation \eqref{pde} for $F(\mathbf{r})$.  Thus in
each microscopic event of the model, the time should be advanced by
$\Delta t = (V\mathcal{R})^{-1}$.

In a similar fashion, the change in the density of hungry foragers is
\begin{equation}
  \Delta H(\mathbf{r}) =\left[-H(\mathbf{r})(\rho R(\mathbf{r})+\mu)+F(\mathbf{r})\sigma (1-R(\mathbf{r}))+\nabla^2H(\mathbf{r})\right]/V\mathcal{R}\,.
\end{equation}
Finally, the change in the resource density is
\begin{equation}
  \Delta R(\mathbf{r}) =\left[R(\mathbf{r})\big(\alpha(1-R(\mathbf{r}))\big)-(F(\mathbf{r})+H(\mathbf{r}))\right]/V\mathcal{R}\,.
\end{equation}
The equations for $\Delta F$, $\Delta H$, and $\Delta R$ would then reproduce
the partial differential equations \eqref{pde} when the time step for an
elemental event is taken to be $\Delta t = (V\mathcal{R})^{-1}$.


There are various aspects of the simulation that require discussion.  First,
single-particle and two-particle reactions must be treated differently.  In
the former case, a particle of the right type is picked at random.  In the
latter case, a random \emph{pair} of particles of the appropriate types is
picked.  Again, there is no constraint on the number of foragers on any site,
but given that we are describing a harsh environment, the parameters should
be chosen so that this number is not large.  The carrying capacity of the
resource has been set to 1, so that the number of individual resources at any
site must be either 0 or 1.

It would be nice if there was a simple way to adjust parameters in the
lattice simulation so as to recover the mean-field limit.  I describe a
simple modification of the steps in the event-driven lattice simulation that
I think should recover the mean-field limit.  We use the same rate as in
Eq.~\eqref{R}, but the actual update steps are slightly modified compared to
the rules of the lattice simulation above.  The differences are highlighted in
italics.  I'm also assuming that the simulation with steps (ii'), (iii'),
(v'), and (vi'') are being used.
\begin{enumerate}
\item Reproduction.  With probability $\lambda F/\mathcal{R}$, pick one of
  the full foragers at random and let it reproduce.  \emph{Place the
    offspring anywhere.}
\item Starvation.  With probability $\sigma F/\mathcal{R}$, \emph{change a
    random full forager to hungry with probability $(1-R)$.}
\item Recruitment.  With probability $\rho H/\mathcal{R}$, pick a random
  hungry forager and \emph{change it to full with probability $R$.}
\item Death.  With probability $\mu H/\mathcal{R}$, pick and remove a random
  hungry forager.
\item Growth.  With probability $\alpha R(1-R)/\mathcal{R}$, pick a random
  neighboring pair of sites in which one contains a resource unit and the
  neighbor does not contain a resource unit.  Create a resource on the site
  without a resource (see below for a simpler alternative).
\item With probability $(F+H)/\mathcal{R}$, pick a random forager
  (either full or hungry).  \emph{With probability $R$ remove a randomly
    chosen resource unity.}
\item Consumption.  With probability $(F+H)\,R/\mathcal{R}$, pick a random
  resource unit on a site that is also occupied by a forager of either type
  and remove the resource.
\item Diffusion.  With probability $D_F\, F/\mathcal{R}$, pick a random full
  forager and move it to a neighboring site.
\item Diffusion.  With probability $D_H\, H/\mathcal{R}$, pick a random
  hungry forager and move it to a neighboring site.
\end{enumerate}
By explicit construction, the probabilities for the above events sum to 1.

It is important to appreciate that there are multiple ways to simulate the
elemental reaction steps that involve a tradeoff between bookkeeping
complexity and computational efficiency and still reproduce the same behavior
as in the rate equations \eqref{RE} or the reaction-diffusion equations
\eqref{pde} in the limit of a large system.


\end{document}



