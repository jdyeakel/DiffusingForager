%\usepackage{scicite}
%\usepackage{times}
%\usepackage{graphicx}
%\usepackage{verbatim}
%\usepackage{subfigure}
%\usepackage{tikz}
%\usepackage{array}
%\usepackage{tabularx}
%\usepackage{multirow}
%\usepackage{multicol}
%\usepackage{multibox}
%\usepackage{rotating}
%\usepackage{layout}
%\usepackage{epstopdf}
%\usepackage{afterpage}
%\usepackage[left]{lineno}

%% PNAStmpl.tex
%% Template file to use for PNAS articles prepared in LaTeX
%% Version: Apr 14, 2008

%%%%%%%%%%%%%%%%%%%%%%%%%%%%%%
%% BASIC CLASS FILE

\documentclass{pnastwo}

%%%%%%%%%%%%%%%%%%%%%%%%%%%%%%
%% OPTIONAL GRAPHICS STYLE FILE

\usepackage[pdftex]{graphicx}

%%%%%%%%%%%%%%%%%%%%%%%%%%%%%%
%% OPTIONAL POSTSCRIPT FONT FILES

\usepackage{pnastwof}

%%%%%%%%%%%%%%%%%%%%%%%%%%%%%%
%% ADDITIONAL OPTIONAL STYLE FILES

%\usepackage{amssymb,amsfonts,amsmath}
\usepackage{amsmath}
\usepackage{amssymb}

\usepackage[font=large]{caption}
\usepackage{subfigure}
\usepackage{epstopdf}
\usepackage{graphicx}
\usepackage{verbatim}
\usepackage{array}
\usepackage{tabularx}
\usepackage{multirow}
\usepackage{color}
%\usepackage{multibox}
\usepackage{rotating}
\usepackage{color}
%\usepackage{lineno}
\usepackage[left]{lineno}
\usepackage[comma,sort&compress]{natbib}

%%%%%%%%%%%%%%%%%%%%%%%%%%%%%%
%% OPTIONAL MACRO FILES

\bibliographystyle{pnas}

%%%%%%%%%%%%%%%%%%%%%%%%%%%%%%

\newcommand{\sid}[1]{\textcolor{red}{\bf [#1]}}


%% Don't type in anything in the following section:
%%%%%%%%%%%%
%% For PNAS Only:
\contributor{Submitted to Proceedings
of the National Academy of Sciences of the United States of America}
\url{www.pnas.org/cgi/doi/10.1073/pnas.0709640104}
\copyrightyear{2011}
\issuedate{Issue Date}
\volume{Volume}
\issuenumber{Issue Number}
%%%%%%%%%%%%

\begin{document}

\linenumbers
\setlength\linenumbersep{3pt}


%\title{Ecological and evolutionary implications of starvation and body size}

%\title{Survival of the Stable: Dynamic Consequences of Starvation and Reproduction}

%\title{Size Constraints on Starvation \& Reproduction Dynamics}

%\title{Eco-evolutionary Consequences of Starvation and Reproduction: Survival
%of the Stable}

\title{Supporting Information for ``The dynamics of starvation and recovery''}%: Eco-evolutionary feedbacks}

\author
{Justin D. Yeakel\affil{1}{School of Natural Science, University of California Merced, Merced, CA} \affil{2}{The Santa Fe Institute, Santa Fe, NM}\affil{4}{To whom correspondence should be addressed: jdyeakel@gmail.com}, Christopher P. Kempes \affil{2}{The Santa Fe Institute, Santa Fe, NM}, \and Sidney Redner \affil{2}{The Santa Fe Institute, Santa Fe, NM}\affil{3}{Department of Physics, Boston University, Boston MA}
}

\contributor{Submitted to Proceedings of the National Academy of Sciences
of the United States of America}

\maketitle

\begin{article}

\section*{Mechanisms of Starvation and Recovery}
Our overall goal is to understand the dynamics of starvation, recovery, reproduction, and resource competition, where our framework partitions starvation and reproduction into two classes of the consumer: a full class that is able to reproduce and a hungry class that experiences mortality at a given rate and is unable to reproduce. For the dynamics of growth, reproduction, and resource consumption past efforts have combined the overall metabolic rate as dictated by body size with a growth rate that is dependent on resource abundance and in turn dictates resource consumption (see Refs. \cite{Kempes:2012hy,kempes2014morphological} for a brief review of this perspective). This approach has been used to understand a range of phenomena including a derivation of ontogenetic growth curves from a partitioning of metabolism into maintenance and biosynthesis (e.g. \cite{West:2001bv,moses2008rmo,hou,Kempes:2012hy}) and predictions of the steady-state resource abundance in communities of cells \cite{kempes2014morphological}. Here we can leverage these mechanisms with several additional concepts to define our nutritional state model.

We consider the following generalized set of explicit dynamics for starvation, recovery, reproduction, and resource growth and consumption
\begin{align}
\begin{split}
\dot{F_{d}} &= \lambda\left(R_{d}\right) F_{d} + \rho\left(R_{d}\right)H_{d} - \sigma \left(1-\frac{R_{d}}{C}\right)F_{d},  \\
\dot{H_{d}} &= \sigma \left(1-\frac{R_{d}}{C}\right)F_{d} - \rho\left(R_{d}\right)H_{d} - \mu H_{d},  \\
\dot{R_{d}} &= \alpha R_{d}\left(1-\frac{R_{d}}{C}\right) - \\
&\left[\left(\frac{\rho\left(R_{d}\right)}{Y}+P_{H}\right)H_{d}+\left(\frac{\lambda\left(R_{d}\right)}{Y}+P_{F}\right)F_{d}\right]
\end{split}
\end{align}
where each term has a mechanistic meaning that we detail below (we will denote the dimensional equations with $_{d}$ before introducing the nondimensional form which is what is later derived and presented in the main text).
%(note this is not fully explicit because I don't know how to deal with the response of $\sigma$ to resources, although I have an idea for a derivation which may be necessary given the following approximations)
In this set of equations $\lambda\left(R_{d}\right)$ and $\rho\left(R_{d}\right)$ are the growth and recovery functions as functions of the current resource availability. Typically these can be written as $\lambda\left(R_{d}\right)=\lambda_{max}S\left(R_{d}\right)$ or $\lambda\left(R_{d}\right)=\lambda_{max}S\left(R_{d}\right)$ where $\lambda_{max}$ and $\rho_{max}$ are the maximum growth and recovery rates respectively (each which scale with body size as discussed later), and $S\left(R_{d}\right)$ is a saturating function of resources, for example a Michaelis-Menten or Monod function of the form $\frac{R_{d}}{k+R_{d}}$, where $k$ is the half-saturation constant. A simplified version of the Michaelis-Menten or Monod functional form, which captures the essential features, is a linear function that saturates to a constant value above a certain abundance of $R_{d}$. In these equations $Y$ represents the yield coefficient (e.g. \cite{pirt,Heijnen}) which is the quantity of resources required to build a unit of organism (e.g. gram of mammal produced per gram of grass consumed) and $P$ is the specific maintenance rate of resource consumption (g resource $\cdot$ s$^{-1}$ $\cdot$ g organism). If we pick $F_{d}$ and $H_{d}$ to have units of (g organisms $\cdot$ m$^{-2}$), then all of the terms of $\dot{R_{d}}$, such as $\frac{\rho\left(R_{d}\right)}{Y}H_{d}$, have units of (g resource $\cdot$ m$^{-2}$ $\cdot$ s$^{-1}$) which are the units of net primary productivity (NPP) a natural choice for $\dot{R_{d}}$. This choice also gives $R_{d}$ as (g $\cdot$ m$^{-2}$) which is also a natural unit and is simply the biomass density. In this system of units $\alpha$ (s$^{-1}$) is the specific growth rate of $R_{d}$ and $C$ is the carrying capacity or maximum density of $R_{d}$ in a particular environment.

Before describing the values of each of these constants and a general nondimensionalization of the system of equations, it is important to consider the resource regimes associated with the above equations as this leads to a simplification. As discussed above, the resource saturation function should be defined by a linear regime proportional to $R_{d}$ when $R_{d}<<k$ and a constant value for $R_{d}>>k$. Thus for hungry individuals, $H_{d}$, where $R_{d}<<k$, we have that $\rho\left(R_{d}\right)\approx\rho_{max}R_{d}/k$, and for the full class, $F_{d}$, of organisms $\lambda\left(R_{d}\right)\approx\lambda_{max}$ such that the above relationships reduce to
\begin{align}
\begin{split}
\dot{F_{d}} &= \lambda_{max} F_{d} + \rho_{max}R_{d}H_{d}/k - \sigma \left(1-\frac{R_{d}}{C}\right)F_{d},  \\
\dot{H_{d}} &= \sigma \left(1-\frac{R_{d}}{C}\right)F_{d} - \rho_{max}R_{d} H_{d}/k - \mu H_{d},  \\
\dot{R_{d}} &= \alpha R_{d}\left(1-\frac{R_{d}}{C}\right) -\\
& \left[\left(\frac{\rho_{max}R_{d}}{Y_{H}k}+P_{H}\right)H_{d}+\left(\frac{\lambda_{max}}{Y_{F}}+P_{F}\right)F_{d}\right].
\end{split}
\end{align}
%where $\beta=\frac{\lambda_{max}}{Y_{F}}+P$ which is just a constant that depends on the size of an organisms via the allometries for $\lambda_{max}$ and $P$ discussed later.

We can formally nondimensionalize this system by choosing the general rescaling of $F=fF_{d}$, $H=fH_{d}$, $R=qR_{d}$, $t=st_{d}$, in which case our system of equations becomes
%(ignoring the $\sigma (1-R)F$ terms which I don't have a dimensional form for yet),:
\begin{align}
\begin{split}
\dot{F} &= \frac{1}{s}\left[\lambda_{max} F + \rho_{max}\frac{R}{qk}H - \sigma \left(1-\frac{R}{qC}\right)F\right],  \\
\dot{H} &= \frac{1}{s}\left[\sigma \left(1-\frac{R}{qC}\right)F - \rho_{max}\frac{R}{qk} H - \mu H\right],  \\
\dot{R} &= \frac{1}{s}\left[\alpha R\left(1-\frac{R}{qC}\right) -\frac{q}{f}\left[\left(\frac{\rho_{max}R}{Y_{H}kq}+P_{H}\right)H+\left(\frac{\lambda_{max}}{Y_{F}}+P_{F}\right) F\right]\right].
\end{split}
\end{align}
If we make the natural choice of $s=1$, $q=1/C$, and $f=1/Y_{H}k$, then we are left with
\begin{align}
\begin{split}
\dot{F} &= \lambda F + \xi \rho RH - \sigma \left(1-R\right)F,  \\
\dot{H} &= \sigma \left(1-R\right)F - \xi \rho RH - \mu H,  \\
\dot{R} &= \alpha R\left(1-R\right) -\left(\rho R+\delta\right)H-\beta F
\end{split}
\end{align}
where we have dropped the subscripts on $\lambda_{max}$ and $\rho_{max}$ for simplicity, and $\xi=C/k$, $\delta=Y_{H}kP_{H}/C$, and $\beta=Y_{H}k\left(\frac{\lambda_{max}}{Y_{F}}+P_{F}\right)/C$. The above equations represent the system of equations presented in the main text.

\section*{Parameter Values and Estimates}

%\subsection*{Standard synthesis and metabolic parameters}

All of the parameter values employed in our model have either been directly measured in previous studies or can be estimated from combining several previous studies. Below we outline previous measurements and simple estimates of the parameters.

Metabolic rate has been generally reported to follow an exponent close to $\eta=0.75$ (e.g. \cite{West:2001bv,moses2008rmo} and the supplement of \cite{hou}). We make this assumption in the current paper, although alternate exponents, which are know to vary between roughly $0.25$ and $1.5$ for single species \cite{moses2008rmo}, could be easily incorporated into our framework, and this variation is effectively handled by the $20\%$ variations that we consider around mean trends. It is important to note the exponent, because it not only defines several scalings in our framework but also the value of the metabolic normalization constant, $B_{0}$, given a set of data.  For mammals the metabolic normalization constant has been reported to vary between $0.018$ (W g$^{-0.75}$) and $0.047$ (W g$^{-0.75}$) \cite{hou,West:2001bv}, where the former value represents basal metabolic rate and the latter represents the field metabolic rate. We employ the field metabolic rate for our NSM model which is appropriate for active mammals (Table 1).

An important feature of our is the starting size, $m_{0}$, of a mammal which adjusts the overall timescales for reproduction. This starting size is known to follow an allometric relationship with adult mass of the form ... where estimates for ... range between ... \cite{}. We choose ....

The energy to synthesize a unit of biomass, $E_{m}$, has been reported to vary between $1800$ to $9500$ (J g$^{-1}$) (e.g. \cite{West:2001bv,moses2008rmo,hou}) in mammals with a mean value across many taxonomic groups of $5,774$ (J g$^{-1}$) \cite{moses2008rmo}. The unit energy available during starvation, $E^{\prime}$, could range between $7000$ (J g$^{-1}$), the return of the total energy stored during ontogeny \cite{hou} to a biochemical upper bound of $E^{\prime}=36,000$ (J g$^{-1}$) for the energetics of palmitate \cite{stryer,hou}. For our calculations we use the measured value for bulk tissues of $7000$ which assumes that the energy stored during ontogeny is returned during starvation \cite{hou}.

For the scaling of body composition it has been shown that fat mass follows $M_{\rm fat}=f_{0}M^{\gamma}$, with measured  relationships following  $0.018M^{1.25}$ \cite{Dunbrack:1993ec}, $0.02M^{1.19}$ \cite{Lindstedt:1985hm}, and $0.026M^{1.14}$ \cite{Lindstedt:2002td}. We use the values from \cite{Lindstedt:1985hm} which falls in the middle of this range. Similarly, the muscle mass follows $M_{\rm musc}=u_{0}M^{\zeta}$ with $u_{0}=0.383$ and $\zeta=1.00$ \cite{Lindstedt:2002td}.

We also connect the resource growth rate to the total metabolic rate of an organism. That is, we are interested in the relative rates of resource recovery and consumption by the total population. From \cite{allen2002global} the total resource use of a population with an individual body size of $M$ is given by $B_{pop}=0.00061x^{-0.03}$ (W m$^{-2}$). Considering an energy density of $18200$ (J g$^{-1}$) of grass \cite{estermann} and an NPP between and $1.59\times10^{-6}$ and $7.92\times10^{-5}$ (g s$^{-1}$ m$^{-2}$) would give a range of resource rates between  $0.029$ and $1.44$ (W m$^{-2}$). This gives a ratio of total resource consumption to supply rates between $0.00042$ and $0.021$, and we used a value of $0.002$ in our calculations and simulations.

Typically the value of $\xi=C/k$ should roughly be $2$. The value of $\rho$, $\lambda$, $\sigma$, and $\mu$ are all simple rates (note that we have not rescaled time in our nondimensionalization) as defined in the maintext. Given that our model considers transitions over entire stages of ontogeny or nutritional states the value of $Y$ must represent yields integrated over entire life stages. Given an energy density of $E_{d}=18200$ (J g$^{-1}$) for grass \cite{estermann} the maintenance value is given by $P_{F}=B_{0}M^{3/4}/ME_{d}$, and the yield for a full organism will be given by $Y_{F}=ME_{d}/B_{\lambda}$ (g individual $\cdot$ g grass $^{-1}$) where $B_{\lambda}$ is the lifetime energy use for reaching maturity given by
\begin{equation}
B_{\lambda}=\int_{0}^{t_{\lambda}}B_{0}m\left(t\right)^{\eta}dt.
\end{equation}
Similarly, the maintenance for hungry individuals $P_{H}=B_{0}(\epsilon_{\sigma}M)^{3/4}/(\epsilon_{\sigma}M)E_{d}$ and the yield for hungry individuals (representing the cost on resources to return to the full state) is given by $Y_{H}=ME_{d}/B_{\rho}$ where
\begin{equation}
B_{\rho}=\int_{\tau\left(\epsilon_{\sigma}\epsilon_{\lambda}\right)}^{t_{\lambda}}B_{0}m\left(t\right)^{\eta}dt.
\end{equation}
which, in combination, allows us to calculate $\delta$ and $\beta$.

Finally, the value of $\alpha$ can be roughly estimated by the NPP divided by the corresponding biomass densities. This has a range of $Value$ to $Value$ globally. It should be noted that the value of $\alpha$ sets the overall scale of the $F^{*}$ and $H^{*}$ steady states along with the $B_{tot}$ for each type, and as such, we use $\alpha$ as our fit parameter such that these steady states match the scale of know data from Damuth \cite{damuth1987interspecific}. We find that the best fit is $\alpha=9.45\times10^{-9}$ (s$^{-1}$). However, two points are important to note here: first, our overall framework predicts the overall scaling of $F^{*}$ and $H^{*}$ independently of $\alpha$ and this correctly matches data, and second, the asymptotic behavior of $F^{*}$ and $H^{*}$ is also independent of $\alpha$, that is, our prediction of the maximum mammal size does not depend on $\alpha$.

%For the growth rate $\lambda$ we consider the standard model of $\ln\left(\upsilon\right)/t_{\lambda}$

%More complicated models of fecundity (which, for example, account for the average length of adulthood and the number of individuals produced over this span) could be employed. However, the scaling of population growth rate has been studied in detail before and follows a relationship of $$ which matches the theory well for $\phi=.95$ and $=$.

%In our calculations we include $20\%$ variation around this value which could account for differences in efficiency during

 \begin{table}[h]
\caption{Parameter values for mammals}
\label{param}
    \begin{center}
    \small
     \begin{tabular}{ p{1.2cm} p{3.2cm} l p{2.2cm}|}
     \hline
     Parameter & Value & References  \\
     \hline
   $\eta$ & $3/4$  &  (e.g. \cite{West:2001bv,moses2008rmo,hou}) \\
   $E_{m}$ & $5774$ (J gram$^{-1}$)  &  \cite{moses2008rmo,West:2001bv,hou} \\
   $E_{m}^{\prime}$ & $36,000$  & \cite{stryer,hou} \\
   $B_{0}$ & $0.047$ (W g$^{-0.75}$)    & \cite{hou}  \\
%   $a$ & $1.78\times10^{-6}$  \quad \quad \\
%   $\lambda_{0}$ & $3.39\times10^{-7}$ (s$^{-1}$ gram$^{1-\eta}$) \quad \quad \\
   $\gamma$ & $1.19$ & \cite{Lindstedt:1985hm} \\
   $f_{0}$ & $0.02$ & \cite{Lindstedt:1985hm}\\
   $\zeta$ & $1.00$  & \cite{Lindstedt:2002td} \\
   $u_{0}$ & $0.38$  & \cite{Lindstedt:2002td} \\

   \hline
    \end{tabular}
    \end{center}
   \end{table}


\subsection*{Rate equations for invaders with modified body mass}
We allow an invading subset of the resident population with mass $M$ to have an altered mass $M^\prime = M(1+\chi)$ where $\chi$ varies between $chi_{\rm min} <0$ and $chi_{\rm max}>0$, where $\chi<0$ denotes a leaner invader and $\chi > 0$ denotes an invader with additional endogenous reserves.
Importantly, we assume that the invading and resident individuals have the same proportion of non-fat tissues.
Thus $\chi_{\rm}$ is limited by the proportion of lean mass, such that $\chi_{\rm min} = -f_0M^{\gamma-1}$.
Similarly, we assume that the invading organisms do not add endogenous reserves above the asymptyotic mass of the species $M$, such that $(1+\chi)\epsilon_\lambda M < M$, and $\chi_{\rm max} \approx 0.05$.

Although the starved state of invading organisms remains unchanged, the rate of starvation from the modified full state to the starved state, the rate of recovery from the starved state to the modified full state, and the maintenance rates of both, will be different, such that $\sigma^\prime = \sigma(M^\prime)$, $\rho^\prime = \rho(M^\prime)$, $\beta^\prime = \beta(M^\prime)$, $\delta^\prime = \delta(M^\prime)$.
Rates of starvation and recovery for the invading population are easily derived by adjusting the starting or ending state before and after starvation and recovery, leading to the following timescales:

\begin{align}
t_{\sigma^\prime} &= \frac{-M^{1/4}}{B_0/E_m^\prime}\log \left(\frac{\epsilon_\sigma}{\chi +1}\right), \\ \nonumber
t_{\rho^\prime} &= \frac{-4 M^{1/4} }{B_0/E_m^\prime}\log \left(\frac{1-( \epsilon_\lambda(\chi +1))^{1/4}}{1-(\epsilon_\lambda \epsilon_\sigma)^{1/4}}\right).
\end{align}


The maintenance rates for the invading population require more careful consideration.
First, we must recalculate the yields $Y$, as they must now be integrated over life stages that have also been slightly modified by the addition or subtraction of endogenous reserves.
Given an energy density of $E_{d}=18200$ (J g$^{-1}$) for grass \cite{estermann} the maintenance value of the invading population is given by $P_{F}=B_{0}(1+\chi)M^{3/4}/(1+\chi)ME_{d}$, and the yield for a full organism will be given by $Y_{F}=(1+\chi)ME_{d}/B^{\prime}_{\lambda}$ (g individual $\cdot$ g grass $^{-1}$) where $B^{\prime}_{\lambda}$ is the lifetime energy use for the invading population reaching maturity given by
\begin{equation}
B^{\prime}_{\lambda}=\int_{0}^{t_{\lambda^\prime}}B_{0}m\left(t\right)^{\eta}dt.
\end{equation}
where
\begin{equation}
t_{\lambda^\prime} = \frac{-4 M^{1/4} }{B_0/E_m}\log \left(\frac{1-(m_0/M)^{1/4}}{(1-\epsilon_\lambda (1+\chi))^{1/4}} \right).
\end{equation}
Note that we do not use this timescale to determine the reproductive rate of the invading consumer---which is assumed to remain the same as the resident population---but only to calulate the lifetime energy use.
Similarly, the maintenance for hungry individuals $P^\prime_{H}=B_{0}(\epsilon_{\sigma}(1+\chi)M)^{3/4}/(\epsilon_{\sigma}(1+\chi)M)E_{d}$ and the yield for hungry individuals (representing the cost on resources to return to the full state) is given by $Y^\prime_{H}=(1+\chi)ME_{d}/B^{\prime}_{\rho}$ where
\begin{equation}
B^{\prime}_{\rho}=\int_{\tau\left(\epsilon_{\sigma}\epsilon_{\lambda}\right)}^{t_{\lambda^\prime}}B_{0}m\left(t\right)^{\eta}dt.
\end{equation}
Finally, we can calculate the maintenance of the invaders as

\begin{align}
  \delta^\prime &= P^\prime_{H}Y^\prime_{H}/\xi \\ \nonumber
  \beta^\prime &= \left(\frac{\lambda_{\rm max}}{Y^\prime_{F}}+P^\prime_{F} \right)Y^\prime_{H}/\xi
\end{align}


\bibliography{aa_starving_supplement}







%\let\cleardoublepage\clearpage

%FIGURE 1




\end{article}





\end{document}
