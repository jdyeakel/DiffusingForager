\documentclass[ucm,12pt]{ucletter}


\usepackage{setspace}
\usepackage{graphicx}

\setlength{\parindent}{5ex}
\linespread{1.2}

%\name{Justin D. Yeakel}
\telephone{(209)~285-9571}
\email{jyeakel@ucmerced.edu}

\begin{document}
\begin{letter}{
    \\ Nature Communications\\
    Nature Publishing Group\\
    The Macmillan Building\\
    4 Crinan Street\\ 
    London N1 9XW\\
    United Kingdom\\ \\
    \vspace{1mm}
    \centerline{\bf{Re: The Dynamics of Starvation and Recovery}} \\
}


\opening{To the Editorial Board at \emph{Nature Communications},}



Please find attached the manuscript entitled "The dynamics of starvation and recovery" co-authored by Justin Yeakel, Christopher Kempes, and Sid Redner, which we would like to submit for publication in \emph{Nature Communications}.

A longstanding question in ecology concerns how individuals invest energy in somatic growth and maintenance vs. reproduction. Specific solutions to this problem contribute greatly to the the diversity of species’ life histories, however a  solid connection between somatic energetics and reproduction with an understanding of how these processes influence population dynamics (and vice versa) is lacking. In the accompanying manuscript we introduce a principled mathematical Nutritional State-structured Model (NSM), which describes how reproduction, starvation, and recovery from starvation, influence the dynamics of resource and consumer populations. Within this framework we incorporate allometric relationships for all timescales and rates, and in so doing connect the NSM to realistic mammalian populations. Our macro-scale model makes a number of predictions concerning the dynamic implications of energetic rate laws for consumer populations and helps elucidate important limits and potential drivers of body size evolution. We make several contributions:



\begin{itemize}
  \item Across the observed body size range, terrestrial mammals occupy a region of parameter space where sustained cyclic dynamics are not permitted. Relevant to observations of cycles in natural populations, the NSM predicts  that small mammal populations should be more prone to — whereas large mammals are buffered against — large transient oscillations, and this is a consequence of the timescales of starvation vs. recovery, both of which are governed by metabolism.
  \item Even transient cyclic dynamics increase the probability of extinction, and we argue that mammals should have energetic rates that avoid parameter regions close to transient oscillatory regimes. Moreover, regions of parameter space with small steady-state population sizes also increase the extinction probability. Thus we define a ``refuge'' of parameter space where the extinction probability is minimized between high probabilities of extinction via transience and high probabilities of extinction via low steady state densities, and show that mammals have allometrically constrained rates of starvation and recovery that place them squarely within this lower-risk space.
  \item Steady-state population sizes in the NSM support the energy equivalence hypothesis (equal energy use across populations of species with different individual body size) for a wide range of body sizes. Our model predicts that this energy equivalence breaks down at the smallest and largest body sizes observed for mammals, suggesting that there are strong dynamical constraints on mammalian body size. 
  \item We further investigate dynamical constraints on the range of mammalian body sizes by considering the invasion of a steady-state population by mammals with altered amounts of endogenous energetic reserves. We show that mammals with greater amounts of endogenous reserves are able to invade up to the size of the largest recorded mammal (ca. $10^7$g) above which leaner individuals can invade, such that this maximum body size bound serves as an evolutionary stable state. This suggests that the dynamics of starvation and recovery described in the NSM may provide a within-lineage mechanism for the evolution of larger body size among terrestrial mammals, a well-documented phenomenon known as Cope’s rule. This result is also supported by our finding that the steady state resource abundance is decreasing for larger mammals, which, in the context of classic resource competition theory, suggests that larger mammals will outcompete smaller ones given a shared resource. 
\end{itemize}


Our theory is analytic, relatively simple, and synthesizes a variety of fundamental ecological theories (energy equivalence hypothesis, resource competition theory, Cope’s rule, the fasting endurance hypothesis) and shows how many of these are the natural consequence of a simple dynamical framework that incorporates the energetic consequences of starvation and recovery combined with allometric timescales. We believe that our findings will be of general interest to ecologists, evolutionary biologists, and paleoecologists, such that consideration for publication in a broad interest journal such as PNAS is warranted.

We appreciate your consideration of our manuscript.

Suggested Referees:
\begin{itemize}
\item[] Alan Hastings, UC Davis, amhastings@ucdavis.edu
\item[] Felisa Smith, University of New Mexico, fasmith@unm.edu
\item[] Richard Sibly, University of Reading, r.m.sibly@reading.ac.uk
\item[] John DeLong, University of Nebraska, jpdelong@unl.edu
\item[] Seth Finnegan, Integrative Biology, UC Berkeley, sethf@berkeley.edu
\item[] Van Savage, University of California Los Angeles, vsavage@ucla.edu
\item[] Pablo Marquet, Department of Ecology, Pontificia Universidad Cat\`olica de Chile, pmarquet@bio.puc.cl
\end{itemize}

\vspace{5mm}

\singlespacing
\closing{Sincerely,\\
\fromsig{\includegraphics[scale=0.2]{signature.jpg}}\\
\fromname{
Justin D. Yeakel,\\
Christopher Kempes,\\
Sidney Redner}
}

\end{letter}
\end{document}
