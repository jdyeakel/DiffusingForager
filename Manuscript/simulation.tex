\documentclass[11pt]{iopart}
\usepackage{amssymb}
\usepackage{epsfig}
\usepackage{subfigure}
\usepackage{graphicx}
\usepackage{textcomp}
\usepackage{cite}
\usepackage{float}
\makeatletter
\makeatother

\expandafter\let\csname equation*\endcsname\relax
\expandafter\let\csname endequation*\endcsname\relax
\usepackage{amsmath}
\newcommand{\agt}{\mathrel{\raise.3ex\hbox{$>$\kern-.75em\lower1ex\hbox{$\sim$}}}}
\begin{document}

\title{Role of Starvation on Foraging Dynamics}


\author{J. D. Yeakel} \address{Santa Fe Institute, 1399 Hyde Park Road, Santa Fe, New
  Mexico 87501, USA}

\author{S. Redner} \address{Santa Fe Institute, 1399 Hyde Park Road, Santa Fe, New
  Mexico 87501, USA}

\begin{abstract}

  We outline an event-driven simulation approach for the foraging model,
  which involves a resource that renews by logistic growth, as well as two
  classes of foragers---full and hungry.  Full foragers reproduce at a fixed
  rate and are not vulnerable to mortality.  However, a full forager can
  starve when resources are scarce; conversely, a hungry forager can become
  full when the resource is abundant.  Hungry foragers do not reproduce and
  die at rate $\mu$.
\end{abstract}

%\maketitle

\section{The Model}

We assume that foragers can exist in two discrete states---full and hungry.
Full foragers $F$ are those that have just encountered and consumed a unit of
resource $R$.  On the other hand, a full forager that does not encounter
a resource as it wanders is converted into a hungry forager $H$ with rate
$\sigma$.  Whenever a forager, either full or hungry, encounters resources,
one unit of the resource is consumed.  If the forager was hungry, it is
recruited into the full population with rate $\rho$.  During the time
that a forager is hungry, it dies with a fixed mortality rate $\mu$, while
full foragers do not experience mortality risk.  Furthermore, full
foragers reproduce with rate $\lambda$.  Finally, we assume that, in the
absence of foragers, the underlying resource undergoes logistic growth, with
growth rate $\alpha$ and carrying capacity equal to one.

According to these processes and also under the assumption that the densities
of full foragers, hungry foragers, and resources (also denoted by $F$,
$H$, and $R$, respectively) are perfectly mixed, they evolve according to the
rate equations:
\begin{align}
  \label{RE}
\begin{split}
\dot F &= \lambda F + \rho  RH - \sigma (1-R)F\,,\\
\dot H &= \sigma (1-R)F - \rho RH - \mu H\,, \\
\dot R &= \alpha R(1-R) -  R(F+H),\\
\end{split}
\end{align}
where the overdot denotes time derivative.

We now outline our event-driven algorithm to simulate these rate equations.
We will also generalize to the situation where the full and hungry foragers
undergo diffusion with possibly different diffusion coefficients.  Suppose
that the system at some time consists of $N_F$ individuals of type $F$, $N_H$
individuals of type $H$, and $N_R$ individuals of type $R$.  The total number
of particles $N=N_F+N_H+N_R$.

In each simulation step an individual is picked: a full individual is picked
with probability $N_F/N$, a hungry individual is picked with probability
$N_H/N$, and a individual resource is picked with probability $N_R/N$.  If an
$F$ is picked, it reproduces at rate $\lambda$ and becomes hungry with rate
$\sigma(1-R)$.  If an $H$ is picked, it becomes full with rate $\rho R$ and
dies with rate $\mu$.  Finally, if an $R$ is picked, it grows with rate
$\alpha(1-R)$ and is eaten with rate $(F+H)$.  Thus the total rate for each
individual event is
\begin{equation}
\mathcal{R}=F\big[\lambda +\sigma(1-R)\big]+ H\big[\rho R+\mu\big]+
R\big[\alpha(1-r)+(F=H)\big]\,.
\end{equation}

Now let's look at how the system evolves according to the various constituent
processes.  If an $F$ is picked, then it may either reproduce or go hungry
according to the processes outlined above.
That is:
\begin{align*}
&\mathrm{growth,~ prob.\ } \lambda/\mathcal{R}\hskip 0.72 in N_F\to N_F\!+\!1\,,\\
&\mathrm{starve,~ prob.\ } \sigma(1-R)\mathcal{R} \hskip 0.35 in N_F\to N_F\!-\!1,\,
  N_S\to N_S\!+\!1\,.
\end{align*}
Similarly, if an $H$ is picked, it may either become full or die following the
processes given above.  That is:
\begin{align*}
&\mathrm{become\ full,~  prob.\ } \rho R/\mathcal{R}\hskip 0.3in N_H\to N_H\!-\!1,\,
  N_F\to N_F\!+\!1\,,\\
&\mathrm{die,~ prob.\ } \mu/\mathcal{R}\hskip 1 in N_H\to N_H\!-\!1\,.
\end{align*}
Finally, if an $R$ is picked, it may either grow or be eaten following the
processes given above.  That is:
\begin{align*}
&\mathrm{grow,~ prob.\ } \alpha(1\!-\!R)/\mathcal{R}\hskip 0.4in N_R\to N_R\!+\!1\,,\\
&\mathrm{eaten,~ prob.\ } (F\!+\!S)/\mathcal{R}\hskip 0.43in  N_R\to N_R\!-\!1\,.
\end{align*}

With these steps, let's now determine the change in the expected number of
full individuals in a single event.  This change is
\begin{subequations}
\begin{equation}
  \Delta N_F =\left[\frac{N_F}{N}\left(\lambda-\sigma(1-R)\right)+\frac{N_S}{N}\rho
      R\right]/\mathcal{R}\,.
\end{equation}
Consequently, the change in the density of full individuals simply is
\begin{equation}
  \Delta F =\left[\frac{N_F}{N}\left(\lambda-\sigma(1-R)\right)+\frac{N_S}{N}\rho
      R\right]/N\mathcal{R}\,.
\end{equation}
\end{subequations}
Thus is we take the time step to be $\Delta t = (N\mathcal{R})^{-1}$, the
above reduces to the rate equation \eqref{RE} for $F$.

In a similar fashion, the change in the expected number of hungry individuals
in a single event is given by 
\begin{subequations}
\begin{equation}
  \Delta N_H =\left[-\frac{N_H}{N}\left(\mu+\rho R\right)+\frac{N_F}{N}\sigma
    (1-R)\right]/\mathcal{R}\,,
\end{equation}
so that the change in the density of hungry individuals simply is
\begin{equation}
  \Delta H =\left[-\frac{N_H}{N}\left(\mu+\rho R\right)+\frac{N_F}{N}\sigma (1-R)\right]/N\mathcal{R}\,.
\end{equation}
\end{subequations}
Finally, the change in the expected number of individual resources in a
single event is given by
\begin{subequations}
\begin{equation}
  \Delta N_R =\left[\frac{N_R}{N}\left(\alpha(1-R)\right)-\frac{N_F}{N}-\frac{N_H}{N}\right]/\mathcal{R}\,,
\end{equation}
so that the change in the density of resources is
\begin{equation}
  \Delta R ==\left[\frac{N_R}{N}\left(\alpha(1-R)\right)-\frac{N_F}{N}-\frac{N_H}{N}\right]/N\mathcal{R}\,.
\end{equation}
\end{subequations}



\end{document}
