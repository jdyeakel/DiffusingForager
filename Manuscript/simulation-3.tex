\documentclass[11pt]{iopart}
\usepackage{amssymb}
\usepackage{epsfig}
\usepackage{subfigure}
\usepackage{graphicx}
\usepackage{textcomp}
\usepackage{cite}
\usepackage{float}
\makeatletter
\makeatother

\expandafter\let\csname equation*\endcsname\relax
\expandafter\let\csname endequation*\endcsname\relax
\usepackage{amsmath}
\newcommand{\agt}{\mathrel{\raise.3ex\hbox{$>$\kern-.75em\lower1ex\hbox{$\sim$}}}}
\begin{document}

\title{Event-Driven Simulation of Foraging Dynamics}


\begin{abstract}

  We outline an event-driven simulation for the foraging model, which
  involves a resource that renews by logistic growth, as well as two classes
  of foragers---full and hungry.  Full foragers reproduce at a fixed rate and
  are not vulnerable to mortality.  However, a full forager can become hungry
  when resources are scarce; conversely, a hungry forager can become full
  when the resource is abundant.  Hungry foragers do not reproduce and die at
  a fixed rate.
\end{abstract}

%\maketitle

\section{The Model}

We assume that foragers can exist in two states---full and hungry.  When a
full forager $F$ encounters a resource $R$, the resource is consumed with
rate $m$ so that the forager maintains its fullness.  A full forager that
does not encounter a resource as it wanders converts to a hungry forager $H$
with rate equal to the product of the starvation rate $\sigma$ and the
density of non-resources.  Full foragers also reproduce with rate
$\lambda$. When a hungry forager encounters a resource, the resource is again
consumed and the forager becomes full with rate $\rho$.  A hungry forager
dies with mortality rate $\mu$, while full foragers have no mortality risk.
We assume that in the absence of foragers the underlying resource undergoes
logistic growth, with growth rate $\alpha$ and carrying capacity equal to 1.

Based on these processes and also under the assumption that the densities of
full foragers, hungry foragers, and resources (also denoted by $F$, $H$, and
$R$, respectively) are perfectly mixed, they evolve according to the rate
equations:
\begin{align}
  \label{RE}
\begin{split}
\dot F &= \lambda F + \rho  RH - \sigma (1-R)F\,,\\
\dot H &= \sigma (1-R)F - \rho RH - \mu H\,, \\
\dot R &= \alpha R(1-R) -  R(mF+\rho H),\\
\end{split}
\end{align}
where the overdot denotes time derivative.  An important feature is the
assumption that the carrying capacity is set equal to 1.  This means that
when the system is completely occupied by resources there can be no further
generation of the resource.  The approach described below can be generalized
to carrying capacity less than 1, but we take the carrying capacity equal to
1 without loss of generality.

We now outline an event-driven algorithm that mimic these rate equations.
Suppose that the system at a given time consists of $N_F$ full foragers,
$N_H$ hungry foragers, and $N_R$ resource units.  The total number of
particles is $N=N_F+N_H+N_R$.  There are six elemental processes that are
embodied by the rate equations \eqref{RE}:
\newpage
\begin{enumerate}
\item Reproduction:  $F\to 2F$
\item Starvation:  $F\to H$
\item Consumption: $F\to F$ and $R\to 0$
\item Death: $H\to 0$
\item Recruitment: $H\to F$ and $R\to 0$
\item Growth: $R\to 2R$
\end{enumerate}
For specificity, we define ``consumption'' as the consumption of the resource
by a full forager and the maintenance of the forager in its full state, while
``recruitment'' is defined as the consumption of the resource by a hungry
forager and conversion of the latter to the full state.

In an event-driven simulation, one of the above events is picked with the
appropriate probability to be defined below, the selected event is
implemented, and then the time is updated accordingly.  Reading off from the
steps (i)--(vi) listed above, the total rate for any event is proportional to
\begin{equation}
\mathcal{R}=F\big[\lambda +\sigma(1-R)+mR\big]+ H\big[\mu+\rho R\big]+
\alpha R(1-R)\,.
\end{equation}
Here $F=N_F/V$, $H=N_H/V$, and $R=N_R/V$, where $N_i$ is the total number of
particles of type $i$, and $V$ is the total number of lattice sites.  Thus
$F$, $H$, and $R$ are the densities of each type of entity.  The total rate
is defined up to an overall constant, but this constant is immaterial because
the time step is also proportional to this same constant.

In the most simple version of a simulation in the mean-field limit, one
merely tracks the three variables $N_F$, $N_H$, and $N_R$ and updates these
numbers according to the steps outlined below.  However, we want a simulation
that can be directly adapted to treat foragers that diffuse on a lattice.
The algorithm outlined below is based on populating a lattice with the three
types of entities and updating the occupancy on the lattice sites according
to the following event-driven simulation:
\begin{enumerate}
\item Reproduction.  With probability $\lambda F/\mathcal{R}$, create a new
  full forager anywhere in the system equiprobably, consistent with the
  absence of spatial structure.
\item Starvation.  With probability $\sigma(1-R)F/\mathcal{R}$, pick a random
  full forager and change it to hungry.  The factor $1-R$ is the fraction of
  resource-free space.
\item Consumption.  With probability $mFR/\mathcal{R}$, pick and remove a random
  resource unit.
\item Death.  With probability $\mu H/\mathcal{R}$, remove a random
  hungry forager.
\item Recruitment.  With probability $\rho HR/\mathcal{R}$, change a random
  hungry forager into a full forager and remove a random resource unit.
\item Growth.  With probability $\alpha R(1-R)/\mathcal{R}$, create another
  resource unit anywhere in the system.
\end{enumerate}
By explicit construction, the probabilities for the above events sum to 1.
It is important to appreciate that one can perform this simulation either by
populating a lattice with the three types of entities or merely be keeping
track of the three variables $N_F$, $N_H$, and $N_R$.  A good test of the
lattice-based simulation is that it should match that of simulating the above
three variables.

Let's determine the change in the expected number of individuals of each type
in a single event.  For full foragers, the change in its number $N_F$ is
\begin{subequations}
\begin{equation}
  \Delta N_F =\left[F\big(\lambda-\sigma(1-R)\big)+\rho H
    R\right]/\mathcal{R}\,.
\end{equation}
The term proportional to $F$ describes processes in which a full forager is
picked, while the term proportional to $H$ describes processes in which a
hungry forager is picked and converted to a full forager.  Thus the change in
the density of full foragers is
\begin{equation}
\label{dF}
  \Delta F =\frac{\Delta N_F}{V} =\left[F\big(\lambda-\sigma(1-R)\big)+\rho H
    R\right]/V\mathcal{R}\,.
\end{equation}
\end{subequations}
If we take the time step for each event to be
$\Delta t = (V\mathcal{R})^{-1}$, the above reduces to the rate equation
\eqref{RE} for $F$.  Thus for each microscopic update event, the time should
be advanced by $\Delta t = (V\mathcal{R})^{-1}$.

In a similar fashion, the change in the expected density of hungry foragers in
a single event is given by
%\begin{subequations}
%\begin{equation}
%  \Delta N_H =\left[-H(\rho R+\mu)+\sigma F
%    (1-R)\right]/\mathcal{R}\,,
%\end{equation}
%so that the change in the density of hungry foragers simply is
\begin{equation}
\label{dH}
  \Delta H =\left[-H(\rho R+\mu)+\sigma F(1-R)\right]/V\mathcal{R}\,,
\end{equation}
%\end{subequations}
and the change in the density of individual resources in a single event is
%\begin{subequations}
%\begin{equation}
%  \Delta N_R =\left[R\big(\alpha(1-R)\big)-(mF+\rho H)\right]/\mathcal{R}\,,
%\end{equation}
%so that the change in the resource density is
\begin{equation}
\label{dR}
  \Delta R =\left[R\big(\alpha(1-R)\big)-(mF+\rho H)\right]/V\mathcal{R}\,.
\end{equation}
%\end{subequations}
The expressions for $\Delta F/\Delta t$, $\Delta H/\Delta t$, and
$\Delta R/\Delta t$ reproduce the original rate equations \eqref{RE} when the
time step for an elemental update event is $\Delta t = (V\mathcal{R})^{-1}$.
To summarize, pick an event according to the probabilities above, update the
densities according \eqref{dF}, \eqref{dH}, and \eqref{dR}, and then
increment the time by $(V\mathcal{R})^{-1}$.  A simulation that is based on
the above steps should reproduce the predictions of the rate equations
\eqref{RE}.  In the three-variable simulation, keep only $N_F$, $N_H$, and
$N_R$.  In an update, pick an event according to the above probabilities, and
then update $N_F$, $N_H$, and $N_R$ and the time appropriately.

In implementing this algorithm for a lattice-based simulation, several points
deserve emphasis.  First, there is no constraint on the number of foragers on
any site, but if we are describing a harsh environment, the parameters should
be chosen so that this number is not large.  Since the carrying capacity of
the resource has been set to 1, $N_R$ cannot exceed $V$.  However, the number
of resource units on any given site can exceed 1.  

We now generalize the above approach to the situation where the foragers are
diffusing on a lattice.  Suppose that the full and hungry foragers diffuse
with respective diffusion coefficients $D_F$ and $D_H$.  In this case, the
rate equations \eqref{RE} generalize to the reaction-diffusion equations
\begin{align}
  \label{pde}
\begin{split}
\frac{\partial F}{\partial t} &= \lambda F + \rho  RH - \sigma (1-R)F+D_F\nabla^2F\,,\\[0.125in]
\frac{\partial H}{\partial t}  &= \sigma (1-R)F - \rho RH - \mu H+D_H\nabla^2H\,, \\[0.125in]
\frac{\partial R}{\partial t}  &= \alpha R(1-R) -  R(m F+\rho H)\,.\\
\end{split}
\end{align}
Here the densities $F,H,R$ are now functions of space and time,
$F=F(\mathbf{r},t)$ and similarly for $H$ and $R$.

To construct an event-driven simulation that mimic these rate equations, we
have to include two additional processes:
\begin{enumerate}
\item[(vii)] Diffusion of full foragers
\item[(viii)] Diffusion of hungry foragers
\end{enumerate}
In an event-driven simulation, one of the eight possible events is picked
with the appropriate probability (see below), the selected event is
implemented, and the time is updated accordingly.
The total rate for any event is
\begin{align}
\label{R}
\mathcal{R}=&\frac{1}{V}\sum_{\mathbf{r}}\Big\{F(\mathbf{r})\big[\lambda
  +\sigma(1-R(\mathbf{r}))+D_F\big]
+ H(\mathbf{r})\big[\mu+\rho R(\mathbf{r})+D_H\big]\Big.\nonumber \\[-0.05in]
&\Big.\hskip 0.5in
  +\alpha R(\mathbf{r})(1-R(\mathbf{r'}))\Big\}\,\nonumber
  \\[0.1in]
&=F\big[\lambda +\sigma(1-R)+D_F\big]
+ H\big[\mu+\rho R+D_H\big]+\alpha R(1-R)\,,
\end{align}
in which the role of diffusion is now included.  Notice also the presence of
a two-particle correlation in the term $R(\mathbf{r})(1-R(\mathbf{r}'))$,
where $\mathbf{r'}$ is a neighbor of $\mathbf{r}$.  Thus going to the second
line of the above equation is not quite kosher, but below I give a
prescription to deal with this issue in a way that avoids using a correlation
function.

The steps of an event-driven simulation are:
\begin{enumerate}
\item Reproduction.  With probability $\lambda F/\mathcal{R}$, pick a full
  forager at random and let it reproduce.  The offspring should be placed
  close to the parent---either on the same site or on a neighboring site.
\item Starvation.  With probability $\sigma(1-R)F/\mathcal{R}$, pick a random
  full forager that is on a resource-free site and change the forager from
  full to hungry.  Note the two-particle nature of this rule: we need both
  the forager type and the resource state on a given site.  To avoid keeping
  track of sites that are simultaneously resource free and occupied by a full
  forager, a simpler alternative is:
\item[(ii')] Starvation'.  With probability $\sigma F/\mathcal{R}$, pick a
  random full forager.  If the site is resource free, change the forager from
  full to hungry.  If not, nothing happens.
\item Consumption.  With probability $mFR/\mathcal{R}$, pick a random on a
  site that is also by a full forager and remove the resource.  To avoid
  tracking sites that simultaneously contain a resource and a full forager,
  perform the following alternative:
\item[(iii$^\prime$)] Consumption$^\prime$.  With probability
  $R/\mathcal{R}$, pick a random resource.  If the site also contains a
  forager, remove the resource with probability $m$.  Note the implicit
  assumption that the density of foragers is not large because the rate of
  resource removal does not increase if multiple foragers occupy the site.
  If the full forager density is large then the following works:
\item[(iii$^{\prime\prime}$)] Consumption$^{\prime\prime}$.  With probability
  $m F/\mathcal{R}$, pick a random full forager.  If the site also contains a
  resource, remove it.  This prescription works if the forager density is
  small or large (as long as the resource density is less than 1) so it's
  best to use this last alternative.
\item Death.  With probability $\mu H/\mathcal{R}$, pick and remove a random
  hungry forager.
\item Recruitment.  With probability $\rho HR/\mathcal{R}$, pick a random
  hungry forager on a site that also contains a resource and change the
  forager from hungry to full and remove the resource.  This process also
  involves a two-body reaction and a simpler way to implement this two-body
  feature is:
\item[(v$^\prime$)] Recruitment$^\prime$.  With probability
  $\rho H/\mathcal{R}$, pick a random hungry forager.  If the site contains a
  resource, change the forager from hungry to full and remove the resource.
\item Growth.  With probability $\alpha R(1-R)/\mathcal{R}$, pick a random
  neighboring pair of sites in which one contains a resource and the neighbor
  does not.  Create a resource on the site without the resource.  A
  computationally simpler alternative is:
\item[(vi$^\prime$)] Growth$^\prime$.  With probability
  $\alpha R/\mathcal{R}$, pick a random resource unit.  A new resource unit
  is created on one of the neighboring sites with probability $1-m/z$, where
  $z$ is the number of nearest-neighbors and $m$ is the number of these
  neighbors that are occupied by resource.  If the local neighborhood is
  devoid of resources, growth necessarily occurs at one of these neighbors,
  while if the local neighborhood is full of resources, there is no growth.
  This rule ensures that the growth rate is proportional to
  $(1-R)_{\rm local}$.
\item Diffusion.  With probability $D_F\, F/\mathcal{R}$, pick a random full
  forager and move it to a neighboring site.
\item Diffusion.  With probability $D_H\, H/\mathcal{R}$, pick a random
  hungry forager and move it to a neighboring site.
\end{enumerate}
By explicit construction, the probabilities for the above events sum to 1.
Note also that the primed steps are computationally simpler than the
corresponding unprimed steps, but have the possibility of a null step; this
introduces some inefficiency.  A true event-driven simulation avoids all null
steps, but perhaps the computational tradeoff in bookkeeping simplicity is
worth the loss of efficiency in allowing for null steps.  This is somewhat a
matter of user taste and testing.

We now determine the change in the expected number of individuals of each
type at a given site $\mathbf{r}$ in a single event.  For full foragers, the
change in its number $N_F(\mathbf{r})$ at site $\mathbf{r}$ is
\begin{subequations}
\begin{equation}
  \Delta N_F(\mathbf{r})  =\Big\{F(\mathbf{r})\big(\lambda-\sigma(1-R(\mathbf{r}))\big)+
\rho H(\mathbf{r})  R(\mathbf{r})
+D_F\big[\sum_{\mathbf{r}'}F(\mathbf{r}')-zF(\mathbf{r})\big]\Big\}/\mathcal{R}\,.
\end{equation}
The terms proportional to $F$ accounts for processes in which a full forager
is picked, while the term proportional to $H$ accounts for processes in which
a hungry forager is picked.  The terms proportional to $D_F$ account for
nearest-neighbor hopping: $F(\mathbf{r})$ increases because full foragers at
neighboring sites $\mathbf{r}'$ hop to $\mathbf{r}$, while $F(\mathbf{r})$
decreases because full foragers at $\mathbf{r}$ hop to one of the $z$
neighboring sites.  Thus the change in the density of full foragers at site
$\mathbf{r}$ is (after taking the continuum limit for nearest-neighbor
hopping)
\begin{equation}
  \Delta F(\mathbf{r}) =\frac{\Delta N_F(\mathbf{r})}{V} 
  =\left[F(\mathbf{r})\big(\lambda-\sigma(1-R(\mathbf{r}))\big)+\rho H(\mathbf{r})
    R(\mathbf{r})+\nabla^2F(\mathbf{r})\right]/V\mathcal{R}\,.
\end{equation}
\end{subequations}
If we take the time step for each event to be
$\Delta t = (V\mathcal{R})^{-1}$, the above reduces to the partial
differential evolution equation \eqref{pde} for $F(\mathbf{r})$.  In each
microscopic event of the model, the time should be advanced by
$\Delta t = (V\mathcal{R})^{-1}$.

In a similar fashion, the change in the density of hungry foragers is
\begin{equation}
  \Delta H(\mathbf{r}) =\left[-H(\mathbf{r})(\rho R(\mathbf{r})+\mu)+F(\mathbf{r})\sigma (1-R(\mathbf{r}))+\nabla^2H(\mathbf{r})\right]/V\mathcal{R}\,.
\end{equation}
Finally, the change in the resource density is
\begin{equation}
  \Delta R(\mathbf{r}) =\left[R(\mathbf{r})\big(\alpha(1-R(\mathbf{r}))\big)-(F(\mathbf{r})+H(\mathbf{r}))\right]/V\mathcal{R}\,.
\end{equation}
The equations for $\Delta F$, $\Delta H$, and $\Delta R$ would then reproduce
the partial differential equations \eqref{pde} when the time step for an
elemental event is taken to be $\Delta t = (V\mathcal{R})^{-1}$.


In this algorithm, notice that single-particle and two-particle reactions are
treated differently.  In the former case, a particle of the right type is
picked at random.  In the latter case, a random \emph{pair} of particles of
the appropriate types is picked.  Again, there is no constraint on the number
of foragers on any site, but given that we are describing a harsh
environment, the parameters should be chosen so that this number is not
large.  The carrying capacity of the resource has been set to 1, so that the
number of individual resources at any site must be either 0 or 1.

It would be nice if there was a simple way to adjust parameters in the above
lattice simulation so as to recover the mean-field limit.  I describe a
simple modification of the above lattice simulation steps that should recover
the mean-field limit.  We use the same rate as in Eq.~\eqref{R}, but the
actual update steps are slightly modified compared to the rules of the
lattice simulation above.  The differences are highlighted in italics.  I'm
also assuming that the simulation with steps (ii$^\prime$),
(iii$^\prime\prime$), (v$^\prime$), and (vi$^\prime$) are being used.
\begin{enumerate}
\item Reproduction.  With probability $\lambda F/\mathcal{R}$, pick a random
  full forager and let it reproduce.  \emph{Place the offspring anywhere.}
\item Starvation.  With probability $\sigma F/\mathcal{R}$, \emph{change a
    random full forager to hungry with probability $(1-R)$.}
\item Consumption.  With probability $m F/\mathcal{R}$, pick a random full
  forager.  If the site is also occupied by a resource, remove it.
\item Death.  With probability $\mu H/\mathcal{R}$, pick and remove a random
  hungry forager.
\item Recruitment.  With probability $\rho H/\mathcal{R}$, pick a random
  hungry forager. \emph{Change it to full with probability $R$ and also
    remove a random resource.}
\item Growth.  With probability $\alpha R/\mathcal{R}$, pick a random
  resource.  Create a resource anywhere at random with probability $1-R$.
\item $F$ Diffusion.  With probability $D_F\, F/\mathcal{R}$, pick a random full
  forager and move it to a neighboring site.
\item $H$ Diffusion.  With probability $D_H\, H/\mathcal{R}$, pick a random
  hungry forager and move it to a neighboring site.
\end{enumerate}
By explicit construction, the probabilities for the above events sum to 1.

It is important to appreciate that there are multiple ways to simulate the
elemental reaction steps that involve a tradeoff between bookkeeping
complexity and computational efficiency and still reproduce the same behavior
as in the rate equations \eqref{RE} or the reaction-diffusion equations
\eqref{pde} in the limit of a large system.


\end{document}



