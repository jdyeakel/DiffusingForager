\documentclass[11pt]{iopart}
\usepackage{amssymb}
\usepackage{epsfig}
\usepackage{subfigure}
\usepackage{graphicx}
\usepackage{textcomp}
\usepackage{cite}
\usepackage{float}
\makeatletter
\makeatother

\expandafter\let\csname equation*\endcsname\relax
\expandafter\let\csname endequation*\endcsname\relax
\usepackage{amsmath}
\newcommand{\agt}{\mathrel{\raise.3ex\hbox{$>$\kern-.75em\lower1ex\hbox{$\sim$}}}}
\begin{document}

\title{Event-Driven Simulation of Foraging Dynamics}


\begin{abstract}

  We outline an event-driven simulation for the foraging model, which
  involves a resource that renews by logistic growth, as well as two classes
  of foragers---full and hungry.  Full foragers reproduce at a fixed rate and
  are not vulnerable to mortality.  However, a full forager can become hungry
  when resources are scarce; conversely, a hungry forager can become full
  when the resource is abundant.  Hungry foragers do not reproduce and die at
  a fixed rate.
\end{abstract}

%\maketitle

\section{The Model}

We assume that foragers can exist in two discrete states---full and hungry.
Full foragers $F$ are those that have just encountered and consumed a unit of
resource $R$.  On the other hand, a full forager that does not encounter a
resource as it wanders converts to a hungry forager $H$ with rate equal to
the product of a parameter $\sigma$ and the density of non-resources.
Whenever a forager, either full or hungry, encounters a resource, one unit of
the resource is consumed.  If the forager was hungry, it turns into a full
forager with rate $\rho$.  During the time that a forager is hungry, it dies
with mortality rate $\mu$, while full foragers do not experience mortality
risk.  Furthermore, full foragers reproduce with rate $\lambda$.  Finally, we
assume that, in the absence of foragers, the underlying resource undergoes
logistic growth, with growth rate $\alpha$ and carrying capacity equal to
one.

According to these processes and also under the assumption that the densities
of full foragers, hungry foragers, and resources (also denoted by $F$,
$H$, and $R$, respectively) are perfectly mixed, they evolve according to the
rate equations:
\begin{align}
  \label{RE}
\begin{split}
\dot F &= \lambda F + \rho  RH - \sigma (1-R)F\,,\\
\dot H &= \sigma (1-R)F - \rho RH - \mu H\,, \\
\dot R &= \alpha R(1-R) -  R(F+H),\\
\end{split}
\end{align}
where the overdot denotes time derivative.  An important feature is the
assumption that the carrying capacity is set equal to 1.  This means that
when the system is completely occupied by resources there can be no further
generation of the resource.  The approach described below can be generalized
to carrying capacity less than 1, but one can take the carrying capacity
equal to 1 without loss of generality.

We now outline an event-driven algorithm that mimic these rate equations.
Suppose that the system at a given time consists of $N_F$ full foragers,
$N_H$ hungry foragers, and $N_R$ individual resources.  The total number of
particles $N=N_F+N_H+N_R$.  There are six basic processes embodied by the
rate equations \eqref{RE}:
\begin{enumerate}
\item Reproduction:  $F\to F+1$
\item Starvation:  $F\to H$
\item Recruitment: $H\to F$
\item Death: $H\to H-1$
\item Resource growth: $R\to R+1$
\item Resource consumption: $R\to R-1$
\end{enumerate}

In an event-driven simulation, one of the above events is picked with the
appropriate probability (to be defined below), the selected event is
implemented, and then the time is updated accordingly.  The total rate for
any event is proportional to
\begin{equation}
\mathcal{R}=F\big[\lambda +\sigma(1-R)\big]+ H\big[\rho R+\mu\big]+
R\big[\alpha(1-R)+(F+H)\big]\,.
\end{equation}
Here $F=N_F/V$, $H=N_H/V$, and $R=N_R/V$, where $N_i$ is the total number of
particles of type $i$, and $V$ is the total number of lattice sites.  Thus $F$,
$H$, and $R$ are the densities of each type of entity in the system.  The
total rate is defined only up to an overall constant, but this constant is
immaterial because the time step is also proportional to this same constant.

Now we define the steps of an event-driven simulation.  
\begin{enumerate}
\item With probability $\lambda F/\mathcal{R}$ implement the reproduction step.
  This step consists of picking one of the full foragers and allowing it to
  reproduce.
\item With probability $\sigma(1-R)F/\mathcal{R}$ implement the starvation
  step; a full forager is picked and it changes to a hungry forager.
\item With probaibility $\rho HR/\mathcal{R}$ implement the recruitment step;
  a hungry forager is picked and becomes full.
\item With probability $\mu H/\mathcal{R}$ implement the death step; a hungry
  forager is picked and is removed. 
\item With probability $\alpha R(1-R)/\mathcal{R}$ implement the growth step;
  a unit of resource is created.
\item With probability $(F+H)R/\mathcal{R}$ implement the consumption step; a
  unit of resource is removed.
\end{enumerate}
By explicit construction, the probabilities for the above events sum to 1.

We now determine the change in the expected number of individuals of each
type in a single event.  For
full foragers, the change in its number $N_F$ is
\begin{subequations}
\begin{equation}
  \Delta N_F =\left[F\big(\lambda-\sigma(1-R)\big)+\rho H
    R\right]/\mathcal{R}\,.
\end{equation}
The term proportional to $F$ comes from processes in which a full forager is
picked, while the term proportional to $N_H$ comes from processes in which a
hungry forager is picked and converted to a full forager.  Thus the change in
the density of full foragers is
\begin{equation}
  \Delta F =\frac{\Delta N_F}{V} =\left[F\big(\lambda-\sigma(1-R)\big)+H\rho
    R\right]/V\mathcal{R}\,.
\end{equation}
\end{subequations}
Thus if we take the time step for each event to be
$\Delta t = (V\mathcal{R})^{-1}$, the above reduces to the rate equation
\eqref{RE} for $F$.  Thus in each microscopic event of the model, the time
should be advanced by $\Delta t = (V\mathcal{R})^{-1}$.

In a similar fashion, the change in the expected number of hungry foragers in
a single event is given by
\begin{subequations}
\begin{equation}
  \Delta N_H =\left[-H(\rho R+\mu)+F\sigma
    (1-R)\right]/\mathcal{R}\,,
\end{equation}
so that the change in the density of hungry foragers simply is
\begin{equation}
  \Delta H =\left[-H(\rho R+\mu)+F\sigma (1-R)\right]/V\mathcal{R}\,.
\end{equation}
\end{subequations}
Finally, the change in the expected number of individual resources in a
single event is given by
\begin{subequations}
\begin{equation}
  \Delta N_R =\left[R\big(\alpha(1-R)\big)-(F+H)\right]/\mathcal{R}\,,
\end{equation}
so that the change in the resource density is
\begin{equation}
  \Delta R =\left[R\big(\alpha(1-R)\big)-(F+H)\right]/V\mathcal{R}\,.
\end{equation}
\end{subequations}
The equations for $\Delta F$, $\Delta H$, and $\Delta R$ would then reproduce
the original rate equations \eqref{RE} when the time step for an elemental
event is taken to be $\Delta t = (V\mathcal{R})^{-1}$.  To summarize, pick an
event according to the probabilities enumerated above and update the
densities according (3b), (4b), and (5b), and then increment the time by
$(N\mathcal{R})^{-1}$ after each event.

There are several details of the simulation that need to be specified.
First, there is no constraint on the number of foragers on any site, but
given that we are describing a harsh environment, the parameters should be
chosen so that this number is not large.  The carrying capacity of the
resource has been set to 1, so that the number of individual resources at any
site should be either 0 or 1.  Finally, when a full forager reproduces, its
offspring should be placed anywhere.  With these details, a simulation that
is based on the above sprocedure should reproduce the predictions of the rate
equations \eqref{RE}.


We now generalize the above approach to the situation where the foragers are
diffusing on a lattice.  Suppose that the full and hungry foragers diffuse
with respective diffusion coefficients $D_F$ and $D_H$.  In this case, the
rate equations \eqref{RE} generalize to the set of partial differential
equations
\begin{align}
  \label{pde}
\begin{split}
\frac{\partial F}{\partial t} &= \lambda F + \rho  RH - \sigma (1-R)F+D_F\nabla^2F\,,\\[0.125in]
\frac{\partial H}{\partial t}  &= \sigma (1-R)F - \rho RH - \mu H+D_H\nabla^2H\,, \\[0.125in]
\frac{\partial R}{\partial t}  &= \alpha R(1-R) -  R(F+H).\\
\end{split}
\end{align}
Here the densities $F,H,R$ are now functions of space and time,
$F=F(\mathbf{r},t)$ and similarly for $H$ and $R$.

To construct an event-driven simulation that mimic these rate equations, we
have to include two additional processes:
\begin{enumerate}
\item[vii] Full forager diffusion
\item[viii] Hungry forager diffusion
\end{enumerate}
In an event-driven simulation, one of the eight possible events is picked
with the appropriate probability (see below), the selected event is
implemented, and the time is updated accordingly.

The total rate for any event is
\begin{equation}
\mathcal{R}=F\big[\lambda +\sigma(1-R)+D_F\big]+ H\big[\rho R+\mu+D_H\big]+
R\big[\alpha(1-R)+(F+H)\big]\,.
\end{equation}
Now we define the steps of an event-driven simulation.  
\begin{enumerate}
\item With probability $\lambda F/\mathcal{R}$ implement the reproduction step.
  This step consists of picking one of the full foragers and allowing it to
  reproduce.
\item With probability $\sigma(1-R)F/\mathcal{R}$ implement the starvation
  step; a full forager is picked and it changes to a hungry forager.
\item With probaibility $\rho HR/\mathcal{R}$ implement the recruitment step;
  a hungry forager is picked and becomes full.
\item With probability $\mu H/\mathcal{R}$ implement the death step; a hungry
  forager is picked and is removed. 
\item With probability $\alpha R(1-R)/\mathcal{R}$ implement the growth step;
  a unit of resource is created.
\item With probability $(F+H)R/\mathcal{R}$ implement the consumption step; a
  unit of resource is removed.
\item With probability $D_F F/\mathcal{R}$, a full forager moves; pick one
  such forager at random and move it.
\item With probability $D_H H/\mathcal{R}$, a full forager moves; pick one
  such forager at random and move it.
\end{enumerate}
By explicit construction, the probabilities for the above events sum to 1.


We now determine the change in the expected number of individuals of each
type at a given site $\mathbf{R}$ in a single event.  For full foragers, the
change in its number $N_F(\mathbf{R})$ at site $\mathbf{R}$ is
\begin{subequations}
\begin{equation}
  \Delta N_F(\mathbf{R}) =\left[F(\mathbf{R})\big(\lambda-\sigma(1-R(\mathbf{R}))\big)+\rho H
    R\right]/\mathcal{R}\,.
\end{equation}
The term proportional to $F$ comes from processes in which a full forager is
picked, while the term proportional to $N_H$ comes from processes in which a
hungry forager is picked and converted to a full forager.  Thus the change in
the density of full foragers is
\begin{equation}
  \Delta F =\frac{\Delta N_F}{V} =\left[F\big(\lambda-\sigma(1-R)\big)+H\rho
    R\right]/V\mathcal{R}\,.
\end{equation}
\end{subequations}
Thus if we take the time step for each event to be
$\Delta t = (V\mathcal{R})^{-1}$, the above reduces to the rate equation
\eqref{RE} for $F$.  Thus in each microscopic event of the model, the time
should be advanced by $\Delta t = (V\mathcal{R})^{-1}$.

In a similar fashion, the change in the expected number of hungry foragers in
a single event is given by
\begin{subequations}
\begin{equation}
  \Delta N_H =\left[-H(\rho R+\mu)+F\sigma
    (1-R)\right]/\mathcal{R}\,,
\end{equation}
so that the change in the density of hungry foragers simply is
\begin{equation}
  \Delta H =\left[-H(\rho R+\mu)+F\sigma (1-R)\right]/V\mathcal{R}\,.
\end{equation}
\end{subequations}
Finally, the change in the expected number of individual resources in a
single event is given by
\begin{subequations}
\begin{equation}
  \Delta N_R =\left[R\big(\alpha(1-R)\big)-(F+H)\right]/\mathcal{R}\,,
\end{equation}
so that the change in the resource density is
\begin{equation}
  \Delta R =\left[R\big(\alpha(1-R)\big)-(F+H)\right]/V\mathcal{R}\,.
\end{equation}
\end{subequations}
The equations for $\Delta F$, $\Delta H$, and $\Delta R$ would then reproduce
the original rate equations \eqref{RE} when the time step for an elemental
event is taken to be $\Delta t = (V\mathcal{R})^{-1}$.  To summarize, pick an
event according to the probabilities enumerated above and update the
densities according (3b), (4b), and (5b), and then increment the time by
$(N\mathcal{R})^{-1}$ after each event.

There are several details of the simulation that need to be specified.
First, there is no constraint on the number of foragers on any site, but
given that we are describing a harsh environment, the parameters should be
chosen so that this number is not large.  The carrying capacity of the
resource has been set to 1, so that the number of individual resources at any
site should be either 0 or 1.  Finally, when a full forager reproduces, its
offspring should be placed anywhere.  With these details, a simulation that
is based on the above sprocedure should reproduce the predictions of the rate
equations \eqref{RE}.



\end{document}

